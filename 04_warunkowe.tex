\documentclass{mp}

\graphicspath{{04_warunkowe/bayes/},{04_warunkowe/}}

\makeatletter
\newlength{\koza@len}
\newcommand{\koza}[2][]
{
\setlength{\koza@len}{1357pt*\real{#2}}

\begingroup\edef\x
{
	\endgroup
	\noexpand\includegraphics[clip, #1, viewport=0 0 {\the\koza@len} 1108pt]{monty_hall/koza.jpg}
	\noexpand\includegraphics[clip, #1, viewport={\the\koza@len} 0 1357pt 1108pt]{monty_hall/ferrari.jpg}
}
\x
}
\makeatother


\subtitle{Prawdopodobieństwo warunkowe}
\begin{document}
\frame{\titlepage}
\part{Podstawy interpretacji wyników badań medycznych}
\frame{\partpage}
\begin{frame}{Badanie raka}
\newcommand{\eR}{\ensuremath\textcolor{color2}{R}}
\newcommand{\enR}{\ensuremath\textcolor{color3}{R'}}
\newcommand{\eM}{\ensuremath\textcolor{color4}{M}}
\only<1-7>{
\begin{block}{}
Grupa kobiet w wieku 40 lat bierze udział w przesiewowej mammografi, znane są następujące fakty:
\begin{itemize}
\item $\only<3->{P(\eR)=}1\%$ ma raka \only<2>{($\eR$)}
\item $\only<3->{P(\eM|\eR)=}80\%$ kobiet chorych na raka \only<2>{($\eR$) }otrzymuje pozytywny wyniki mammografi \only<2>{($\eM$)}
\item $\only<3->{P(\eM|\enR)=}9{,}6\%$ kobiet zdrowych \only<2>{($\enR$) }otrzymuje pozytywny wynik mammografii \only<2>{($\eM$)}
\end{itemize}
Pewna kobieta otrzymała pozytywny wyniki mammografi. Jakie jest prawdopodobieństwo\only<4->{ $P(\eR|\eM)$}, że ma raka?
\end{block}
}
\begin{tikzpicture}[x=0.1cm,y=.5cm]
\only<5->
{
\fill [color2] (0,0) rectangle (1,1);
\fill [color3] (1,0) rectangle (100, 1);
\draw (0,0) rectangle (100,1);
\node at (50,1.5) {$100\%$};
\node [color2] at ($(2.5,-0.5)$) {$1\%$};
\node [color3] at ($(1,-0.5)+.5*(99,0)$) {$99\%$};
}
\only<6->
{
\draw [dashed] (0,0) -- (0,-2);
\draw [dashed] (1,0) -- (45,-2);
\fill [color4] (0,-2) rectangle ++($(0,-1)+.8*(45,0)$);
\draw [color2] (0,-2) rectangle ++(45,-1);
\node [color2] at ($(0,-1.5)+.5*(45,0)$) {$100\%$};
\node [color4] at ($(0,-3.5)+.5*.8*(45,0)$) {$80\%$};
}
\only<7->
{
\draw [dashed] (1,0) -- (55,-2);
\draw [dashed] (100,0) -- (100,-2);
\fill [color4] (55,-2) rectangle ++($(0,-1)+.096*(45,0)$);
\draw [color3] (55,-2) rectangle ++(45,-1);
\node [color3] at ($(55,-1.5)+.5*(45,0)$) {$100\%$};
\node [color4] at ($(55,-3.5)+.7*.096*(45,0)$) {$9{,}6\%$};
}
\end{tikzpicture}
\only<1>{\footnotesize Źródło: \url{http://www.yudkowsky.net/rational/bayes}}
\let\eR\undefined
\let\enR\undefined
\let\eM\undefined
\note<7>{
	\begin{gather*}
		P(M)=P(M|R)P(R)+P(M|R')P(R')=\frac{8}{10}\frac{1}{100}+\frac{96}{1000}\frac{99}{100}=\frac{10304}{100000} \\
		P(R|M)=\frac{P(R\cap M)}{P(M)}=\frac{P(M|R)P(R)}{P(M)}=\frac{8}{10}\frac{1}{100}\frac{100000}{10304}=\frac{800}{10304}\approx 7{,}76\%
	\end{gather*}
}
\end{frame}
\part{Formalizm matematyczny}
\frame{\partpage}
\begin{frame}{Prawdopodobieństwo warunkowe}
\[ P(A|B)=\frac{P(A\cap B)}{P(B)} \qquad P(B)>0 \]
\only<2>
{
\begin{block}{Twierdzenie o prawdopodobieństwie warunkowym}
Prawdopodobieństwo warunkowe spełnia aksjomaty Kołmogorowa
\end{block}
}
\only<3>
{
	\begin{block}{Twierdzenie}
	\begin{align*}
		P(A_1 \cap A_2)= & P(A_2|A_1)P(A_1) \\
		P(A_1 \cap A_2 \cap A_3)=& P(A_3|A_1\cap A_2)P(A_2|A_1)P(A_1) \\
		P(A_1 \cap A_2 \cap A_3\cap A_4)=& P(A_4|A_1\cap A_2\cap A_3) P(A_3|A_1\cap A_2)P(A_2|A_1)P(A_1) \\
		\ldots
	\end{align*}
	\end{block}
}
\only<4-5>
{
\begin{block}{\only<4>{Twierdzenie o prawdopodobieństwie całkowitym (zupełnym)}\only<5>{Twierdzenie Bayesa}}
Niech:
\begin{itemize}
\item $A_1, A_2, \ldots \in \mathcal{A}$
\item $\forall i\neq j\colon A_i\cap A_j\equiv \emptyset$
\item $A_1\cup A_2\cup \ldots \equiv \Omega$
\item $\forall i\colon P(A_i)>0$
\end{itemize}
wtedy:
\[
\only<4>{P(B)=P(B|A_1)P(A_1)+P(B|A_2)P(A_2)+\ldots}
\only<5>{P(A_i|B)=\frac{P(B|A_i)P(A_i)}{P(B|A_1)P(A_1)+P(B|A_2)P(A_2)+\ldots}}
\]
\end{block}
}
\end{frame}
\begin{frame}{Thomas Bayes}
\begin{center}
\includegraphics[height=.7\textheight]{Thomas_Bayes.png}\\
Thomas Bayes (1702--1761)
\end{center}
{\tiny \url{http://commons.wikimedia.org/wiki/File:Thomas_Bayes.gif} domena publiczna}
\end{frame}
\begin{frame}{Niezależność pary zdarzeń}
\[ P(A|B)=P(A) \iff P(A\cap B)=P(A)P(B) \]
\only<2>
{
\begin{block}{Twierdzenie}
Dla dowolnych zdarzeń $A$ i $B$ nad przestrzenią $\Omega$:
\begin{enumerate}
\item $A$ i $\Omega$ są niezależne
\item $A$ i $\emptyset$ są niezależne
\item $\Omega$ i $\emptyset$ są niezależne
\item jeżeli $A$ i $B$ są niezależne, to $A$ i $B'$ także są niezależne
\end{enumerate}
\end{block}
}
\end{frame}
\begin{frame}{Niezależność większej liczby zdarzeń}
Zdarzenia ze zbioru co najwyżej przeliczalnego $\set{A}=\{A_1, A_2, \ldots\}$ są (wzajemnie) niezależne wtedy, i tylko wtedy gdy dla dowolnego, skończonego $\set{B}\subseteq \set{A}$ zachodzi:
\[ \prod_{A_i\in \set{B}} P(A_i)=P\left(\bigcap_{A_i\in \set{B}}A_i\right)\]
\only<2>
{
	\begin{block}{Przypadki szczególne}
	\begin{itemize}
	\item niezależność parami
	\item niezależność trójkami
	\item \ldots
	\end{itemize}
	\end{block}
}
\only<3>
{
	\begin{block}{Twierdzenie}
	Dla niezależnych zdarzeń $A, B, C$ zachodzi:
	\begin{enumerate}
	\item $A\cup B$ i $C$ są niezależne
	\item jeżeli $P(C)>0$, to $P(A\cap B|C)=P(A|C)P(B|C)$
	\end{enumerate}
	\end{block}
}
\end{frame}
\part{Jak działa filtr antyspamowy, czyli dzień z~życia nauczyciela akademickiego}
\frame{\partpage}
\begin{frame}{Filtr antyspamowy}
\only<1,7>
{
	\begin{minipage}{.50\textwidth}
\begin{block}{List 1}
Proszę przepisać ocenę.\\
Student Wytrwały
\end{block}
\end{minipage}
\hfill
\begin{minipage}{.45\textwidth}
\includegraphics[width=\textwidth]{Sleeping_students.jpg}\\
\end{minipage}
\[ P(\cmark | \text{prosić, przepisać, ocena, student})=1 \]
}
\only<2,8>
{
	\begin{minipage}{.50\textwidth}
\begin{block}{List 2}
Proszę wystawić oceny.\\
Profesor Ważny
\end{block}
\end{minipage}
\hfill
\begin{minipage}{.45\textwidth}
\includegraphics[width=\textwidth]{scientist-28748_1280.png}\\
\end{minipage}
\[ P(\cmark | \text{prosić, wystawić, ocena, profesor})=1 \]
}
\only<3,9>
{
	\begin{minipage}{.50\textwidth}
\begin{block}{List 3}
Proszę o ocenę 3.\\
Student Panda
\end{block}
\end{minipage}
\hfill
\begin{minipage}{.45\textwidth}
\includegraphics[width=\textwidth]{Panda_Cub_from_Wolong,_Sichuan,_China.JPG}\\
\end{minipage}

\[ P(\xmark | \text{prosić, ocena, student})=1 \]
}
\only<4,10>
{
	\begin{minipage}{.50\textwidth}
\begin{block}{List 4}
Proszę przepisać szafę
\end{block}
\end{minipage}
\hfill
\begin{minipage}{.45\textwidth}
\begin{centering}
\includegraphics[height=.4\textheight]{Eichenschrank_Oberschwaben.jpg}\\
\end{centering}
\end{minipage}
\[ P(\xmark | \text{prosić, przepisać, szafa})=1 \]
}
\only<5,11>
{
	\begin{minipage}{.50\textwidth}
\begin{block}{List 5}
Wystaw profesora.\\
Student Tajniak
\end{block}
\end{minipage}
\hfill
\begin{minipage}{.45\textwidth}
\begin{centering}
\def\svgwidth{.8\textwidth}
\input{04_warunkowe/bayes/Spy_silhouette.pdf_tex}
\end{centering}
\end{minipage}
\[ P(\xmark | \text{wystawić, profesor, student})=1 \]
}

\only<6,12,22>
{
	\begin{minipage}{.45\textwidth}
	\begin{block}{List 6\only<22>{\cmark}}
	Wystaw oceny!!!
	\end{block}
	\only<-21>
	{
	\begin{align*}
	P(\cmark |& \text{wystawić, ocena})=?\\
	P(\xmark |& \text{wystawić, ocena})=?
	\end{align*}
	}
	\end{minipage}
	\hfill
	\begin{minipage}{.45\textwidth}
	\begin{block}{List 7\only<22>{\xmark}}
	Wystaw studenta
	\end{block}
	\only<-21>
	{
	\begin{align*}
	P(\cmark |& \text{wystawić, student})=?\\
	P(\xmark |& \text{wystawić, student})=?
	\end{align*}
	}
	\end{minipage}
}
\only<7-21>
{
\small
\begin{tabular}{c|c|cccccccr}
list & & prosić & przepisać & wystawić & ocena & student & profesor &szafa & \\
\hline
\only<-14>
{
\only<7-> {1 & \cmark & 1 & 1 & 0 & 1 & 1 & 0 & 0& }
\only<8->  {\\2 & \cmark & 1 & 0 & 1 & 1 & 0 & 1 & 0& }
\only<9->  {\\3 & \xmark & 1 & 0 & 0 & 1 & 1 & 0 & 0& }
\only<10-> {\\4 & \xmark & 1 & 1 & 0 & 0 & 0 & 0 & 1& }
\only<11-> {\\5 & \xmark & 0 & 0 & 1 & 0 & 1 & 1 & 0& }
}
\only<15-> {1, 2 & \cmark& 2 & 1 & 1 & 2 & 1 & 1 & 0&\\
3, 4, 5& \xmark & 2 & 1 & 1 & 1 & 2 & 1 & 1&}
\only<12-> {\\\hline
	      6 & \only<-18>{?}\only<19->{\cmark} 	 & 0 & 0 & 1 & 1 & 0 & 0 & 0&
	    \\7 & \only<-20>{?}\only<21->{\xmark}	 & 0 & 0 & 1 & 0 & 1 & 0 & 0& }
\end{tabular}
\only<13-15>
{
\begin{align*}
P(\cmark |& \text{wystawić, ocena})=\frac{P(\cmark)P(\text{wystawić, ocena}|\cmark)}{P(\text{wystawić, ocena})}
\only<14->{\\&\sim P(\cmark)P(\text{wystawić, ocena}|\cmark)
\only<15->{\approx P(\cmark)P(\text{wystawić}|\cmark)P(\text{ocena}|\cmark)}
} \\
P(\xmark |& \text{wystawić, ocena})=\frac{P(\xmark)P(\text{wystawić, ocena}|\xmark)}{P(\text{wystawić, ocena})}
\only<14->{\\&\sim P(\xmark)P(\text{wystawić, ocena}|\xmark)
\only<15->{\approx P(\xmark)P(\text{wystawić}|\xmark)P(\text{ocena}|\xmark)}
}
\end{align*}
}
\only<16-19>
{
\begin{align*}
P(\cmark)P(\text{wystawić}|\cmark)P(\text{ocena}|\cmark) = \frac{2}{5}\cdot\only<17->{\frac{1}{2}\cdot\frac{2}{2}=\only<18->{\frac{1}{5}} } \\
P(\xmark)P(\text{wystawić}|\xmark)P(\text{ocena}|\xmark) = \frac{3}{5}\cdot\only<17->{\frac{1}{3}\cdot\frac{1}{3}=\only<18->{\frac{1}{15}} }
\end{align*}
}
\only<20->
{
	\begin{align*}
	P(\cmark)P(\text{wystawić}|\cmark)P(\text{student}|\cmark) = \ldots\only<21->{=\frac{1}{10}} \\
	P(\xmark)P(\text{wystawić}|\xmark)P(\text{student}|\xmark) = \ldots\only<21->{=\frac{2}{15}}
	\end{align*}
}
}
\end{frame}
\begin{frame}{Credits}
Rysunki i zdjęcia pochodzą z następujących źródeł:
\begin{description}
\item[List 1]  \url{http://commons.wikimedia.org/wiki/File:Sleeping_students.jpg} domena publiczna
\item[List 2] \url{http://pixabay.com/p-28748/} domena publiczna
\item[List 3] \url{http://commons.wikimedia.org/wiki/File:Panda_Cub_from_Wolong,_Sichuan,_China.JPG} domena publiczna
\item[List 4] \url{http://commons.wikimedia.org/wiki/File:Eichenschrank_Oberschwaben.jpg} domena publiczna
\item[List 5] \url{http://commons.wikimedia.org/wiki/File:Spy_silhouette.svg} CC BY-SA 3.0 by \emph{Itzhak Baum} aka \emph{Setreset}
\end{description}
\end{frame}
\end{document}
\begin{frame}{Monty Hall}
\begin{minipage}{.30\textwidth}
\includegraphics<1>[width=\textwidth]{monty_hall/koza.jpg}
\includegraphics<2->[width=\textwidth]{monty_hall/stage-406306_1280.jpg}
\only<3->{\koza[scale=.08]{0.66}}
\end{minipage}
\hfill
\begin{minipage}{.30\textwidth}
\includegraphics<1>[width=\textwidth]{monty_hall/ferrari.jpg}
\includegraphics<2-3>[width=\textwidth]{monty_hall/stage-406306_1280.jpg}
\only<4->{\colorbox{green}{\includegraphics[width=\textwidth]{monty_hall/stage-406306_1280.jpg}}}
\only<3->{\koza[scale=.08]{0.66}}
\end{minipage}
\hfill
\begin{minipage}{.30\textwidth}
\includegraphics<1>[width=\textwidth]{monty_hall/koza.jpg}
\includegraphics<2-4>[width=\textwidth]{monty_hall/stage-406306_1280.jpg}
\includegraphics<5>[width=\textwidth]{monty_hall/curtain-152112_1280.png}
\only<3-4>{\koza[scale=.08]{0.66}}
\only<5->{\koza[scale=.08]{1}}
\end{minipage}
\end{frame}
\begin{frame}{Credits}
\begin{itemize}
\item koza: Photographer: Armin Kübelbeck, CC-BY-SA, Wikimedia Commons, \url{http://creativecommons.org/licenses/by-sa/3.0/}, \url{http://commons.wikimedia.org/wiki/File:Hausziege_04.jpg}
\item ferrari: domena publiczna \url{http://commons.wikimedia.org/wiki/File:Ferrari_California_Paris_August_2010.jpg}
\item kurtyny: domena publiczna \url{http://pixabay.com/pl/etap-kurtyna-teatr-czerwony-406306/} \url{http://pixabay.com/pl/kurtyna-etap-teatr-filmy-kino-152112/}
\end{itemize}
\end{frame}

% \newlength{\koza}
% \setlength{\koza}{1357pt*\real{.8}}
% 
% \begingroup\edef\x
% {
% 	\endgroup
% 	\noexpand\includegraphics[clip, scale=.1, viewport=0 0 {\the\koza} 1108pt]{monty_hall/koza.jpg}
% 	%\noexpand\includegraphics[clip, scale=.1, viewport={\the\ferrari} 0 3676pt 1108pt]{monty_hall/ferrari.jpg}
% 	\noexpand\includegraphics[clip, scale=.1, viewport={\the\koza} 0 1357pt 1108pt]{monty_hall/ferrari.jpg}
% }
% \x
%\def\x{\includegraphics[scale=.1,clip,viewport=0 0 }\expandafter\x\the\koza 1108]{monty_hall/koza.jpg}}
%\setlength{\kozain}{\koza*\ratio{1}{72}}
%\begin{tikzpicture}
%\node (koza) {\x};
%%\node (ferrari) [xshift=\koza]  {\includegraphics[scale=.1,clip,viewport=\ferrari 0 3676 1108]{monty_hall/ferrari.jpg}};
%\end{tikzpicture}

