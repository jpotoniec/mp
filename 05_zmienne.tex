\documentclass{mp}
\renewcommand{\d}[1]{\ensuremath\,\text{d}#1}
\graphicspath{{03_aksjomatyka/}}
\subtitle{Zmienne losowe}
\begin{document}
\frame{\titlepage}
\begin{frame}{Zmienna losowa}
\begin{center}
\begin{tikzpicture}[x=1cm,y=1cm]
\coordinate (A) at (0,0);
\coordinate (A1) at ($(A)+(.5,.5)$);
\coordinate (A2) at ($(A)+(-.5,.5)$);
\coordinate (A3) at ($(A)+(.5,0)$);
\coordinate (A4) at ($(A)+(-.5,0)$);
\coordinate (A5) at ($(A)+(0,-.5)$);
\coordinate (B) at (6,0);
\coordinate (B1) at ($(B)+(.5,.5)$);
\coordinate (B2) at ($(B)+(-.5,.5)$);
\coordinate (B3) at ($(B)+(-.5,-.5)$);
\coordinate (B4) at ($(B)$);
\draw[set,color2] (A) circle [x radius=1.5cm,y radius=2cm];
\foreach \p in {(A1),(A2),(A3),(A4),(A5)}
	\fill[color2] \p circle (.05);
\draw[set,color3] (B) circle [x radius=1.5cm,y radius=2cm];
\foreach \p in {(B1),(B2),(B3),(B4)}
	\fill[color3] \p circle (.05);
	%(.5,.5) (2,1) (4,1) (6,0)
\draw[->,alt={<1>{color=black}{color=gray!50!white}}] (A1) .. controls +(1.5,.5) and +(-2,1) .. (B4);
\draw[->,alt={<1>{color=black}{color=gray!50!white}}] (A2) .. controls +(1.5,.5) and +(-2,1) .. (B2);
\draw[->,alt={<1>{color=black}{color=gray!50!white}}] (A3) .. controls +(1.5,.5) and +(-2,0) .. (B3);
\draw[->,alt={<1>{color=black}{color=gray!50!white}}] (A4) .. controls +(1.5,.5) and +(-2,1) .. (B4);
\draw[->,alt={<1>{color=black}{color=gray!50!white}}] (A5) .. controls +(1.5,-.5) and +(1,-2) .. (B1);
\node at ($(3,1.8)$) {\only<1>{$f(\cdot)$}\only<2>{$f[\cdot]$}\only<3>{$f^{-1}[\cdot]$}\only<4->{$X(\cdot)$}};
\path[name path=p1,rotate=45] (A4) -- ++(3,0);
\path[name path=p2,rotate=135] (B4) -- ++(3,0);
\only<2>
{
	\draw[name path=a,color4] ($.5*(A1)+.5*(A2)$) circle [x radius=.7,y radius=0.3];
	\draw[name path=b,rotate=-45,color4] ($.5*(B2)+.5*(B4)$) circle [x radius=.7,y radius=0.3];
	\draw[name intersections={of=p1 and a,name=i},name intersections={of=p2 and b,name=j},->] (i-1) .. controls +(1.5,.5) and +(-2,1) .. (j-1);
}
\only<3>
{
	\draw[name path=a,rotate=30,color4] ($.33*(A1)+.33*(A2)+.33*(A4)$) circle [x radius=1,y radius=0.5];
	\draw[name path=b,rotate=-45,color4] ($.5*(B2)+.5*(B4)$) circle [x radius=.7,y radius=0.3];
	\draw[name intersections={of=p1 and a,name=i},name intersections={of=p2 and b,name=j},<-] (i-1) .. controls +(1.5,.5) and +(-2,1) .. (j-1);
}
\only<4->
{
	\node at ($(A)+(0,2.3cm)$) {$\Omega$};
	\node at ($(B)+(0,2.3cm)$) {$\R$};
}
\end{tikzpicture}
\only<5>
{
	\[ \forall x\in\R\colon X^{-1}[(-\infty,x)]\in\mathcal{A}\]	
}
\end{center}
\end{frame}
\begin{frame}{Kostki}
	\begin{center}
		\begin{tikzpicture}[every token/.style={dice token}]
			\only<1-2>
			{
				\node (d1) [dice,tokens=1] 		{};
				\node (d2) [dice,tokens=2,right=of d1]  {};
				\node (d3) [dice,tokens=3,right=of d2]  {};
				\node (d4) [dice,tokens=4,right=of d3]  {};
				\node (d5) [dice,tokens=5,right=of d4]  {};
				\node (d6) [dice,tokens=6,right=of d5]  {};
				\node (v1) [below=of d1] {\only<1>{1}\only<2>{100}};
				\node (v2) [below=of d2] {\only<1>{2}\only<2>{200}};
				\node (v3) [below=of d3] {\only<1>{3}\only<2>{300}};
				\node (v4) [below=of d4] {\only<1>{4}\only<2>{400}};
				\node (v5) [below=of d5] {\only<1>{5}\only<2>{500}};
				\node (v6) [below=of d6] {\only<1>{6}\only<2>{600}};
				\foreach \x/\y in {d1/v1,d2/v2,d3/v3,d4/v4,d5/v5,d6/v6}
					\draw[->] (\x.south) -- (\y.north);
			}
			\only<3>
			{
				\node (d1) [dice,tokens=1] 		{};
				\node (d3) [dice,tokens=3,right=of d1]  {};
				\node (d5) [dice,tokens=5,right=of d3]  {};
				\node (d2) [dice,tokens=2,right=of d5]  {};
				\node (d4) [dice,tokens=4,right=of d2]  {};
				\node (d6) [dice,tokens=6,right=of d4]  {};
				\node (v0) [below=of d3] {0};
				\node (v1) [below=of d4] {1};
				\foreach \x/\y in {d1/v0,d2/v1,d3/v0,d4/v1,d5/v0,d6/v1}
					\draw[->] (\x.south) -- (\y.north);
			}
		\end{tikzpicture}
	\end{center}
\end{frame}
\begin{frame}{Typy zmiennych losowych}
\begin{center}
\begin{tikzpicture}
\node (w) [diagram] {możliwe wartości}
	child { node (det) [diagram,align=left,below left=of w.center] {typu skokowego\\dyskretne} edge from parent node[left] {$\leq\left|\mathbb{N}\right|$}}
	child { node (los) [diagram,align=left,below right=of w.center] {typu ciągłego} edge from parent node[left] {$=\left|\R\right|$}};
\end{tikzpicture}
\end{center}
\end{frame}
\begin{frame}{Rozkład prawdopodobieństwa zmiennej losowej}
\begin{itemize}
\item<+-> $ P_X(S)=P(\{\omega: X(\omega)\in S\}) $
\item<+-> $ P(X<a)=P(\{\omega: X(\omega)<a\}) $
\item<+-> $ P(X=a)=P(\{\omega: X(\omega)=a\}) $
\item<+-> $ P(a<X\leq b)=P(\{\omega: a<X(\omega)\leq b\}) $
\end{itemize}
\end{frame}
\newcommand{\dice}[1]{\tikz{\node [dice,tokens=#1] {};}}
\begin{frame}{Kostki}
\begin{tabular}{l|c|c|c|c|c|c}
%& \tikz{\node [dice,tokens=1] {};} & \tikz{\node [dice,tokens=2] {};} & \tikz{\node [dice,tokens=3] {};} & \tikz{\node [dice,tokens=4] {};} & \tikz{\node [dice,tokens=5] {};} & \tikz{\node [dice,tokens=6] {};} \\\
& \dice{1} & \dice{2} & \dice{3} & \dice{4} & \dice{5}  & \dice{6} \\
\only<+->{$\Omega$ & $\omega_1$ & $\omega_2$ & $\omega_3$ & $\omega_4$ & $\omega_5$ & $\omega_6$ \\}
\only<+->{
$x$ & 1 & 2 & 3 & 4 & 5 & 6 \\}
\only<+->{$P(X=x)$ & $\frac{1}{6}$ & $\frac{1}{6}$ & $\frac{1}{6}$ & $\frac{1}{6}$ & $\frac{1}{6}$ & $\frac{1}{6}$\\}
\end{tabular}
\end{frame}
\begin{frame}{Dystrybuanta}
\begin{center}
\begin{tikzpicture}[x=1cm,y=5cm]
\draw[->] (-3,0)--(9,0);
\draw[->] (0,-.2)--(0,1.2);
\foreach \x in {1,2,3,4,5,6}
	\node at (\x,-.05) {$\x$};
\foreach \y in {1,2,3,4,5,6}
	\draw ($(-.1,0)+\y/6*(0,1)$) -- ++(.2,0);
\foreach \y in {1,2,3,4,5}
	\node at ($(-.3,0)+\y/6*(0,1)$) {$\frac{\y}{6}$};
\node at ($(-.3,1)$) {$1$};
\node at (9,-.05) {$x$};
\only<2->
{
	\begin{scope}[color2]
\foreach \x in {1,2,3,4,5,6}
{
	\fill ($(\x,0)+1/6*(0,1)$) circle (1pt);
	\draw ($(\x,0)$) circle (1pt);
	\draw ($(\x,0)+(-1,0)+(1pt,0)$) -- ++($(1,0)-(2pt,0)$);
}
\draw (-3,0)--(0,0) (6,0)--(9,0);
\node at (-.9,1.2) {$P(X=x)$};
\end{scope}
}
\only<3>
{
\begin{scope}[color4]
\draw[thick] (-3,0) -- (1,0);
\foreach \x in {1,2,3,4,5,6}
{
	\draw[thin] ($(\x,0)+\x/6*(0,1)$) circle (1pt);
	\draw[thick] ($(\x,0)+\x/6*(0,1)+(1pt,0)$) -- ++(1,0);
}
\draw[thick] (7,1)--(9,1);
\node at (0.5,1.2) {$F(x)$};
\end{scope}
}
\end{tikzpicture}
\end{center}
\end{frame}
\begin{frame}{Dystrybuanta formalnie}
\[ F_X(x)=P(X<x) \qquad x\in\R \]
\begin{block}{Właściwości}
\begin{enumerate}
\item \[ F_X(-\infty)=0 \qquad F_X(\infty)=1 \]
\item \[ \forall x\in\R\colon F_X(x)\geq0 \]
\item \[ \forall x_1<x_2\in\R\colon F_X(x_1)\leq F_X(x_2) \]
\item \[ \forall x_0\in\R\colon \lim_{x\to x_0^{-}} F_X(x)=F_X(x_0) \]
\end{enumerate}
\end{block}
\end{frame}
\begin{frame}{Zmienne losowe typu skokowego (dyskretne)}
\begin{align*}
 P(X\in & S)=1 \qquad S=\{x_1,x_2,\ldots\} \\
\only<2->{\forall x_i\in & S\colon P(X=x_i)=p_i>0 \\}
\only<3->{\forall x\in &\R\colon F(x)=\sum_{x_i<x} p_i}
\end{align*}
\end{frame}
\begin{frame}{Funkcje zmiennej losowej}
	\[ Y=f(X) \qquad P(Y=y)=P(f(X)=y)=P(X\in f^{-1}(y)) \]
	\only<2>{\[Y=(X-3)^2 \qquad X=\tikz[baseline]{\node [dice] {?};}\]}
	% P(Y=0)=P(X=3) P(Y=1)=P(X=4 | X=2) P(Y=4)=P(X=5 | X=1) P(Y=9)=P(X=6)
\end{frame}
\begin{frame}{Wartość średnia}
\[ m=\mu=E(X)=EX=\sum_{x_i} x_ip_i=x_1p_1+x_2p_2+\ldots \]
\only<2-3>
{
	\begin{center}
	\dice{1} \dice{2} \dice{3} \dice{4} \dice{5} \dice{6}\\
	\only<3->{\dice{1} \dice{2} \dice{3} \dice{6} \dice{6} \dice{6}}
	\end{center}
}
\begin{block}<4->{Twierdzenie}
\only<4>
{
	\[\forall a\in\R\colon \left( P(A=a)=1 \to EA=a \right) \]
}
\only<5->
{
Dla dowolnej stałej $a\in\R$ i dowolnych zmiennych losowych $X$ i $Y$ zachodzi:
\begin{enumerate}
\item<5-> $E(aX)=aEX$
\item<6-> $E(X+a)=EX+a$
\item<7-> $E(X-\mu)=0$
\item<8-> $E(X+Y)=EX+EY$ %dowód wymaga zmiennych losowych dwuwymiarowych
\end{enumerate}
}
\end{block}
\end{frame}
\begin{frame}{Wariancja i odchylenie standardowe}
\begin{align*}
 \sigma^2=D^2(X)&=D^2X=E(X-EX)^2 =\sum_{x_i}(x_i-\mu)^2p_i \\ & =
 (x_1-\mu)^2p_1+(x_2-\mu)^2p_2+\ldots
\end{align*}
\only<2-3>
{
	\begin{center}
	\dice{1} \dice{2} \dice{3} \dice{4} \dice{5} \dice{6}\\
	\only<3->{\dice{1} \dice{2} \dice{3} \dice{6} \dice{6} \dice{6}}
	\end{center}
}
\begin{block}<4->{Twierdzenie}
	Dla dowolnego $a\in\R$ i dowolnej zmiennej losowej $X$
\begin{enumerate}
\item<4-> $D^2(X+a)=D^2X $
\item<5-> $D^2X=E^2X-EX^2 $
\item<6-> $(a\neq \mu)\to (D^2X<E(X-a)^2) $
\item<7-> $P(A=a)=1 \to D^2A=0 $
\item<8-> $D^2(aX)=a^2D^2(X) $
\end{enumerate}
\end{block}
%TODO: tu powinien być jakiś rysunek podpierający własności ze skalowaniem i przesuwaniem średnich i wariancji
\end{frame}
\begin{frame}{Nierówność Czebyszewa}
\[\sigma^2<\infty \to \forall k>0\colon P(\left|X-\mu\right|\geq k\sigma)\leq \frac{1}{k^2} \]
\end{frame}

\begin{frame}{Translacja i skalowanie}
\begin{center}
\begin{tikzpicture}[x=.5cm,y=5cm]
\draw [->] (-5,0)--(11,0) node at ++(0,-.05) {$k$};
\draw [->] (0,-.1)--(0,.6) node at ++(-2,0) {$P(X=k)$};
\draw (1,.02)-- ++(0,-.04) node at ++(0,-.04) {1};
\draw (.15,.1)-- ++(-.3,0) node at ++(-.4,0) {$0{,}1$};
\newcommand{\ex}{5.5}
\newcommand{\dx}{1.57}
\newcommand{\h}{.3}
\newcommand{\dist}{4}
\newcommand{\scale}{.5}
\begin{scope}[onslide=<-3>{color4},onslide=<4->{color4!25}]
	\only<2->{\draw (\ex,0)-- ++(0,\h);}
	\only<3->{\fill[semitransparent] ($(\ex,0)-(\dx,0)$) rectangle ++($2*(\dx,0)+(0,\h)$);}
\end{scope}
\only<5>
{
\begin{scope}[color4]
	\draw ($(\ex,0)-(\dist,0)$)-- ++(0,\h);
	\fill[semitransparent] ($(\ex,0)-(\dx,0)-(\dist,0)$) rectangle ++($2*(\dx,0)+(0,\h)$);
\end{scope}
}
\only<7>
{
\begin{scope}[color4]
	\draw ($\scale*(\ex,0)$)-- ++(0,\h);
	\fill[semitransparent] ($\scale*(\ex,0)-\scale*(\dx,0)$) rectangle ++($2*\scale*(\dx,0)+(0,\h)$);
\end{scope}
}
%%rozkład bernouliego n=10 p=.6
%\foreach \x/\p in {0/0.0001,1/0.0016,2/0.0106,3/0.0425,4/0.1115,5/0.2007,6/0.2508,7/0.2150,8/0.1209,9/0.0403,10/0.0060}
%rozkład bernouliego n=10 p=.55
\foreach \x/\p in {0/0.0003,1/0.0042,2/0.0229,3/0.0746,4/0.1596,5/0.2340,6/0.2384,7/0.1665,8/0.0763,9/0.0207,10/0.0025}
{
	\fill[alt=<-3>{color2}{color2!50}] (\x,\p) circle (1pt);
	\fill[alt=<4-5>{color2}{invisible}] ($(\x,\p)-(\dist,0)$) circle (1pt);
	\fill[alt=<6-7>{color2}{invisible}] ($\scale*(\x,0)+(0,\p)$) circle (1pt);
%	\fill[color3] ($(\x,\p)-(3,0)$) circle (1pt);
%	\fill[color4] ($.5*(\x,0)+(0,\p)$) circle (1pt);
}
% \draw[thick,color2] plot[domain=-10:-2] (\x,0) plot[domain=-2:2] (\x,{0.11*\x/4+3*0.11/2}) plot[domain=2:4] (\x,{4*0.11/\x}) plot[domain=4:9] (\x,0);
% \draw[thick,color3] plot[domain=-2:2] ({2*\x},{(0.11*\x/4+3*0.11/2)/2}) plot[domain=2:4] ({2*\x},{(4*0.11/\x)/2});
% \draw[dashed,color4] (1.02,0) -- ++(0,0.5);
% \draw[dashed,color4] ($(1.02,0)-(1.58,0)$) -- ++(0,0.5);
% \draw[dashed,color4] ($(1.02,0)+(1.58,0)$) -- ++(0,0.5);
%\draw[thick,color3] plot[domain=-10:-2] (2*\x,0) plot[domain=-2:2] (2*\x,{0.11*\x/4+3*0.11/2}) plot[domain=2:4] (2*\x,{4*0.11/\x}) plot[domain=4:9] (2*\x,0);
%\draw[thick,color4] (-10,0)--(-2,0) plot[domain=-2:2] ({2*\x},{0.11*\x/4+3*0.11/2}) plot[domain=2:4] ({2*\x},{4*0.11/\x}) (4,0)--(9,0);
\end{tikzpicture}
\end{center}
\end{frame}

\end{document}
\begin{frame}{Niezależność zmiennych losowych}
\begin{align*}
P(X<x \land Y<y)&=P(\{\omega: X(\omega)<x\}\cap\{\omega: Y(\omega)<y\})\\
&=P(\{\omega: X(\omega)<x\})\cdot P(\{\omega: Y(\omega)<y\})=P(X<x)P(Y<y) 
\end{align*}
\end{frame}

\begin{frame}{Momenty zmiennych losowych}
\begin{block}{Moment rzędu $l$ względem punktu $c$}
\[ E(X-c)^l=\begin{cases} \sum_{x_i} x_i^lp_i & $X$\text{ typu skokowego} \\
\int_{-\infty}^\infty x^lf(x)\d{x} & $X$\text{ typu ciągłego}
\end{cases} \]
\end{block}
\end{frame}


\begin{block}<2->{Typu ciągłego}
\begin{align*}
\forall u\in\R\colon & f(u)\geq 0 \\
\forall x\in\R\colon & F(x)=\int_{-\infty}^x f(u)\d{u} \\
\end{align*}
\end{block}
