\documentclass{mp}
%\graphicspath{{05_zmienne/}}
\subtitle{Momenty zmiennych losowych}
\begin{document}
\frame{\titlepage}

\begin{frame}{Wartość średnia}
\[ m=\mu=E(X)=EX=\sum_{x_i} x_ip_i=x_1p_1+x_2p_2+\ldots \]
\only<2-3>
{
	\begin{center}
	\dice{1} \dice{2} \dice{3} \dice{4} \dice{5} \dice{6}\\
	\only<3->{\dice{1} \dice{2} \dice{3} \dice{6} \dice{6} \dice{6}}
	\end{center}
}
\begin{block}<4->{Twierdzenie}
\only<4>
{
	\[\forall a\in\R\colon \left( P(A=a)=1 \to EA=a \right) \]
}
\only<5->
{
Dla dowolnej stałej $a\in\R$ i dowolnych zmiennych losowych $X$ i $Y$ zachodzi:
\begin{enumerate}
\item<5-> $E(aX)=aEX$
\item<6-> $E(X+a)=EX+a$
\item<7-> $E(X-\mu)=0$
\item<8-> $E(X+Y)=EX+EY$ %dowód wymaga zmiennych losowych dwuwymiarowych
\end{enumerate}
}
\end{block}
\end{frame}
\begin{frame}{Odchylenie standardowe i wariancja}
\begin{align*}
 \sigma=D(X)&=DX=\sqrt{E(X-EX)^2} =\sqrt{\sum_{x_i}(x_i-\mu)^2p_i} \\ & =
 \sqrt{(x_1-\mu)^2p_1+(x_2-\mu)^2p_2+\ldots}
\end{align*}
\only<2-3>
{
	\begin{center}
	\dice{1} \dice{2} \dice{3} \dice{4} \dice{5} \dice{6}\\
	\only<3->{\dice{1} \dice{2} \dice{3} \dice{6} \dice{6} \dice{6}}
	\end{center}
}
\begin{block}<4->{Twierdzenie}
	Dla dowolnego $a\in\R$ i dowolnej zmiennej losowej $X$
\begin{enumerate}
\item<4-> $D(X+a)=DX $
\item<5-> $D^2X=EX^2-E^2X $
\item<6-> $(a\neq \mu)\to (D^2X<E(X-a)^2) $
\item<7-> $P(A=a)=1 \to DA=0 $
\item<8-> $D(aX)=aD(X) $
\end{enumerate}
\end{block}
\end{frame}

\begin{frame}{Para niezależnych zmiennych losowych}
\begin{block}{Twierdzenie}
Dla niezależnych zmiennych losowych $X$, $Y$ zachodzi:
\begin{enumerate}
\item<+-> $E(XY)=EX\cdot EY$
\item<+-> $D^2(X\pm Y)=D^2(X)+D^2(Y)$
\end{enumerate}
\end{block}
\end{frame}

\begin{frame}{Translacja i skalowanie}
\begin{center}
\begin{tikzpicture}[x=.5cm,y=5cm]
\draw [->] (-5,0)--(11,0) node at ++(0,-.05) {$k$};
\draw [->] (0,-.1)--(0,.6) node at ++(-2,0) {$P(X=k)$};
\draw (1,.02)-- ++(0,-.04) node at ++(0,-.04) {1};
\foreach \x in {1,...,10}
	\draw (\x,.02)--++(0,-.04);
\draw (.15,.1)-- ++(-.3,0) node at ++(-.4,0) {$0{,}1$};
\newcommand{\ex}{7.5}
\newcommand{\dx}{1.37}
\newcommand{\h}{.3}
\newcommand{\dist}{5}
\newcommand{\scale}{.5}
\begin{scope}[onslide=<-3>{color4},onslide=<4->{color4!25}]
	\only<2->{\draw (\ex,0)-- ++(0,\h);}
	\only<3->{\fill[semitransparent] ($(\ex,0)-(\dx,0)$) rectangle ++($2*(\dx,0)+(0,\h)$);}
\end{scope}
\only<5>
{
\begin{scope}[color4]
	\draw ($(\ex,0)-(\dist,0)$)-- ++(0,\h);
	\fill[semitransparent] ($(\ex,0)-(\dx,0)-(\dist,0)$) rectangle ++($2*(\dx,0)+(0,\h)$);
\end{scope}
}
\only<7>
{
\begin{scope}[color4]
	\draw ($\scale*(\ex,0)$)-- ++(0,\h);
	\fill[semitransparent] ($\scale*(\ex,0)-\scale*(\dx,0)$) rectangle ++($2*\scale*(\dx,0)+(0,\h)$);
\end{scope}
}
%%rozkład bernouliego n=10 p=.6
%\foreach \x/\p in {0/0.0001,1/0.0016,2/0.0106,3/0.0425,4/0.1115,5/0.2007,6/0.2508,7/0.2150,8/0.1209,9/0.0403,10/0.0060}
%%rozkład bernouliego n=10 p=.55
%\foreach \x/\p in {0/0.0003,1/0.0042,2/0.0229,3/0.0746,4/0.1596,5/0.2340,6/0.2384,7/0.1665,8/0.0763,9/0.0207,10/0.0025}
%rozkład bernouliego n=10 p=.75
\foreach \x/\p in {0/0.0000,1/0.0000,2/0.0004,3/0.0031,4/0.0162,5/0.0584,6/0.1460,7/0.2503,8/0.2816,9/0.1877,10/0.0563}
{
	\fill[alt=<-3>{color2}{color2!50}] (\x,\p) circle (1.5pt);
	\fill[alt=<4-5>{color2}{invisible}] ($(\x,\p)-(\dist,0)$) circle (1.5pt);
	\fill[alt=<6-7>{color2}{invisible}] ($\scale*(\x,0)+(0,\p)$) circle (1.5pt);
%	\fill[color3] ($(\x,\p)-(3,0)$) circle (1pt);
%	\fill[color4] ($.5*(\x,0)+(0,\p)$) circle (1pt);
}
% \draw[thick,color2] plot[domain=-10:-2] (\x,0) plot[domain=-2:2] (\x,{0.11*\x/4+3*0.11/2}) plot[domain=2:4] (\x,{4*0.11/\x}) plot[domain=4:9] (\x,0);
% \draw[thick,color3] plot[domain=-2:2] ({2*\x},{(0.11*\x/4+3*0.11/2)/2}) plot[domain=2:4] ({2*\x},{(4*0.11/\x)/2});
% \draw[dashed,color4] (1.02,0) -- ++(0,0.5);
% \draw[dashed,color4] ($(1.02,0)-(1.58,0)$) -- ++(0,0.5);
% \draw[dashed,color4] ($(1.02,0)+(1.58,0)$) -- ++(0,0.5);
%\draw[thick,color3] plot[domain=-10:-2] (2*\x,0) plot[domain=-2:2] (2*\x,{0.11*\x/4+3*0.11/2}) plot[domain=2:4] (2*\x,{4*0.11/\x}) plot[domain=4:9] (2*\x,0);
%\draw[thick,color4] (-10,0)--(-2,0) plot[domain=-2:2] ({2*\x},{0.11*\x/4+3*0.11/2}) plot[domain=2:4] ({2*\x},{4*0.11/\x}) (4,0)--(9,0);
\end{tikzpicture}
\end{center}
\end{frame}

\begin{frame}{Nierówność Markowa}
%to sformułowanie z prawdopodobieństwem jest u A.A.Borowkow
\[ P(X\geq 0)=1 \to \forall k>0\colon P(X\geq k\mu)\leq \frac{1}{k} \]
\end{frame}

%o tym nie opowiadam, niech to się na statystyce pojawi
%\begin{frame}{Parametry pozycyjne}
%\begin{center}
%\begin{tikzpicture}[x=.5cm,y=5cm]
%\draw [->] (-5,0)--(11,0) node at ++(0,-.05) {$k$};
%\draw [->] (0,-.1)--(0,.6) node at ++(-2,0) {$P(X=k)$};
%\draw (1,.02)-- ++(0,-.04) node at ++(0,-.04) {1};
%\foreach \x in {1,...,10}
%	\draw (\x,.02)--++(0,-.04);
%\draw (.15,.1)-- ++(-.3,0) node at ++(-.4,0) {$0{,}1$};
%\newcommand{\ex}{7.5}
%\newcommand{\dx}{1.37}
%\newcommand{\h}{.3}
%\newcommand{\dist}{5}
%\newcommand{\scale}{.5}
%\foreach \x/\p in {0/0.0000,1/0.0000,2/0.0004,3/0.0031,4/0.0162,5/0.0584,6/0.1460,7/0.2503,8/0.2816,9/0.1877,10/0.0563}
%{
%
%	\ifnum\x=8
%		\fill[color4] (\x,\p) circle (1.5pt);
%	\else
%		\fill[color2] (\x,\p) circle (1.5pt);
%	\fi
%}
%\end{tikzpicture}
%\end{center}
%\end{frame}

\end{document}



