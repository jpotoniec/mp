\documentclass{mp}
\graphicspath{{08_montecarlo}}
\subtitle{Podstawowa metoda Monte Carlo}
\newcounter{good}
\begin{document}
\frame{\titlepage}
\begin{frame}{Przybliżanie wartości $\pi$}
\setcounter{good}{0}
\def\smgend{1}
\only<3-4>{\def\smgend{40}}
\only<5>{\def\smgend{400}}
\only<6>{\def\smgend{4000}}
\begin{center}
\begin{tikzpicture}
\node at (-1.2,-1.2) {\tiny $(-1,-1)$};
\node at (1.2,1.2) {\tiny $(1,1)$};
\draw (-1,-1) rectangle (1,1);
\draw (0,0) circle (1);
\foreach \n in {1,...,\smgend}
{
	\pgfmathparse{rand} \edef\x{\pgfmathresult}
	\pgfmathparse{rand} \edef\y{\pgfmathresult}
	\pgfmathparse{sqrt(pow(\x,2)+pow(\y,2))<=1};
%	\typeout{x=\x\  y=\y\  dst=\pgfmathresult}
	\ifnum\pgfmathresult=1
		\fill[color2] (\x,\y) circle (0.7pt);
		\addtocounter{good}{1}
	\else
		\fill[color4] (\x,\y) circle (0.7pt);
	\fi
}
\end{tikzpicture}
\begin{gather*}
	Z=\begin{cases} 1 & \sqrt{X^2+Y^2}\leq 1 \\ 0 & \text{wpp} \end{cases} \\
	\only<2->{P(Z=1)=\alert<2>{?} \\}
	\only<3->{U=Z_1+Z_2+\ldots+Z_{\smgend} \qquad EU=\alert<3>{?}\\}
	\only<4->{u=\thegood \qquad \only<5->{\pi\approx\pgfmathparse{\thegood/(\smgend/4.0)}\pgfmathresult}\\}
\end{gather*}
\end{center}
\end{frame}
\begin{frame}{Oszacowanie jakości przybliżenia}
\begin{block}{Nierówność Czebyszewa}
%\[\sigma^2<\infty \to \forall k>0\colon P(\left|X-\mu\right|\geq k\sigma)\leq \frac{1}{k^2} \]
Niech $\mu,\sigma<\infty$. Dla $\varepsilon\in\R_{+}$ i $k=\frac{\varepsilon}{\sigma}$ zachodzą następujące nierówności:
\begin{align}
P(\left|X-\mu\right|\geq k\sigma)\leq & \frac{1}{k^2} \\
P(\left|X-\mu\right|\leq k\sigma)\geq & 1-\frac{1}{k^2} \\
P(\left|X-\mu\right|\leq \varepsilon)\geq & 1-\frac{\sigma^2}{\varepsilon^2}
\end{align}
\end{block}

\only<2->{Ile należy wylosować punktów, żeby z prawdopodobieństwo otrzymania wyniku różnego o nie więcej niż $0{,}1$ od $\pi$ wynosiło przynajmniej $90\%$?}

\note<2>{
	\begin{gather*}
	U\sim B(n,\frac{\pi}{4})\quad EU=n\frac{\pi}{4}\quad D^2U=n\frac{\pi(4-\pi)}{16} \\
	Q=\frac{4}{n}U \quad EQ=\pi \quad D^2Q=\frac{16}{n^2}D^2U=\frac{\pi(4-\pi)}{n} \\
	\varepsilon=0{,}1 \qquad p_{min}=0{,}9 \\
	P(\left|Q-EQ\right|\leq \varepsilon)=P(\left|Q-\pi\right|\leq\varepsilon)\geq 1-\frac{\pi(4-\pi)}{\varepsilon^2n} \\
	1-\frac{\pi(4-\pi)}{\varepsilon^2n}\geq p_{min}\\
	n\geq\frac{\pi(4-\pi)}{\varepsilon^2(1-p_{min})}\approx 2697
	\end{gather*}
	Przy czym skoro nie znamy $\pi$ to należałoby je oszacować z dołu, np. przez 3, wtedy:
	$n\geq\frac{3(4-3)}{\varepsilon^2(1-p_{min})}= 3000$
}

\end{frame}
\end{document}
