\documentclass{mp}
\graphicspath{{11_prawa_wielkich_liczb}}
\subtitle{Prawa wielkich liczb}
\begin{document}
\frame{\titlepage}
\begin{frame}{Założenia}
$(X_1,X_2,\ldots)$ ciąg niezależnych zmiennych losowych\\
\[EX_i=\mu_i \qquad D^2X_i=\sigma_i^2\] \\
\pause
\begin{gather*}
M_n=\frac{1}{n}\sum_{i=1}^n X_i \\
EM_n=\alert{?} \qquad D^2(M_n)=\alert{?}
\end{gather*}
\note{\[EM_n=\frac{1}{n}\sum_{i=1}^n \mu_i \qquad D^2(M_n)=\frac{1}{n^2}\sum_{i=1}^n\sigma_i^2\]}
\end{frame}
\begin{frame}{Słabe prawo wielkich liczb Markowa}
\[ \lim_{n\to\infty} D^2(M_n)=0 \to \forall \varepsilon>0\colon \lim_{n\to\infty} P(\left|M_n-EM_n\right|\geq\varepsilon)=0 \]
\note
{
	Z nierówności Czebyszewa:
	\[P(\left|M_n-EM_n\right|\geq\varepsilon)\leq \frac{D^2(M_n)}{\varepsilon}\]
	Przechodząc obustronnie do granicy otrzymujemy, że prawa strona dąży do 0, a ponieważ lewa strona jest nieujemna, więc również musi dążyć do zera.
}
\end{frame}
\begin{frame}{Słabo prawo wielkich liczb Chinczyna}
$(X_1,X_2,\ldots)$ o tym samym rozkładzie, $EX_i=EM_n=\mu$
\[ \forall \varepsilon>0\colon \lim_{n\to\infty} P(\left|M_n-\mu\right|\geq\varepsilon)=0 \]
\note
{
	Ponieważ zmienne są niezależne i mają identycznye rozkłady, niech $D^2X_i=\sigma^2$, wtedy $D^2M_n=\frac{\sigma^2}{n}$. Z nierówności Czebyszewa
	\[P(\left|M_n-\mu\right|\geq\varepsilon)\leq \frac{\sigma^2}{n\varepsilon} \]
	Przechodząc obustronnie do granicy otrzymujemy, że prawa strona dąży do 0, a ponieważ lewa strona jest nieujemna, więc również musi dążyć do zera.
}
\end{frame}
\begin{frame}{Mocne prawo wielkich liczb}
$(X_1,X_2,\ldots)$ o tym samym rozkładzie, $EX_i=EM_n=\mu$
\[P(\lim_{n\to\infty} M_n=\mu)=1\]
\note{Nie dowodzimy, dowód jest dość złożony}
\end{frame}
%TODO: flowchart
\end{document}
