\documentclass{mp}
\graphicspath{{12_procesy}}
\subtitle{Procesy losowe}
\begin{document}
\frame{\titlepage}
\begin{frame}{Sumaryczny ruch w punkcie wymiany ruchu \emph{Thinx}}
\uncover<3->{\[\{X_t(\omega)\colon t\in T, \omega\in\Omega\} \quad T\subseteq \R \quad \uncover<4-5>{\only<4>{\alert{t=const}}\only<5>{\alert{\omega=const}}}\]}
\center
\begin{tikzpicture}[x=.1cm,y=2cm]
\input{12_procesy/thinx.tex}
\end{tikzpicture}

{\tiny Dane pochodzą z \url{http://www.thinx.pl/siec/statystyki_ruchu}}
\note<1>
{
	Dla ustalonego $t$: wartość procesu losowego \\
	Dla ustalonego $\omega$: realizacja procesu losowego \\
	Jeżeli $\left|T\right|\leq\aleph_0$, to powstaje ciąg lub wektor. Dla skupienia uwagi $T$ jest przedziałem.
}
\end{frame}
\begin{frame}{Rozkład procesu losowego}
%troche z tego jest na poczatku rozdzialu 12-tego u Borowkowa, troche jest w notatkach od profa, a troche wypracowalem sam na podstawie definicji funkcji mierzalnej
\begin{gather*}
\uncover<+->{(\Omega,\mathcal{A}_\Omega,P) \qquad \{X(t,\omega)\colon t\in T, \omega\in\Omega\} \\}
\uncover<+->{\delta\colon \Omega\to \Omega' \qquad \delta(\omega)=X(\cdot,\omega) \\}
\uncover<+->{\mathcal{A}_{\Omega'}=\{A\subseteq \Omega' \colon \uncover<+->{\delta^{-1}(A)=\{\omega\in\Omega\colon X(\cdot,\omega)\in A\}\in\mathcal{A}_\Omega} \} \\}
\uncover<+->{P'\colon \mathcal{A}_{\Omega'}\to \left<0,1\right> \qquad \forall A\in \mathcal{A}_{\Omega'}\colon P'(A)=P(\delta^{-1}(A)) \\}
\uncover<+->{(\Omega',\mathcal{A}_{\Omega'},\alert{P'})}
\end{gather*}
\end{frame}
\begin{frame}{$n$-wymiarowy rozkład procesu losowego}
\begin{gather*}
\uncover<+->{t_1,t_2,\ldots,t_n \in T \\}
\uncover<+->{\left(X_{t_1},X_{t_2},\ldots,X_{t_n}\right) \\}
\uncover<+->{F(x_1,t_1; x_2,t_2; \ldots; x_n,t_n)=P(X_{t_1}<x_1,X_{t_2}<x_2,\ldots,X_{t_n}<x_n) }
\end{gather*}
\uncover<+->
{
	%http://en.wikipedia.org/wiki/Kolmogorov_extension_theorem
	%TODO: ustalić pełną nazwę twierdzenia po polsku -- jest u Borowkowa zdaje sie
\begin{block}{Twierdzenie (wniosek z twierdzenia Kołmogorowa)}
Znajomość wszystkich $n$-wymiarowych rozkładów dla dowolnego, skończonego $n$ potrzeba i wystarcza do pełnego scharakteryzowania procesu losowego.
\end{block}
}
\end{frame}
\begin{frame}{Momenty procesu losowego}
\begin{description}<+->
\item[wartość średnia] \[ \mu(t)=E\left[X(t)\right] \]
\item[odchylenie standardowe] \[ \sigma(t)=D(X(t))=\sqrt{E\left[(X(t)-\mu(t))^2\right]} \]
\item[funkcja autokorelacji] \[R_{XX}(t_1,t_2)=E\left[(X(t_1)-\mu(t_1))(X(t_2)-\mu(t_2))\right] \]
\item[współczynnik autokorelacji] \[r_{XX}(t_1,t_2)=\frac{R_{XX}(t_1,t_2)}{\sigma(t_1)\sigma(t_2)} \]
\item[$n$-wymiarowa funkcja korelacyjna]
\end{description}
\end{frame}
\end{document}
