\documentclass{mp}
\graphicspath{{13_lancuchy_markowa}}
\subtitle{Łańcuchy Markowa}
\usetikzlibrary{automata}
\begin{document}
\frame{\titlepage}
\begin{frame}{Łańcuch Markowa}
\[ \{X_t,T\} \]
\begin{itemize}
\item $T=\{0,1,\ldots,n\}$ lub $T=\{0,1,\ldots,n,\ldots\}$
\item $P(X_t\in\{x_0,x_1,\ldots\})=1$
\item $\{X_t\}$ jest procesem Markowa
\item \[\forall x_i,x_j,t\colon P(X_t=x_j|X_{t-1}=x_i)=P_{i,j}\]
\end{itemize}
\end{frame}
\begin{frame}{Macierz przejść}
\begin{gather*}
\bm{P}=\begin{bmatrix} 
P_{0,0} & P_{0,1} & \ldots & P_{0,j} & \ldots \\
P_{1,0} & P_{1,1} & \ldots & P_{1,j} & \ldots \\
\vdots & \vdots & \ddots & \vdots & \ddots \\
P_{i,0} & P_{i,1} & \ldots & P_{i,j} & \ldots \\
\vdots & \vdots & \ddots & \vdots & \ddots \\
\end{bmatrix} \\
\bm{P(X_t)}=\begin{bmatrix} P(X_t=a_1) & P(X_t=a_2) & \ldots & P(X_t=a_j) & \ldots \end{bmatrix} \\
\uncover<2->{\bm{P(X_t)}\bm{P}=\alert<2>{?}\\}
\uncover<3->{\bm{P(X_t)}\bm{P^k}=\alert<3>{?}}
\end{gather*}
\note<2>{
	Ile wynoszą sumy w wierszach w macierzy? \\
	$\bm{P(X_t)}\bm{P}=\bm{P(X_{t+1})}$}
\end{frame}
\begin{frame}{Przykład: krótka kolejka}
\only<-4>
{
\center
\begin{tikzpicture}[every state/.style={circle}]
\node (n0) [state] {0};
\node (n1) [state,right=of n0] {1};
\node (n2) [state,right=of n1] {2};
\node (n3) [state,right=of n2] {3};
\node (n4) [state,right=of n3] {4};
\node (n5) [state,right=of n4] {5};
\foreach \src/\dst in {n0/n1,n1/n2,n2/n3,n3/n4,n4/n5}
{
	\path[->] (\src) edge[bend left] node[above] {$0{,}1$} (\dst);
	\path[->] (\dst) edge[bend left] node[below] {$0{,}2$} (\src);
}
\path[->] (n0) edge[loop above] node[above] {$0{,}9$} (n0);
\foreach \n in {n1,n2,n3,n4}
	\path[->] (\n) edge[loop above] node[above] {$0{,}7$} (\n);
\path[->] (n5) edge[loop above] node[above] {$0{,}8$} (n5);
\end{tikzpicture}
}
\only<5->
{
	\begin{align*}
	 \bm{P}(X_0) & = \begin{bmatrix} 0{,}10 & 0{,}10 & 0{,}15 & 0{,}25 & 0{,}25 & 0{,}15 \end{bmatrix}  \\
	 \uncover<6->{\bm{P}(X_0)\bm{P}^{10} & = \begin{bmatrix} 0{,}23 &  0{,}18 &  0{,}18 &  0{,}17 &  0{,}14 &  0{,}09 \end{bmatrix}}
	\end{align*}
}
\uncover<2->
{
\begin{columns}[T]
\begin{column}{.49\textwidth}
\tabcolsep=.8\tabcolsep
\begin{tabular}{l|llllll}
$\bm{P}$ & 0 & 1 & 2 & 3 & 4 & 5 \\
\hline
0 & $0{,}9$ & $0{,}1$ &  &  &  &  \\
1 & $0{,}2$ & $0{,}7$ & $0{,}1$ &  &  &  \\
2 &  & $0{,}2$ & $0{,}7$ & $0{,}1$ &  &  \\
3 &  &  & $0{,}2$ & $0{,}7$ & $0{,}1$ &  \\
4 &  &  &  & $0{,}2$ & $0{,}7$ & $0{,}1$ \\
5 &  &  &  &  & $0{,}2$ & $0{,}8$  \\
\end{tabular}
\end{column}
\begin{column}{.49\textwidth}
\uncover<3->
{
\tabcolsep=.5\tabcolsep
\only<3>
{
\begin{tabular}{l|llllll}
$\alert{\bm{P}^2}$ & 0 & 1 & 2 & 3 & 4 & 5 \\
\hline
0 & $0{,}83$ & $0{,}16$ & $0{,}01$ &  &  & \\
1 & $0{,}32$ & $0{,}53$ & $0{,}14$ & $0{,}01$ &  &  \\
2 & $0{,}04$ & $0{,}28$ & $0{,}53$ & $0{,}14$ & $0{,}01$ & \\
3 &  & $0{,}04$ & $0{,}28$ & $0{,}53$ & $0{,}14$ & $0{,}01$ \\
4 &  &  & $0{,}04$ & $0{,}28$ & $0{,}53$ & $0{,}15$ \\
5 &  &  &  & $0{,}04$ & $0{,}30$ & $0{,}66$
\end{tabular}
}
\only<4->
{
\begin{tabular}{l|llllll}
$\alert<4>{\bm{P}^{10}}$ & 0 & 1 & 2 & 3 & 4 & 5\\
\hline
 0 &  $0{,}63$ & $0{,}26$ & $0{,}09$ & $0{,}02$ &  &  \\
 1 &  $0{,}52$ & $0{,}27$ & $0{,}13$ & $0{,}05$ & $0{,}01$ &  \\
 2 &  $0{,}35$ & $0{,}27$ & $0{,}20$ & $0{,}12$ & $0{,}05$ & $0{,}02$ \\
 3 &  $0{,}18$ & $0{,}20$ & $0{,}24$ & $0{,}20$ & $0{,}12$ & $0{,}06$ \\
 4 &  $0{,}07$ & $0{,}12$ & $0{,}20$ & $0{,}25$ & $0{,}22$ & $0{,}15$ \\
 5 &  $0{,}02$ & $0{,}06$ & $0{,}14$ & $0{,}24$ & $0{,}30$ & $0{,}25$ \\
\end{tabular}
}
}
\end{column}
\end{columns}
}
\end{frame}
\end{document}
