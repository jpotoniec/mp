\documentclass{mp}
\graphicspath{{13_lancuchy_markowa}}
\subtitle{Łańcuchy Markowa}
\usetikzlibrary{automata}
\begin{document}
\frame{\titlepage}
\begin{frame}{Łańcuch Markowa}
\[ \{X_t,T\} \]
\begin{itemize}
\item $T=\{0,1,\ldots,n\}$ lub $T=\{0,1,\ldots,n,\ldots\}$
\item $P(X_t\in\{x_0,x_1,\ldots\})=1$
\item $\{X_t\}$ jest procesem Markowa
\item \[\forall x_i,x_j,t\colon P(X_t=x_j|X_{t-1}=x_i)=P_{i,j}\]
\end{itemize}
\end{frame}
\begin{frame}{Macierz przejść}
\begin{gather*}
\bm{P}=\begin{bmatrix} 
P_{0,0} & P_{0,1} & \ldots & P_{0,j} & \ldots \\
P_{1,0} & P_{1,1} & \ldots & P_{1,j} & \ldots \\
\vdots & \vdots & \ddots & \vdots & \ddots \\
P_{i,0} & P_{i,1} & \ldots & P_{i,j} & \ldots \\
\vdots & \vdots & \ddots & \vdots & \ddots \\
\end{bmatrix} \\
\bm{P(X_t)}=\begin{bmatrix} P(X_t=a_1) & P(X_t=a_2) & \ldots & P(X_t=a_j) & \ldots \end{bmatrix} \\
\uncover<2->{\bm{P(X_t)}\bm{P}=\alert<2>{?}\\}
\uncover<3->{\bm{P(X_t)}\bm{P^k}=\alert<3>{?}}
\end{gather*}
\note<2>{
	Ile wynoszą sumy w wierszach w macierzy? \\
	$\bm{P(X_t)}\bm{P}=\bm{P(X_{t+1})}$}
\end{frame}
\begin{frame}{Przykład: krótka kolejka}
\only<-4>
{
\center
\begin{tikzpicture}[every state/.style={circle}]
\node (n0) [state] {0};
\node (n1) [state,right=of n0] {1};
\node (n2) [state,right=of n1] {2};
\node (n3) [state,right=of n2] {3};
\node (n4) [state,right=of n3] {4};
\node (n5) [state,right=of n4] {5};
\foreach \src/\dst in {n0/n1,n1/n2,n2/n3,n3/n4,n4/n5}
{
	\path[->] (\src) edge[bend left] node[above] {$0{,}1$} (\dst);
	\path[->] (\dst) edge[bend left] node[below] {$0{,}2$} (\src);
}
\path[->] (n0) edge[loop above] node[above] {$0{,}9$} (n0);
\foreach \n in {n1,n2,n3,n4}
	\path[->] (\n) edge[loop above] node[above] {$0{,}7$} (\n);
\path[->] (n5) edge[loop above] node[above] {$0{,}8$} (n5);
\end{tikzpicture}
}
\only<5->
{
	\begin{align*}
	 \bm{P}(X_0) & = \begin{bmatrix} 0{,}10 & 0{,}10 & 0{,}15 & 0{,}25 & 0{,}25 & 0{,}15 \end{bmatrix}  \\
	 \uncover<6->{\bm{P}(X_0)\bm{P}^{10} & = \begin{bmatrix} 0{,}23 &  0{,}18 &  0{,}18 &  0{,}17 &  0{,}14 &  0{,}09 \end{bmatrix}}
	\end{align*}
}
\uncover<2->
{
\begin{columns}[T]
\begin{column}{.49\textwidth}
\tabcolsep=.8\tabcolsep
\begin{tabular}{l|llllll}
$\bm{P}$ & 0 & 1 & 2 & 3 & 4 & 5 \\
\hline
0 & $0{,}9$ & $0{,}1$ &  &  &  &  \\
1 & $0{,}2$ & $0{,}7$ & $0{,}1$ &  &  &  \\
2 &  & $0{,}2$ & $0{,}7$ & $0{,}1$ &  &  \\
3 &  &  & $0{,}2$ & $0{,}7$ & $0{,}1$ &  \\
4 &  &  &  & $0{,}2$ & $0{,}7$ & $0{,}1$ \\
5 &  &  &  &  & $0{,}2$ & $0{,}8$  \\
\end{tabular}
\end{column}
\begin{column}{.49\textwidth}
\uncover<3->
{
\tabcolsep=.5\tabcolsep
\only<3>
{
\begin{tabular}{l|llllll}
$\alert{\bm{P}^2}$ & 0 & 1 & 2 & 3 & 4 & 5 \\
\hline
0 & $0{,}83$ & $0{,}16$ & $0{,}01$ &  &  & \\
1 & $0{,}32$ & $0{,}53$ & $0{,}14$ & $0{,}01$ &  &  \\
2 & $0{,}04$ & $0{,}28$ & $0{,}53$ & $0{,}14$ & $0{,}01$ & \\
3 &  & $0{,}04$ & $0{,}28$ & $0{,}53$ & $0{,}14$ & $0{,}01$ \\
4 &  &  & $0{,}04$ & $0{,}28$ & $0{,}53$ & $0{,}15$ \\
5 &  &  &  & $0{,}04$ & $0{,}30$ & $0{,}66$
\end{tabular}
}
\only<4->
{
\begin{tabular}{l|llllll}
$\alert<4>{\bm{P}^{10}}$ & 0 & 1 & 2 & 3 & 4 & 5\\
\hline
 0 &  $0{,}63$ & $0{,}26$ & $0{,}09$ & $0{,}02$ &  &  \\
 1 &  $0{,}52$ & $0{,}27$ & $0{,}13$ & $0{,}05$ & $0{,}01$ &  \\
 2 &  $0{,}35$ & $0{,}27$ & $0{,}20$ & $0{,}12$ & $0{,}05$ & $0{,}02$ \\
 3 &  $0{,}18$ & $0{,}20$ & $0{,}24$ & $0{,}20$ & $0{,}12$ & $0{,}06$ \\
 4 &  $0{,}07$ & $0{,}12$ & $0{,}20$ & $0{,}25$ & $0{,}22$ & $0{,}15$ \\
 5 &  $0{,}02$ & $0{,}06$ & $0{,}14$ & $0{,}24$ & $0{,}30$ & $0{,}25$ \\
\end{tabular}
}
}
\end{column}
\end{columns}
}
\end{frame}
\begin{frame}{Klasyfikacja stanów}
\note<1>{
	$r_{i,j}^t$ prawdopodobieństwo, że przejdziemy z $x_i$ do $x_j$ w dokładnie $t$ krokach.
	Nie tworzy zmiennej losowej, bo nie musi się sumować do 1. \\
	$h_{i,j}$ to średni czas przejścia $i\to j$. \\
	Do stanu powracającego łańcuch wraca nieskończenie wiele razy \\
	Stany zerowe powracające nie mogą występować w łańcuchach skończonych.
}
\begin{gather*}
 r_{i,j}^t=P\left(X_t=x_j, X_{t-1}\neq x_j, X_{t-2}\neq x_j, \ldots, X_1\neq x_j,  | X_0=x_i\right) \\
 h_{i,j}=\sum_{t\in \{1,2,\ldots\}} tr_{i,j}^t
\end{gather*}
\begin{description}
\item[dodatni powracający] \[\sum_{t\in \{1,2,\ldots\}} r_{i,i}^t =1 \land h_{i,i}<\infty\]
\item[zerowy powracający] \[\sum_{t\in \{1,2,\ldots\}} r_{i,i}^t =1 \land h_{i,i}=\infty\]
\item[przejściowy] \[\sum_{t\in \{1,2,\ldots\}} r_{i,i}^t <1 \]
\end{description}
\end{frame}
\begin{frame}{Przykład}
\center
\begin{tikzpicture}
\node (n1) [state] {1};
\node (n2) [state,right=of n1] {2};
\node (n3) [state,right=of n2] {3};
\node (dots1) [right=of n3] {\ldots};
\node (abovedots1) [above=of dots1] {};
\node (nn) [state,right=of dots1] {n};
\node (dots2) [right=of nn] {\ldots};


\path[->] (n1) edge[bend left] node[above] {$\frac{1}{2}$} (n2);
\path[->] (n1) edge[loop above] node[above] {$\frac{1}{2}$} (n1);
\path[->] (n2) edge[bend left] node[above] {$\frac{2}{3}$} (n3);
\path[->] (n2) edge[bend left] node[below] {$\frac{1}{3}$} (n1);
\path[->] (n3) edge[bend left] node[below] {$\frac{1}{4}$} (n1);
\path[->] (n3) edge[bend left] node[above] {$\frac{3}{4}$} (dots1);
\path[->] (dots1) edge[bend left] node[above] {$\frac{n-1}{n}$} (nn);
\path[->] (nn) edge[bend left] node[above] {$\frac{n}{n+1}$} (dots2);
\path[->] (nn) edge[bend left] node[below] {$\frac{1}{n+1}$} (n1);
\end{tikzpicture}
\note<1>
{
\begin{gather*}
r_{1,1}^t=\prod_{j=1}^{t-1} \frac{j}{j+1}\cdot\frac{1}{t+1}=\frac{1}{2}\cdot\frac{2}{3}\cdot\ldots\cdot\frac{t-1}{t}\cdot\frac{1}{t+1}=\frac{1}{t(t+1)} \\
\sum_{t=1}^\infty \frac{1}{t(t+1)}=\lim_{n\to\infty} \frac{n}{n+1}=1 \\
h_{i,i}=\sum_{t=1}^\infty \frac{t}{t(t+1)}=\sum_{t=1}^\infty \frac{1}{t+1}=\infty
\end{gather*}
}
\begin{gather*}
\uncover<2->{r_{1,1}^t=\alert<2>{?} \\}
\uncover<3->{\sum_{t=1}^\infty r_{1,1}^t=\alert<3>{?} \\}
\uncover<4->{h_{1,1}^t=\alert<4>{?}}
\end{gather*}
\end{frame}

%\begin{block}{Stan okresowy $x_j$}
%\begin{gather*}
% \exists \Delta\in\{2,3,\ldots\} \forall t,t+s\in T\colon \\ \left( P(X_{t+s}=x_j|X_t=x_j)=0 \iff s\mod\Delta\neq 0\right) 
%\end{gather*}
%\end{block}


\end{document}
