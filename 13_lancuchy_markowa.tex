\documentclass{mp}
\graphicspath{{13_lancuchy_markowa}}
\subtitle{Łańcuchy Markowa}
\begin{document}
\frame{\titlepage}
\begin{frame}{Łańcuch Markowa}
\[ \{X_t,T\} \]
\begin{itemize}
\item $T\subseteq \N$
\item $P(X_t\in\{x_0,x_1,\ldots\})=1$
\item $\{X_t\}$ jest procesem Markowa
\item \[\forall x_i,x_j,t\colon P(X_t=x_j|X_{t-1}=x_i)=P_{i,j}\]
\end{itemize}
\end{frame}
\begin{frame}{Macierz przejść}
\begin{gather*}
\bm{P}=\begin{bmatrix} 
P_{0,0} & P_{0,1} & \ldots & P_{0,j} & \ldots \\
P_{1,0} & P_{1,1} & \ldots & P_{1,j} & \ldots \\
\vdots & \vdots & \ddots & \vdots & \ddots \\
P_{i,0} & P_{i,1} & \ldots & P_{i,j} & \ldots \\
\vdots & \vdots & \ddots & \vdots & \ddots \\
\end{bmatrix} \\
\bm{P(X_t)}=\begin{bmatrix} P(X_t=a_1) & P(X_t=a_2) & \ldots & P(X_t=a_j) & \ldots \end{bmatrix} \\
\uncover<2->{\bm{P(X_t)}\bm{P}=\alert<2>{?}\\}
\uncover<3->{\bm{P(X_t)}\bm{P^k}=\alert<3>{?}}
\end{gather*}
\note<2>{
	Ile wynoszą sumy w wierszach w macierzy? \\
	$\bm{P(X_t)}\bm{P}=\bm{P(X_{t+1})}$}
\end{frame}
\begin{frame}{Przykład}
%TODO: chciałbym tutaj coś podobnego do rysunku 7.1 u Mitzenmachera
\end{frame}
\end{document}
