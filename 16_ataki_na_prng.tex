\documentclass{mp}
\usepackage[linesnumbered]{algorithm2e}
\DontPrintSemicolon
\graphicspath{{15_prng}}
\subtitle{Generatory liczb pseudolosowych}

\usepackage{listings}

\begin{document}
\frame{\titlepage}
\begin{frame}{Odwrotność modulo}
\begin{description}
\item[odwrotność] \[ x\cdot \alert{x^{-1}} = 1 \qquad x^{-1}\in\R \]
\pause
\item[odwrotność modulo] \[ x\cdot \alert{x^{-1}} \mod m = 1 \qquad x^{-1}\in \{0,1,\ldots, m-1\} \]
\end{description}
\pause
\begin{block}{Twierdzenie}
$x^{-1}$ istnieje wtw $\nwd(x, m)=1$
\end{block}
\end{frame}

\begin{frame}{Rozszerzony algorytm Euklidesa}
\begin{minipage}{.04\textwidth} %takie przesuniecie, zeby nr linii nie wychodzily za ekran
~
\end{minipage}
\begin{minipage}{.45\textwidth}
\begin{algorithm}[H]
$r_0 \leftarrow m \qquad r_1 \leftarrow x$ \;
$s_0 \leftarrow 0 \qquad s_1 \leftarrow 1$ \;
$i \leftarrow 1$ \;
\While{$r_i \neq 0$}
{
	$q \leftarrow r_{i-1} \div r_i$ \;
	$r_{i+1} \leftarrow r_{i-1} - q r_i$ \;
	$s_{i+1} \leftarrow s_{i-1} - q s_i$ \;
	$i \leftarrow i+1$ \;
}
\lIf{$s_{i-1}<0$}{\Return{$s_{i-1}+m$}}
\lElse{\Return{$s_{i-1}$}}
\end{algorithm}
\end{minipage}
\begin{minipage}{.49\textwidth}
$3^{-1} \mod 17 = \alert{?}$ \\
\pause
\begin{tabular}{r|rrr}
$i$ & $q$ & $r_i$ & $s_i$ \\
\hline
0 & & 17 & 0 \\
1 & & 3 & 1 \\
\pause
2 & 5 & 2 & $-5$ \\
\pause
3 & 1 & 1 & \alert<6>{6} \\
\pause
4 & 2 & \alert<5>{0} & $-17$
\end{tabular}
\pause\\
$3\cdot \alert{6} = 18 = 1\cdot 17+1 $
\end{minipage}
\end{frame}

\begin{frame}{Łamanie LCG}
\begin{description}
\item[wyjście LCG] 17, 15, 76, 38, 63, 70, 59, 30, 64, 80
\item[zadanie] $m=?$, $a=?$, $c=?$
\end{description}
\end{frame}

\begin{frame}{Jeżeli znamy $m$}
\begin{gather*}
x_2 = ax_1+c \mod m \\
x_3 = ax_2+c \mod m \\
x_3-x_2 = a(x_2-x_1) \mod m \\
\left[(x_3-x_2)(x_2-x_1)^{-1} \equiv a \right] \mod m
%\left[
\end{gather*}
\end{frame}

\begin{frame}{Odtwarzanie $m$}
\begin{align*}
\only<1-4>{x_2 = & (ax_1+c)\mod m \\}
\only<2-4>{x_3 = & (ax_2+c)\mod m=\alt<-2>{\alert<3>{\ldots}}{(a^2x_1+(a+1)c)\mod m}\\}
\only<3-4>{x_4 = & (ax_3+c)\mod m=\alt<-3>{\alert{\ldots}}{(a^3x_1+(a^2+a+1)c)\mod m}\\}
\only<4-10>{x_n = & \alt<-4>{\alert{\ldots}}{(a^{n-1}x_1+(a^{n-2}+\ldots+1)c)\mod m}\\}
\only<5-10>{x_{n+1}-x_n = & \alt<-5>{\alert{\ldots}}{a^{n-1}(x_2-x_1) \mod m=a^{n-1}\alert<7>{\delta}\mod m \\}}
\only<6-10>{x_{n+2}-x_n = & \alt<-6>{\alert{\ldots}}{a^{n-1}(a+1)\delta\mod m} \\}
\only<7->{s= & (x_4-x_2)(x_2-x_1)=\alt<-7>{\alert<7>{\ldots}}{a(a+1)\delta\cdot\delta} \\
t= & (x_3-x_2)(x_3-x_1)=\alt<-7>{\alert<7>{\ldots}}{a\delta(a+1)\cdot \delta} \\
}
\only<8->{s-t\mod m = & \alert<8>{\ldots} \\}
\only<9->{u=& (x_3-x_2)^2=\alert<9>{\ldots} \\
v=& (x_4-x_3)(x_2-x_1)=\alert<9>{\ldots} \\
\only<10->{u-v\mod m = & \alert<10>{\ldots} \\}
\only<11->{\nwd(s-t, u-v)= & \alt<-11>{\alert<11>{\ldots}}{\hat{m}} \\}
%\only<12->{x_3-x_2\equiv a\delta \mod m \to (x_3-x_2)\delta^{-1}\equiv a \mod m \\}
}
\end{align*}
\end{frame}
%\only<14>{.}
%}
%\note<4>
%{
%Dla pierwszych czterech:
%\begin{gather*}
%s=(38-15)\cdot(15-17)=-46 \qquad t=(76-15)\cdot(76-17)=3599 \qquad s-t=-3645 \\
%u=(76-15)^2=3721 \qquad v=(38-76)\cdot(15-17)=76 \qquad u-v=3645 \\
%\nwd(s-t, u-v) = 3645
%\end{gather*}
%
%Dla liczb 2-5:
%\begin{gather*}
%s=-732 \qquad t=-874 \qquad u=1444 \qquad v=1586 \\
%\nwd(s-t, u-v, 3645) = 142
%\end{gather*}
%}
%\end{frame}
\end{document}