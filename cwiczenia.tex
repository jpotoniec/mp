\documentclass{mwart}
\usepackage[utf8]{inputenc}
\usepackage{polski}
\usepackage{amsmath,amssymb}
\usepackage[margin=2cm]{geometry}
\usepackage{tikz}
\usepackage{diagbox}
\pagestyle{empty}
\linespread{1.3}
\newcommand{\abs}[1]{\left|#1\right|}
\ifx\odp\undefined
\newcommand{\ans}[1]{}
\else
\newcommand{\ans}[1]{\emph{Odpowiedź:} #1}
\fi
\begin{document}
\section{Przestrzeń probabilistyczna}
\begin{enumerate}
\item Na ile sposobów można ustawić 5 osób w kolejce? \ans{$5!$}
\item Ile słów pięcioliterowych (nawet tych bezsensownych) można utworzyć z liter \emph{A, B, C}? \ans{$3^5=243$}
\item Dane są dwa pojemniki. W pierwszym z nich znajduje się 11 kul: 7 białych i 4 czarne. W~drugim pojemniku jest 6 kul: 3 białe i 3 czarne. Z każdego pojemnika losujemy po dwie kule.
Rozpatrz poniższe pytania w dwóch wariantach: wszystkie kule w danym kolorze są identyczne oraz gdy kule są rozróżnialne (np. numerowane).
\begin{enumerate}
\item Ile jest możliwych układów? \ans{${11 \choose 2}{6 \choose 2}=825$}
\item Ile jest układów składających się wyłącznie z kul czarnych? \ans{${4 \choose 2}{3 \choose 2}=18$}
\item Ile jest układów składających się wyłącznie z kul białych? \ans{${7 \choose 2}{3 \choose 2}=63$}
\item Ile jest układów składających się z pary białej i pary czarnej? \ans{${7\choose 2}{3 \choose 2}+{4 \choose 2}{3 \choose 2}+7\cdot4\cdot3\cdot3=333$}
\item Ile jest układów składających się z jednej kuli białej i trzech czarnych? \ans{$7\cdot4\cdot{3\choose 2}+3\cdot3\cdot{4\choose 2}=138$}
\item Ile jest układów składających się z jednej kuli czarnej i trzech białych?  \ans{$4\cdot7\cdot{3\choose 2}+3\cdot3\cdot{7\choose 2}=273$}
\end{enumerate}
\item Z partii towaru zawierającej sztuki dobre i niedobre losujemy 3 sztuki (\emph{próba}). Niech $A$ oznacza zdarzenie: \emph{dokładnie jedna sztuka dobra w próbie}, $B$ zdarzenie: \emph{co najwyżej jedna sztuka dobra w próbie}, $C$ zdarzenie: \emph{co najmniej jedna sztuka dobra w próbie}. Opisz słownie następujące zdarzenia:
\begin{enumerate}
\item $A'$, $B'$, $C'$
\item $A\cup B$
\item $A\cap B$
\item $B\cup C$
\item $B\cap C$
\item $B'\cap C'$
\end{enumerate}
\item Inżynier projektuje magazyn do przechowywania kartonów puszek żywności. Kartony mają kształt sześcianów o krawędzi 4 dm i masie 50 kg każdy. Zakłada się, że kartony nie mogą być ułożone w wieżę wyższą niż 24 dm. Zaproponować przestrzeń zdarzeń losowych dla następujących doświadczeń:
\begin{enumerate}
\item Obserwacja całkowitego obciążenia 16 $dm^2$ powierzchni pochodzącego z jednego stosu kartonów.
\item Obserwacja całkowitego obciążenia 16 $dm^2$ powierzchni pochodzącego z dwóch stosów kartonów, przy założeniu, że jest to obciążenie wywołane przezpolowę masy każdego z dwóch stosów.
\end{enumerate}
W obu przestrzeniach opisać następujące zdarzenia:
\begin{description}
\item[A] całkowite obciążenie wynosi co najmniej 150 kg;
\item[B] całkowite obciążenie wynosi nie więcej niż 200 kg;
\item[C] całkowite obciążenie przekracza 250 kg;
\end{description}
\item Na 10 kartkach napisano liczby od 1 do 10 i wrzucono do pudełka. Losujemy w sposób przypadkowy dwie kartki. Niech $A$ odpowiada zdarzeniu \emph{wylosowanie kartki z numerem 1}, a $B$ zdarzeniu \emph{wylosowanie pary liczb, których suma jest większa od 4}. Zakładamy, że wylosowanie każdej z kartek jest równoprawdopodobne.
\begin{enumerate}
\item Przedstaw przestrzeń zdarzeń elementarnych. Podaj jej rozmiar. \ans{$\Omega=\{\omega_{\{i,j\}}|i,j\in\{1,\ldots,10\} \land i\neq j\}, \left|\Omega\right|={10 \choose 2}=45$}
\item Zdefiniuj zdarzenia $A$ i $B$ jako zbiory zdarzeń elementarnych. \ans{$A=\{\omega_{\{i,j\}}\in\Omega|i=1 \lor j=1\}$, $B=\{\omega_{\{i,j\}}\in\Omega|i+j>4\}$}
\item Ile jest zdarzeń sprzyjających zdarzeniom $A$ i $B$. \ans{$\left|A\right|=9$, $\left|B\right|=45-4=41$}
\item Oblicz $P(A)$ i $P(B)$ \ans{$P(A)=\frac{9}{45}, P(B)=\frac{41}{45}$}
\end{enumerate}
\item W gospodzie \emph{Złoty okoń} zorganizowano loterię, w której do sprzedania było 100 biletów, a wygrywał tylko jeden. Każdy mógł kupić najwyżej jeden bilet. Niech zdarzenie $A$ odpowiada sytuacji, w której Estella posiada los wygrywający. Jakie ma szanse na wygraną?
\begin{enumerate}%
\item Przedstaw przestrzeń zdarzeń elementarnych. \ans{$\Omega=\{\omega_n|n=1,2,\ldots,100\}$}%
\item Czy w tej przestrzeni wszystkie zdarzenia elementarne są jednakowo prawdopodobne? \ans{Tak}%
\item Jaki jest rozmiar przestrzeni zdarzeń elementarnych? \ans{$\left|\Omega\right|=100$}%
\item Zdefiniuj zdarzenie $A$ jako zbiór zdarzeń elementarnych. \ans{$A=\{\omega_1$\}}%
\item Oblicz prawdpodobieństwo $P(A)$, dbając o to by jasno przedstawić tok rozumowania. \ans{$P(A)=\frac{1}{100}$}%
\end{enumerate}%
\item  Drewniane pale mają losową długość $L$ nie przekraczającą 12 m. Pale są przeznaczone do wbijania w ziemię, której skalna warstwa stanowiąca opór znajduje się na losowej głębokości $H$, nie większej niż 10 m. Zaproponować przestrzeń zdarzeń elementarnych dla tak opisanego doświadczenia. Zdefiniować przez odpowiednie zdarzenia elementarne następujące doświadczenia:
\begin{description}
\item[A] długość losowo wybranego pala będzie większa od głębokości skalnej warstwy;
\item[B] głębkość skalnej warstwy przekroczy 8 metrów;
\item[C] długość losowo wybranego pala przekroczy 8 metrów;
\item[D] $B\cap C$
\item[E] $B\cup C$
\item[F] $(B\cup C)\cap A'$
\end{description}
\item W gospodzie \emph{Pod Zielonym Smokiem} oferują sześć różnych dań obiadowych. Pięciu klientów wchodzi jeden po drugim do gospody
i~niezależnie od siebie zamawia posiłek. Niech zdarzenie $A$ odpowiada sytuacji, w~której pierwsze danie z menu zamówi dokładnie jedna osoba.%
\begin{enumerate}%
\item Przedstaw przestrzeń zdarzeń elementarnych. \ans{$\Omega=\{\omega_{i_1,\ldots,i_5}|i_j=1,2,\ldots,6\}$}%
\item Czy w tej przestrzeni wszystkie zdarzenia elementarne są jednakowo prawdopodobne? \ans{Tak}%
\item Jaki jest rozmiar przestrzeni zdarzeń elementarnych? \ans{$\left|\Omega\right|=6^5$}%
\item Zdefiniuj zdarzenie $A$ jako zbiór zdarzeń elementarnych. \ans{$A=\{\omega_{i_1,\ldots,i_5}|\exists j: i_j=1 \land \forall k\neq j: i_k\neq 1\}$, $\left|A\right|=5\cdot5^4$}%
\item Oblicz prawdpodobieństwo $P(A)$, dbając o to by jasno przedstawić tok rozumowania. \ans{$P(A)=\frac{5^5}{6^5}=\frac{5}{6}^5\approx 0{,}40$}%
\end{enumerate}%
\item Oblicz, ile jest liczb naturalnych sześciocyfrowych, w zapisie których występuje dokładnie trzy razy cyfra 0 i dokładnie raz występuje cyfra 5. \ans{${8\choose 2}3\frac{5!}{3!}+8\cdot\left(\frac{5!}{3!2!}+\frac{5!}{3!}\right)=1920$}
\item Partia towaru składa się ze 100 elementów, wśród których 5 jest wadliwych. Poddajemy kontroli 50 elementów. Partię przyjmujemy, jeśli wśród kontrolowanych elementów jest nie więcej niż jeden wadliwy. Niech zdarzenie $A$ odpowiada przyjęciu partii.
\begin{enumerate}%
\item Przedstaw przestrzeń zdarzeń elementarnych. \ans{$\Omega=\{\omega_J|\left|J\right|=50 \land J\subset\{1,2,\ldots,100\}\}$}%
\item Czy w tej przestrzeni wszystkie zdarzenia elementarne są jednakowo prawdopodobne? \ans{Tak}%
\item Jaki jest rozmiar przestrzeni zdarzeń elementarnych? \ans{$\left|\Omega\right|={100\choose 50}$}%
\item Zdefiniuj zdarzenie $A$ jako zbiór zdarzeń elementarnych. \ans{$A=\{\omega_J|\left|J\cap \{1,2,\ldots,5\}\right|\leq 1\}$}%
\item Oblicz prawdpodobieństwo $P(A)$, dbając o to by jasno przedstawić tok rozumowania. \ans{$\left|A\right|={5\choose 1}{95\choose 49}+{95\choose 50}, P(A)\approx0{,}181$}%
\end{enumerate}%
\item Winda rusza z siedmioma pasażerami i zatrzymuje się na dziesięciu piętrach. Niech zdarzenie $A$ odpowiada sytuacji, w której żadnych dwóch pasażerów nie opuści windy na tym samym piętrze.
\begin{enumerate}%
\item Przedstaw przestrzeń zdarzeń elementarnych. \ans{$\Omega=\{\omega_{i_1,\ldots,i_7}|i_j=1,2,\ldots,10\}$}%
\item Czy w tej przestrzeni wszystkie zdarzenia elementarne są jednakowo prawdopodobne? \ans{Tak}%
\item Jaki jest rozmiar przestrzeni zdarzeń elementarnych? \ans{$\left|\Omega\right|=10^7$}%
\item Zdefiniuj zdarzenie $A$ jako zbiór zdarzeń elementarnych. \ans{$A=\{\omega_{i_1,\ldots,i_7}|\forall j,k\colon j\neq k\to i_j\neq i_k\}, \left|A\right|=\frac{10!}{3!}=604800$}%
\item Oblicz prawdpodobieństwo $P(A)$, dbając o to by jasno przedstawić tok rozumowania. \ans{$P(A)=\frac{604800}{10^7}=0{,}06$}%
\end{enumerate}%
\item Dwudziestoosobowa grupa studencka, w której jest 6 kobiet, otrzymała 5 biletów do teatru. Bilety rozdziela się drogą losowania. Niech zdarzenie $A$ odpowiada sytuacji, w której wśród posiadaczy biletów znajdą się dokładnie trzy kobiety.
\begin{enumerate}%
\item Przedstaw przestrzeń zdarzeń elementarnych. \ans{$\Omega=\{\omega_J|\left|J\right|=5 \land J\subset\{1,2,\ldots,20\}\}$}%
\item Czy w tej przestrzeni wszystkie zdarzenia elementarne są jednakowo prawdopodobne? \ans{Tak}%
\item Jaki jest rozmiar przestrzeni zdarzeń elementarnych? \ans{$\left|\Omega\right|={20\choose 5}=15504$}%
\item Zdefiniuj zdarzenie $A$ jako zbiór zdarzeń elementarnych. \ans{$A=\{\omega_J|\left|J\cap \{1,2,\ldots,6\}\right|=3\}, \left|A\right|={6\choose 3}{14\choose 2}=1820$}%
\item Oblicz prawdpodobieństwo $P(A)$, dbając o to by jasno przedstawić tok rozumowania. \ans{$P(A)=0{,}12$}%
\end{enumerate}%
\end{enumerate}%
\clearpage
\section{Prawdopodobieństwo warunkowe i niezależność zdarzeń}
\begin{enumerate}
\item W sklepie są sprzedawane baterie dwóch firm A i B. Firma A dostarcza do sklepu dwa razy więcej baterii niż firma B. Braki wśród baterii tych firm stanowią odpowiednio $0{,}9\%$ i $1{,}4\%$. Kupujemy jedną baterię. Jakie jest prawdopodobieństwo kupienia baterii dobrej?
\item Wiadomo, że 90\% produkcji spełnia wymagania techniczne. Przeprowadzono dodatkową kontrolę, przy której mogły być popełnione pewne błędy, a mianowicie: element wadliwy mógł zostać sklasyfikowany jako dobry z prawdopodobieństwem $0{,}05$, a element dobry mógł zostać sklasyfikowany jako wadliwy z prawdopodobieństwem $0{,}02$. Obliczyć prawdopodobieństwo tego, że element, który został sklasyfikowany jako dobry, faktycznie jest dobry. \ans{$P(S|K)=\frac{P(K|S)P(S)}{P(K)}=\frac{(1-P(K'|S))P(S)}{(1-P(K'|S))P(S)+P(K|S')P(S')}=\frac{(1-0{,}02)0{,}9}{(1-0{,}02)0{,}9+0{,}05\cdot0{,}1}\approx0{,}994$}
\item O pewnym roczniku studentów wiadomo, że dzielą się na grupy:
\begin{description}
\item[$G_1$] $5\%$ potrafiące odpowiedzieć na wszystkie pytania;
\item[$G_2$] $30\%$ potrafiące odpowiedzieć na 70\% pytań;
\item[$G_3$] $40\%$ potrafiące odpowiedzieć na 60\% pytań;
\item[$G_4$] $25\%$ potrafiące odpowiedzieć na 50\% pytań;
\end{description}
Wybrano w sposób przypadkowy jednego studenta. Obliczyć:
\begin{enumerate}
\item prawdopodobieństwo, że odpowie on na pytanie; \ans{$P(O)=\sum P(O|G_i)P(G_i)=0{,}625$}
\item prawdopodobieństwo, że należy do grupy drugiej, jeżeli wiadomo, że odpowiedział na pytanie. \ans{$P(G_2|O)=\frac{P(O|G_2)|P(G_2)}{P(O)}=0{,}336$}
\end{enumerate}
\item Prawdopodobieństwo przekazania sygnału przez jeden przekaźnik jest równe $0{,}9$. Przekaźniki działają niezależnie, tzn. awaria jednego z nich nie ma wpływu na działanie pozostałych.
Obliczyć prawdopodobieństwo przekazania sygnału:
\begin{enumerate}
\item przy połączeniu szeregowym dwóch przekaźników (inaczej: muszą działać oba przekaźniki);
\item przy połączeniu równoległym dwóch przekaźników (inaczej: wystarczy, że chociaż jeden przekaźnik będzie działał);
\item przy połączeniu szeregowym trzech przekaźników;
\item przy połączeniu równoległym trzech przekaźników;
\end{enumerate}
\item Telegraficzne przekazywanie informacji odbywa się metodą nadawania
sygnałów kropka-kreska. Statystyczne właściwości zakłóceń są takie, że błędy
następują przeciętnie w 2 przypadkach na 5 przy nadawaniu sygnału kropka i w 1
przypadku na 3 przy nadawaniu sygnału kreska. Wiadomo, że ogólny stosunek
liczby nadawanych sygnałów kropka do sygnałów kreska jest $\frac{5}{3}$.
\begin{enumerate}
\item Odebrano kropkę. Jakie jest prawdopodobieństwo, że nadano kropkę? \ans{$P(N_\cdot|O_\cdot)=\frac{P(O_\cdot|N_\cdot)P(N_\cdot)}{P(O_\cdot|N_\cdot)P(N_\cdot)+P(O_\cdot|N_-)P(N_-)}=\frac{\frac{3}{5}\frac{5}{8}}{\frac{3}{5}\frac{5}{8}+\frac{1}{3}\frac{3}{8}}=\frac{3}{4}$}
\item Odebrano kreskę. Jakie jest prawdopodobieństwo, że nadano kreskę?
\end{enumerate}

\item Brzeczka piwna za pomocą systemu pomp (na rysunku poniżej: czarne koła)
płynie z kadzi warzelnej (punkt $X$) do pojemnika fermentacyjnego (punkt $Y$)
	zgodnie ze schematem pokazanym na poniższym rysunku. Niestety,
	prawodpodobieństwo awarii każdej z pomp w trakcie przepompowywania brzeczki
	wynosi $0{,}1$ i jest stałe, i~niezależne od sprawności pozostałych pomp.
	Niech zdarzenie $A$ odpowiada przepompowaniu brzeczki z~kadzi do pojemnika,
	tzn. istnieniu jakiejkolwiek ścieżki pomiędzy punktami $X$ i $Y$ bez
	uszkodzonej pompy.
	\begin{tikzpicture}
	\draw (0,0) node[anchor=east] {$X$} -- (.5,0) |- (1,.5) -| (1.5,0) -- (2,0) |- (2.5,.5) -| (3,0) -- (3.5,0) node[anchor=west] {$Y$};
\draw (.5,0) |- (1,-.5) -| (1.5,0) -- (2,0) |- (2.5,-.5) -| (3,0);
\filldraw[black] (1,.5) circle (0.1);
\filldraw[black] (1,-.5) circle (0.1);
\filldraw[black] (2.5,.5) circle (0.1);
\filldraw[black] (2.5,-.5) circle (0.1);
\end{tikzpicture}
\begin{enumerate}
\item Przedstaw przestrzeń zdarzeń elementarnych. \ans{$\Omega=\{\omega_{i_1,\ldots,i_4}|i_j\in\{0,1\}\}$}
\item Czy w tej przestrzeni wszystkie zdarzenia elementarne są jednakowo prawdopodobne? \ans{Nie}
\item Jaki jest rozmiar przestrzeni zdarzeń elementarnych? \ans{$\abs{\Omega}=2^4$}
\item Zdefiniuj zdarzenie $A$ jako zbiór zdarzeń elementarnych. \ans{$A=\{
	\omega_{1111},\omega_{1110},\omega_{1101},
		\omega_{1011},\omega_{1010},\omega_{1001},
		\omega_{0111},\omega_{0110},\omega_{0101}
	\}$}
	\item Oblicz prawdpodobieństwo $P(A)$, dbając o to by jasno przedstawić tok rozumowania. \ans{$P(A)=0{,}9^4+4\cdot0{,}9^3\cdot0{,}1+4\cdot0{,}9^2\cdot0{,}1^2=0{,}9801=(0{,}9+0{,}9-0{,}9^2)^2$, np. $P(\omega_{1111})=P(A_1)P(A_2)P(A_3)P(A_4)$}
\end{enumerate}
\clearpage
\item Merry wybrał się do Starego Lasu zbierać pewne rośliny o wielce pożądanych właściwościach na nadchodzącą Sobótkę.
Niestety, Merry nie jest zbyt biegły w rozpoznawaniu roślin.
Jedyne co wie, to że ma zbierać lejkowate kwiaty w jednym z trzech kolorów: białe, różowe lub niebieskie.
Czego Merry'emu nie powiedziano to, że wśród białych kwiatów tylko $30\%$ posiada pożądane właściwości, wśród różowych $55\%$, a wśród niebieskich
aż $90\%$.
Wracając do domu Merry zauważył, że wśród kwiatów, które zebrał liczba białych do liczby różowych ma się jak $3:2$, a~liczba różowych do
liczby niebieskich jak $2:1$. Merry dla zabawy zamknął oczy i na chybił-trafił wybrał jeden z~zebranych kwiatów.
 
\begin{enumerate}
	\item Jakie jest prawdopodobieństwo, że wybrany kwiat jest niebieski?
	\item Jakie jest prawdopodobieństwo, że wybrany kwiat jest biały lub różowy?
	\item Jakie jest prawdopodobieństwo, że wybrany kwiat ma pożądane właściwości, jeżeli wiadomo, że jest niebieski?
	\item Jakie jest prawdopodobieństwo, że wybrany kwiat jednocześnie ma pożądane właściwości i jest niebieski?
	\item Jakie jest prawdopodobieństwo, że wybrany kwiat ma pożądane właściwości, jeżeli nie wiadomo jaki ma kolor?
	\item Co by było gdyby Merry nie wiedział, jakiego koloru wylosował kwiat, ale wiedział, że wybrany kwiat nie ma pożądanych
		właściwości: jakie byłoby wtedy prawdopodobieństwo, że kwiat będzie biały lub niebieski?
\end{enumerate}


%\item
%Krasnoludowie w Morii prowadzą okresowy przegląd wagoników używanych do przewozu urobku z kopalni do huty.
%Frar jako mniej doświadczony został przydzielony do sprawdzenia wagoników jako pierwszy.
%Spośród 100 wagoników, 70 z nich uznał jako sprawne, 20 jako niesprawne, a co do 10 nie był pewien.
%Jego starszy kolega Gamil ocenił, że Frar pomylił się w jednym przypadku na 10 przy wagonikach sprawnych i w 2 przypadkach na 10 przy wagonikach niesprawnych.
%Wśród wagoników, których Frar nie potrafił zdiagnozować, Gamil uznał 4 za sprawne, a 6 za niesprawne.
%Na końcu przyszedł ich przełożony Nali, całkowicie losowo wybrał jeden wagonik i zdiagnozował go jako sprawny.
%\begin{enumerate}
%\item  Zdefiniuj przestrzeń zdarzeń elementarnych dla doświadczenia polegającego na ocenie tego wagonika przez Frara i Gamila. 
%\item  Jaki jest rozmiar przestrzeni zdarzeń elementarnych? 
%\item  Czy w tej przestrzeni wszystkie zdarzenia elementarne są równoprawdopodobne? 
%\item  Nazwij i zdefiniuj jako zbiory zdarzeń elementarnych następujące zdarzenia opisane słownie: 
%\begin{itemize}
%\item wagonik został uznany za sprawny przez Frara;
%\item wagonik został uznany za niesprawny przez Frara;
%\item Frar nie potrafił ocenić wagonika;
%\item wagonik został uznany za sprawny przez Gamila;
%\item wagonik został uznany za niesprawny przez Gamila.
%\end{itemize}
%\item  Jakie jest prawdopodobieństwo, że wybrany wagonik został uznany za sprawny przez Frara?  
%\item  Jakie jest prawdopodobieństwo, że wybrany wagonik został uznany za sprawny przez obu krasnoludów? 
%\item  Jakie jest prawdopodobieństwo, że wybrany wagonik został uznany za sprawny przez Gamila? 
%\item  Jakie jest prawdopodobieństwo, że wagonik został uznany za sprawny przez Frara, jeżeli wiadomo, że został uznany za dobry przez Gamila? 
%\item  Czy zdarzenia: \emph{Frar nie potrafił ocenić wagonika} oraz \emph{wagonik został uznany za sprawny przez Gamila} są zdarzeniami niezależnymi? 
%\end{enumerate}
\item Kuce w Oatbarton od czasu do czasu zapadają na Straszną Chorobę Kuców.
Na szczęście zmyślni, hobbicy farmerzy wymyślili sposób na wykrywanie choroby we wczesnym stadium i izolowanie chorych zwierząt.
Niestety, sposób nie jest idealny: prawdopodobieństwo, że zwierzę zostatnie uznane za chore, wynosi odpowiednio $0,85$ dla zwierzęcia chorego i $0,07$ dla zwierzęcia zdrowego.
Z danych historycznych wiadomo, że prawdopodobieństwo zapadnięcia na Straszną Chorobę Kuców wynosi $0,2$ dla każdego kuca.
Rozpatrujemy doświadczenie polegające na badaniu kuca Ostrouchego.

\begin{enumerate}
\item Zdefiniuj przestrzeń zdarzeń elementarnych. 
\item Jaki jest rozmiar przestrzeni zdarzeń elementarnych? 
\item Czy w tej przestrzeni wszystkie zdarzenia elementarne są równoprawdopodobne? Odpowiedź uzasadnij. 
\item Nazwij i zdefiniuj jako zbiory zdarzeń elementarnych następujące zdarzenia opisane słownie: 
\begin{itemize}
\item Ostrouchy jest chory.
\item Ostrouchy jest zdrowy.
\item Test wskazał, że Ostrouchy jest chory.
\item Test wskazał, że Ostrouchy jest zdrowy.
\end{itemize}
Wykorzystaj tak przypisane nazwy w dalszych obliczeniach!
\item Bez przeprowadzania testu, jakie jest prawdopodobieństwo, że Ostrouchy jest zdrowy? 
\item Jakie jest prawdopodobieństwo, że jednocześnie Ostrouchy jest zdrowy i test wskazał, że Ostrouchy jest zdrowy? 
\item Jakie jest prawdopodobieństwo, że test wskaże, że Ostrouchy jest zdrowy? 
\item Jakie jest prawdopodobieństwo, że Ostrouchy jest zdrowy, jeżeli test wskazał, że Ostrouchy jest zdrowy? 
\item Czy zdarzenia: \emph{Ostrouchy jest zdrowy} oraz \emph{test wskazał, że Ostrouchy jest zdrowy} są niezależne? Odpowiedź uzasadnij odpowiednim rachunkiem. 
\end{enumerate}

\end{enumerate}

\clearpage
\section{Zmienne losowe jednowymiarowe}
\begin{enumerate}
\item Rozważamy doświadczenie polegające na jednokrotnym rzucie uczciwą kostką sześciościenną
\begin{enumerate}
\item Zaproponuj zmienną losową $X$ odpowiednią do tego doświadczenia. \ans{$X\colon\Omega\to\{1,\ldots,6\}$}
\item Podaj rozkład prawdopodobieństwa zmiennej losowej $X$. \ans{$P(X=i)=\frac{1}{6}$}
\item Podaj dystrybuantę zmiennej losowej $X$ i narysuj jej wykres.
\item Oblicz $P(X<4)$ korzystając z rozkładu prawdopodobieństwa. \ans{$P(X<4)=P(X=1)+P(X=2)+P(X=3)$}
\item Oblicz $P(X<4)$ korzystając z dystrybuanty. \ans{$P(X<4)=F_X(4)$}
\item Oblicz $P(X>2)$ korzystając z rozkładu prawdopodobieństwa. \ans{$P(X>2)=P(X=3)+\ldots+P(X=6)$}
\item Oblicz $P(X>2)$ korzystając z dystrybuanty. \ans{$P(X>2)=1-P(X\leq 2)=1-P(X<3)=1-F(3)$}
\item Oblicz wartość średnią. \ans{$EX=\frac{21}{6}=3{,}5$}
\item Oblicz wariancję. \ans{$D^2X=\frac{1}{6}((-2{,}5)^2+(-1{,}5)^2+(-0{,}5)^2+2{,}5^2+1{,}5^2+0{,}5^2)$}
\end{enumerate}
\item Prawdopodobieństwo trafienia do celu w jednym strzale jest równie $\frac{1}{5}$. Niech $X$ przyjmuje wartość 1 jeżeli udało się trafić i 0 w przeciwnym przypadku.
\begin{enumerate}
\item Podaj rozkład zmiennej losowej $X$. \ans{$P(X=1)=p\quad P(X=0)=1-p$}
\item Oblicz średnią liczbę celnych strzałów. \ans{$EX=p=0{,}2$}
\item Oblicz odchylenie standardowe zmiennej losowej $X$. \ans{$DX=\sqrt{p(1-p)}=0{,}4$}
\item Podaj najbardziej prawdopodobną wartość zmiennej losowej $X$. \ans{$0$}
\end{enumerate}
\item Rozważamy doświadczenie polegające na obserwacji sumy oczek na dwóch uczciwych kostkach sześciościennych
\begin{enumerate}
\item Zaproponuj zmienną losową $X$ odpowiednią do tego doświadczenia. \ans{$X\colon\Omega\to\{1,\ldots,12\}$}
\item Podaj rozkład prawdopodobieństwa zmiennej losowej $X$. \ans{$(1,2,3,4,5,6,5,4,3,2,1)/36$}
\item Podaj dystrybuantę zmiennej losowej $X$ i narysuj jej wykres.
\item Oblicz $P(X>5)$ korzystając z dystrybuanty.
\item Oblicz wartość średnią. \ans{$7$}
\item Oblicz odchylenie standardowe. \ans{$\sqrt{D^2X}=\sqrt{54{,}83-49}=2{,}42$}
\end{enumerate}
\item Bilbo bierze udział w grze, w której punkty zdobywa się za trafianie kamykami do celu. Każdemu zawodnikowi przysługuje maksymalnie pięć rzutów, przy czym:
\begin{itemize}
\item na początku każdy zawodnik ma jeden punkt;
\item każdy trafiony rzut powoduje podwojenie liczby posiadanych punktów;
\item rzut nietrafiony oznacza koniec gry dla danego zawodnika.
\end{itemize}
Prawdopodobieństwo, że Bilbo w pojedynczym rzucie trafi do celu wynosi $0{,}7$. Niech $X$ będzie zmienną losową oznaczającą liczbę punktów zdobytych przez
Bilba.
\begin{enumerate}
\item Podaj funkcję prawdopodobieństwa zmiennej losowej $X$.
\item Oblicz dystrybuantę zmiennej losowej $X$.
\item Oblicz $EX$.
\item Oblicz wariancję zmiennej losowej $X$.
\item Oblicz $P(5\leq X\leq 30)$.
\end{enumerate}
\item Rozważmy funkcję \[f(x)=\begin{cases} 0 & x<0 \\ e^{-x} & x\geq 0\end{cases} \]
\begin{enumerate}
\item Udowodnij, że jest to funkcja gęstości prawdopodobieństwa pewnej zmiennej losowej $X$. \ans{nieujemna, $\int_{-\infty}^\infty f(x)dx=0+\int_0^\infty f(x)dx=1$}
\item Oblicz dystrybuantę zmiennej losowej $X$ i narysuj jej wykres.
\item Oblicz $P(X<\frac{1}{2})$
\item Oblicz $P(1\leq X<2)$
\item Oblicz wartość średnią
\end{enumerate}
\item Rozważmy funkcję \[f(x)=\begin{cases} Ax^{-4} & \left|x\right|\geq 1 \\ 0 & \text{wpp} \end{cases} \]
\begin{enumerate}
\item Dla jakiej wartości stałej $A$ ta funkcja jest gęstością prawdopodobieństwa pewnej zmiennej losowej $X$? \ans{$\frac{3}{2}$}
\item Oblicz dystrybuantę zmiennej losowej $X$ i narysuj jej wykres. \ans{\[\begin{cases}-\frac{x^{-3}}{2} & x\leq-1\\\frac{1}{2} & -1<x\leq1\\1-\frac{x^{-3}}{2} & x>1 \end{cases}\]}
\item Oblicz prawdopodobieństwo, że zmienna losowa $X$ przyjmie wartość większą niż 2. \ans{$\frac{1}{16}$}
\item Oblicz prawdopodobieństwo, że zmienna losowa $X$ przyjmie wartość równą 2. \ans{$0$}
\item Oblicz wartość średnią. \ans{$0$}
\item Oblicz wariancję. \ans{$3$}
\end{enumerate}
\item Dana jest następująca gra: gracz rzuca uczciwą kostką sześciościenną tak długo, dopóki nie wyrzuci piątki bądź
szóstki, ale więcej niż trzy razy. Jeżeli uda mu się wyrzucić założoną liczbę oczek w $k$-tym rzucie, wygrywa $5-k$
zł, w przeciwnym razie nie wygrywa nic. Niech zmienna losowa $X$ odpowiada wysokości wygranej. Podaj, dbając
by przedstawić tok rozumowania:
\begin{enumerate}
\item funkcję prawdopodobieństwa zmiennej losowej $X$;
\item dystrybuatnę zmiennej losowej $X$;
\item prawdopodobieństwo, że gracz wygra nie mniej niż 3, a nie więcej niż 4 zł;
\item średnią wartość wygranej;
\item odchylenie standardowe zmiennej losowej $X$.
\end{enumerate}
\item Linia technologiczna składająca samochody składa 5 sztuk w ciągu godziny. Przy okazji składania każdego
z samochodów istnieje prawdopodobieństwo $0{,}1$, że maszyna montująca drzwi kierowcy ulegnie rozregulowaniu
i będzie rysować lakier. Co gorsza, rozregulowanie jest trwałe w tym sensie, że dopóki technik nie wyreguluje
maszyny wszystkie montowane drzwi będą rysowane. W związku z kryzysem firma postanowiła wprowadzić
oszczędności i technik regulujący maszyny dokonuje kontroli tylko raz na godzinę. Porysowanie lakieru na jednych
drzwiach to koszt 600 zł. Niech zmienna $X$ odpowiada kwocie, która firma straciła w ciągu godziny w wyniku
porysowania lakieru przez maszynę montującą drzwi. Podaj, dbając by przedstawić tok rozumowania:
\begin{enumerate}
\item funkcję prawdopodobieństwa zmiennej losowej $X$;
\item dystrybuatnę zmiennej losowej $X$;
\item średnią stratę;
\item odchylenie standardowe zmiennej losowej $X$;
\item prawdopodobieństwo, że firma straci przynajmniej 1000 zł w ciągu godziny
\end{enumerate}
\end{enumerate}

\clearpage
\section{Dyskretne rozkłady prawdopodobieństwa}
\begin{enumerate}
\item Prawdopodobieństwo trafienia do celu w jednym strzale jest równie $\frac{1}{5}$. Niech $X$ oznacza liczbę strzałów celnych w wykonanej serii 5 niezależnych strzałów. 
\begin{enumerate}
\item Podaj rozkład zmiennej losowej $X$. \ans{$P(X=k)={5 \choose k}\frac{4^{5-k}}{5^5}$}
\item Oblicz prawdopodobieństwo, że liczba strzałów celnych będzie nie mniejsza niż 2. \ans{$P(X\geq 2)=1-P(X=0)-P(X=1)=1-\frac{4^5}{5^5}-5\cdot\frac{4^4}{5^5}=\frac{821}{3125}\approx0{,}263$}
\item Oblicz średnią liczbę celnych strzałów. \ans{$EX=np=1$}
\item Oblicz odchylenie standardowe zmiennej losowej $X$. \ans{$DX=\sqrt{np(1-p)}=\sqrt{\frac{4}{5}}$}
\item Podaj najbardziej prawdopodobną liczbę celnych strzałów. \ans{$\lfloor(n+1)p\rfloor=1$}
\end{enumerate}
\item Linia 64 jeżdżąca na trasie Literacka--Kacza jest obsługiwana przez 7 autobusów, które psują się przypadkowo i niezależnie od siebie. Każdy autobus może w ciągu całego dnia zepsuć się z~prawdopodobieństwem $0{,}25$. Niech $X$ oznacza liczbę autobusów, które w ciągu dnia uległy awarii i musiały zjechać do zajezdni.
\begin{enumerate}
\item Podaj rozkład zmiennej losowej $X$.
\item Oblicz prawdopodobieństwo, że w ciągu całego dnia zepsują się przynajmniej 3 autobusy.
\item Oblicz średnią liczbę zepsutych autobusów.
\item Oblicz odchylenie standardowe zmiennej losowej $X$.
\item Podaj najbardziej prawdopodobną liczbę zepsutych autobusów. \ans{$(n+1)p-1=1, (n+1)p=2$}
\end{enumerate}
\item W Minas Tirith gromadzą zapasy na wypadek oblężenia. Jeżeli mięso,
pakowane w beczki, jest źle wysuszone może się zepsuć. Prawdopodobieństwo, że
tak się stanie wynosi $0{,}0045$ niezależnie dla każdej beczki. W~piwnicach zgromadzono 1000 beczek z mięsem,
niech $X$ oznacza liczbę beczek z zepsutym mięsem.  \begin{enumerate}
\item Podaj rozkład zmiennej losowej $X$. \ans{$P(X=k)={1000\choose k}p^k(1-p)^{1000-k}\approx\frac{\lambda^k}{k!}e^{-\lambda} \quad \lambda=1000\cdot0{,}0045=4{,}5$}
\item Oblicz prawdopodobieństwo, że zepsuje się nie więcej niż 5 beczek. \ans{$P(X\leq 5)=0{,}70290$}
\item Oblicz średnią liczbę zepsutych beczek. \ans{$EX=4{,}5$}
\item Oblicz odchylenie standardowe zmiennej losowej $X$. \ans{$DX=\sqrt{4{,}5}$}
\end{enumerate}
\item Hobbici znani są ze swojej intensywnej i obfitej korespondencji. W Michael
Delving mieszka 500 hobbitów, a~w~ciągu jednego dnia do urzędu pocztowego
przychodzi $X$ hobbitów. Hobbici do urzędu pocztowego przychodzą niezależnie od
siebie i~z~równym prawdopodobieństwem $p$, a~każdego dnia jest ich tam
\emph{średnio} 75. Załóż, że zmienna losowa $X$ ma rozkład dwumianowy.
\begin{enumerate}
\item Oblicz wartość prawdopodobieństwa $p$ przyjścia do urzędu przez pojedynczego hobbita.
\item Podaj  (w formie funkcji prawdopodobieństwa) rozkład prawdopodobieństwa zmiennej losowej $X$.
\item Ile różnych wartości może przyjąć dystrybuanta zmiennej losowej $X$? Odpowiedź uzasadnij.
\item Jaką wartość przyjmie dystrybuanta zmiennej losowej $X$ w punkcie 1000?
\item Oblicz prawdopodobieństwo, że przez cały dzień do urzędu nikt nie przyjdzie.
\item Podaj wartość $EX$.
\item Oblicz odchylenie standardowe zmiennej losowej $X$.
\item Oblicz najbardziej prawdopodobną liczbę hobbitów, którzy przyjdą w ciągu dnia do urzędu.
\end{enumerate}
\item Słoń Nino przechodzi przez skład porcelany zawierający 4~regały, z~których każdy zawiera talerze warte 3000 zł.
	Każdy z regałów może zostać przewrócony przez Nina z~prawdopodobieństwem $0{,}2$.
	Niech zmienna $X$ oznacza wartość uszkodzonych talerzy (zakładając, że przerwócenie regału powoduje uszkodzenie się wszystkich zawartych na nim talerzy).
Podaj, dbając by przedstawić tok rozumowania:
\begin{enumerate}
\item funkcję prawdopodobieństwa zmiennej losowej $X$;
\item dystrybuatnę zmiennej losowej $X$;
\item prawdopodobieństwo, że słoń stłucze talerzy za co najmniej $5{,}5$ tys. zł.
\item średnią wartość straty;
\item odchylenie standardowe zmiennej losowej $X$.
\end{enumerate}
\newpage
\item W piwnicach Brandy Hallu zgromadzono 500 butelek soków na zimę.
Niestety, z poprzednich lat wiadomo, że każda butelka ma $0{,}5\%$ szans nie dotrwać zimy: spleśnieje, skwaśnieje itp.
Niech $X$ będzie zmienną losową oznaczającą liczbę zepsutych butelek soku.
Niech $Y$ będzie zmienną losową oznaczającą liczbę zdatnych do spożycia butelek soku, to znaczy $X+Y=500$.

\begin{enumerate}
\item Podaj funkcję prawdopodobieństwa zmiennej losowej $X$.
\item Podaj wartość średnią i odchylenie standardowe zmiennej losowej $X$.
\item Podaj funkcję prawdopodobieństwa zmiennej losowej $Y$.
\item Podaj wartość średnią i odchylenie standardowe zmiennej losowej $Y$.
\item Podaj najbardziej prawdopodobną liczbę zepsutych butelek.
\item Oblicz prawdopodobieństwo, że liczba przynajmniej 496 butelek soku będzie się nadawało do spożycia.
\item Oblicz prawdopodobieństwo, że liczba zepsutych butelek będzie różniła się od oczekiwanej liczby zepsutych butelek o mniej niż odchylenie standardowe.
\item Oszacuj (z dokładnością do $0{,}1$) prawdopodobieństwo, że liczba butelek nadających się do spożycia przekroczy 300. Odpowiedź uzasadnij.
\end{enumerate}
\item W piwnicy Bamfurlong znajduje się 10 słoików szczególnie cennej konfitury ze świecących gigantycznych pieczarek.
Właściciel przechowuje je na Święto Zimowe, ale nie wie, że ze względu na bliskość rzeki, każdy słoik ma 20\% szans spleśnieć do tego czasu!
Niech $X$ będzie zmienną losową oznaczającą liczbę słoików, które nie zapleśniały do Święta.
Niech $Y$ będzie zmienną losową oznaczającą liczbę słoików, które spleśniały.
Inaczej: $X+Y=10$.

\begin{enumerate}
\item Podaj funkcję prawdopodobieństwa zmiennej losowej $X$.
\item Podaj wartość średnią i odchylenie standardowe zmiennej losowej $X$.
\item Podaj funkcję prawdopodobieństwa zmiennej losowej $Y$.
\item Podaj wartość średnią i odchylenie standardowe zmiennej losowej $Y$.
\item Podaj najbardziej prawdopodobną liczbę zepsutych słoików.
\item Oblicz prawdopodobieństwo, że przynajmniej 8 słoików będzie się nadawało do spożycia.
\item Oblicz prawdopodobieństwo, że liczba zepsutych słoików będzie różniła się od oczekiwanej liczby zepsutych słoików o mniej niż odchylenie standardowe.
\end{enumerate}


\end{enumerate}
\vfill
{
	\footnotesize
\noindent\begin{tabular}{|r|r|r|r|r|r|r||r|r|r|r|r|r|r|}
\hline
\multicolumn{14}{|c|}{Rozkład Poissona} \\
\diagbox{$\lambda$}{$k$} & \textbf{0}	& \textbf{1}	& \textbf{2}	& \textbf{3}	& \textbf{4}	& \textbf{5} & \diagbox{$\lambda$}{$k$} & \textbf{0}	& \textbf{1}	& \textbf{2}	& \textbf{3}	& \textbf{4}	& \textbf{5}\\
\hline
\textbf{0,5}	& 0,6065	& 0,3033	& 0,0758	& 0,0126	& 0,0016	& 0,0002 &	\textbf{5,5}	& 0,0041	& 0,0225	& 0,0618	& 0,1133	& 0,1558	& 0,1714 \\
\hline                                                                                                          
\textbf{1,0}	& 0,3679	& 0,3679	& 0,1839	& 0,0613	& 0,0153	& 0,0031 &	\textbf{6,0}	& 0,0025	& 0,0149	& 0,0446	& 0,0892	& 0,1339	& 0,1606 \\
\hline                                                                                                          
\textbf{1,5}	& 0,2231	& 0,3347	& 0,2510	& 0,1255	& 0,0471	& 0,0141 &	\textbf{6,5}	& 0,0015	& 0,0098	& 0,0318	& 0,0688	& 0,1118	& 0,1454 \\
\hline                                                                                                          
\textbf{2,0}	& 0,1353	& 0,2707	& 0,2707	& 0,1804	& 0,0902	& 0,0361 &	\textbf{7,0}	& 0,0009	& 0,0064	& 0,0223	& 0,0521	& 0,0912	& 0,1277 \\
\hline                                                                                                          
\textbf{2,5}	& 0,0821	& 0,2052	& 0,2565	& 0,2138	& 0,1336	& 0,0668 &	\textbf{7,5}	& 0,0006	& 0,0041	& 0,0156	& 0,0389	& 0,0729	& 0,1094 \\
\hline                                                                                                          
\textbf{3,0}	& 0,0498	& 0,1494	& 0,2240	& 0,2240	& 0,1680	& 0,1008 &	\textbf{8,0}	& 0,0003	& 0,0027	& 0,0107	& 0,0286	& 0,0573	& 0,0916 \\
\hline                                                                                                          
\textbf{3,5}	& 0,0302	& 0,1057	& 0,1850	& 0,2158	& 0,1888	& 0,1322 &	\textbf{8,5}	& 0,0002	& 0,0017	& 0,0074	& 0,0208	& 0,0443	& 0,0752 \\
\hline                                                                                                         
\textbf{4,0}	& 0,0183	& 0,0733	& 0,1465	& 0,1954	& 0,1954	& 0,1563 &	\textbf{9,0}	& 0,0001	& 0,0011	& 0,0050	& 0,0150	& 0,0337	& 0,0607 \\
\hline                                                                                                         
\textbf{4,5}	& 0,0111	& 0,0500	& 0,1125	& 0,1687	& 0,1898	& 0,1708 &	\textbf{9,5}	& 0,0001	& 0,0007	& 0,0034	& 0,0107	& 0,0254	& 0,0483 \\
\hline                                                                                                        
\textbf{5,0}	& 0,0067	& 0,0337	& 0,0842	& 0,1404	& 0,1755	& 0,1755 &	\textbf{10,0}	& 0,0000	& 0,0005	& 0,0023	& 0,0076	& 0,0189	& 0,0378 \\
\hline
\end{tabular}
}

\clearpage
\section{Zmienne losowe typu ciągłego}
\begin{enumerate}
\item Autobusy linii 64 przyjeżdżają punktualnie co 20 minut na przystanek. Olga nie zna ich rozkładu jazdy, przychodzi więc na przystanek w całkowicie losowym momencie: jak jej się wyjdzie z domu. Niech $T$ oznacza czas oczekiwania Olgi na autobus.
\begin{enumerate}
\item Podaj rozkład zmiennej losowej $T$. \ans{$f(t)=\begin{cases}\frac{1}{20} & 0\leq t\leq 20 \\ 0 & \text{wpp} \end{cases}$}
\item Oblicz prawdopodobieństwo, że Olga będzie czekała mniej niż 5 minut. \ans{$P(T<5)=F(5)=0{,}25$}
\item Oblicz średni czas oczekiwania na autobus. \ans{$ET=10$}
\item Oblicz odchylenie standardowe zmiennej losowej $T$. \ans{$DT=\sqrt{\frac{400}{12}}$}
\end{enumerate}
\item Prawdopodobieństwo, że aparat fotograficzny nie zepsuje się w ciągu pierwszych pięciu miesięcy użytkowania wynosi $0{,}9$. Niech $X$ oznacza liczbę miesięcy bezawaryjnej pracy aparatu. Zakładamy, że zmienna $X$ ma rozkład wykładniczy.
\begin{enumerate}
\item Oblicz parametry tego rozkładu. \ans{$P(X>5)=\exp(-\frac{5}{\lambda}) \quad \lambda=-\frac{5}{\ln 0{,}9}$}
\item Oblicz prawdopodobieństwo bezawaryjnej pracy aparatu w ciągu 24 miesięcy. \ans{$P(X>24)=1-F(24)$}
\item Oblicz prawdopodobieństwo awarii aparatu w ciągu 36 miesięcy, jeżeli wiadomo, że przepracował bez awarii już 12 miesięcy. \ans{$P(X<36|X>12)=P(X<36-12)=F(24)$}
\end{enumerate}
\item W Minas Tirith gromadzą zapasy na wypadek oblężenia. Losowo wybrany worek zawiera $X$ kg mąki, gdzie $X$ jest zmienną losową o rozkładzie $N(20,2)$.
\begin{enumerate}
\item Utwórz zmienną losową $Y$ będącą standaryzowaną postacią zmiennej losowej $X$. \ans{$Y=\frac{X-20}{2}$}
\item Wyraź dystrybuantę zmiennej $X$ w zależności od dystrybuanty zmiennej $Y$. \ans{$F_X(x)=F_Y(\frac{x-20}{2})$}
\item Oblicz wartość średnią i odchylenie standardowe zmiennej losowej $Y$.
\item Oblicz prawdopodobieństwo, że w worku jest mniej niż 18 kg mąki. \ans{$P(X<18)=F_X(18)=F_Y(-1)=1-F_Y(1)=1-0{,}8413\approx 0{,}16$}
\item Oblicz prawdopodobieństwo, że w worku jest więcej niż $21{,}5$ kg mąki. \ans{$P(X>21{,}5)=1-F_X(21{,}5)=1-F_Y(0{,}75)=1-0{,}7734\approx 0{,}23$}
\item Oblicz prawdopodobieństwo, że ilość mąki w worku różni się od wartości oczekiwanej o nie więcej niż 1~kg. \ans{$P(19<X<21)=F_X(21)-F_X(19)=F_Y(0{,}5)-F_Y(-0{,}5)=2F_Y(0{,}5)-1=2\cdot0{,}6915-1=0{,}38$}
\end{enumerate}
\item Pociągi Kolei Wielkopolskich odjeżdżają w~kierunku Wągrowca co 60 min. Niestety, nastąpiła zmiana rozkładu, strona
internetowa padła, telefony nie działają, a~plakaty z~rozkładem porwał wiatr. Zakładając, że rozkład czasu przybycia pasażera na stację jest
jednostajny, a pociągi jeżdżą punktualnie, obliczyć:
\begin{enumerate}
	\item prawdopodobieństwo, że czas oczekiwania na pociąg przekroczy 10 min, ale będzie nie większe niż 55 min;
	\item średni czas oczekiwania na pociąg.
\end{enumerate}
\item Zakładając, że wysokość losowo wybranego psa rasy Husky pochodzi z~rozkładu normalnego o średniej $53{,}5$ cm i odchyleniu standardowym $1{,}5$ cm (inaczej: $N(53{,}5;1{,}5)$), oblicz jakie jest prawdodpodobieństwo, że taki pies ma wysokość powyżej $56$ cm.
\item Taczki wyprodukowane w zakładzie Hamfasta psują się średnio po trzech latach użytkowania.
Fredegar zakupił jedną z taczek, niech $X$ będzie zmienną losową o rozkładzie wykładniczym, reprezentującą czas w~latach bezawaryjnej pracy taczki Fredegara.

\begin{enumerate}
\item Podaj wartość średnią i odchylenie standardowe zmiennej losowej $X$.
\item Podaj dystrybuantę zmiennej losowej $X$.
\item Oblicz prawdopodobieństwo, że taczka zepsuje się przed upływem dwóch lat. 
\item Oblicz prawdopodobieństwo, że taczka będzie pracowała bez awarii co najmniej 6 lat, jeżeli wiadomo, że jest używana już od trzech lat.
\end{enumerate}
\item W hobbickiej wsi Oatbarton znajduje się wielki spichlerz na owies, w~którym okoliczni farmerzy gromadzą swoje plony.
Masa w tonach $X$ owsa zgromadzonego w spichlerzu bezpośrednio po zbiorach ma rozkład normalny $N(100, 20)$.
Spichlerz ma pojemność 130 ton, a~roczne potrzeby wsi wynoszą 85 ton.

W poniższych zadaniach wyniki podawaj z dokładnością do przynajmniej dwóch miejsc po przecinku.
\begin{enumerate}
\item Podaj $EX$ oraz $DX$.
\item Niech $Y$ będzie standaryzowaną postacią zmiennej losowej $X$. Podaj $EY$ oraz $DY$.
\item Oblicz prawdopodobieństwo, że hobbici wyprodukują dokładnie 100 ton owsa.
\item Oblicz prawdopodobieństwo, że hobbici wychodują najwyżej tyle owsa ile są w stanie zmagazynować.
\item Oblicz prawdopodobieństwo, że hobbici nie wyprodukują dostatecznie dużo owsa i będą musieli go dokupić.
\item Oblicz prawdopodobieńśtwo, że masa owsa wyprodukowanego przez hobbitów będzie się różnić od wartości oczekiwanej o nie więcej niż pół odchylenia standardowego.
\item Oblicz prawdopodobieńśtwo, że masa owsa wyprodukowanego przez hobbitów będzie się różnić od wartości oczekiwanej o nie więcej niż dwie wariancje.
\end{enumerate}

\item W Bucklandzie na jesieni przygotowują sery na zimę. W każdym ze 100 magazynów
zgromadzonych jest po 1000 serów. Prawdopodobieństwo zepsucia się pojedynczego
krążka sera wynosi $4\cdot 10^{-3}$ i~nie zależy od pozostałych serów. Niech
$i=1,2,\ldots,100$ oznacza numer magazynu i niech zmienna losowa $X_i$ oznacza
liczbę zepsutych serów składowanych w $i$-tym magazynie. Ponadto, niech zmienna
losowa $Y=X_1+X_2+\ldots+X_{100}$ odpowiada sumarycznej liczbie zepsutych serów
we wszystkich magazynach. Załóż, że:
\begin{itemize}
\item wszystkie zmienne $X_i$ mają taki sam rozkład;
\item zmienne $X_i$ są niezależne;
\item zmienna $Y$ ma rozkład normalny o wartości średniej 100 razy większej niż wartość średnia dowolnej ze zmiennych $X_i$;
\item zmienna $Y$ ma rozkład normalny o wariancji 100 razy większej niż wariancja dowolnej ze zmiennych $X_i$.
\end{itemize}

\begin{enumerate}
\item Podaj  (w formie funkcji prawdopodobieństwa) rozkład prawdopodobieństwa zmiennej losowej $X_1$.
\item Oblicz (z dokładnością do dwóch miejsc po przecinku) prawdopodobieństwo, że w magazynie nr 10 zepsują się mniej niż trzy sery.
\item Podaj średnią liczbę zepsutych serów w magazynie nr 15.
\item Podaj wartości $EY$ oraz $DY$.
\item Utwórz zmienną losową $Z$ będącą standaryzowaną postacią zmiennej losowej $Y$.
\item Oblicz (z dokładnością do dwóch miejsc po przecinku) prawdopodobieństwo, że w Bucklandzie zepsują się więcej niż 424 sery.
\end{enumerate}


\begin{tabular}{|r|r|r||r|r|r|}
\hline
\multicolumn{6}{|c|}{Rozkład normalny} \\
	& \textbf{0}	& \textbf{0,05}	& 	& \textbf{0}	& \textbf{0,05}\\
\hline
\textbf{0,0}	& 0,500	& 0,520	& \textbf{1,6}	& 0,945	& 0,951 \\
\hline
\textbf{0,1}	& 0,540	& 0,560	& \textbf{1,7}	& 0,955	& 0,960 \\
\hline
\textbf{0,2}	& 0,579	& 0,599	& \textbf{1,8}	& 0,964	& 0,968 \\
\hline
\textbf{0,3}	& 0,618	& 0,637	& \textbf{1,9}	& 0,971	& 0,974 \\
\hline
\textbf{0,4}	& 0,655	& 0,674	& \textbf{2,0}	& 0,977	& 0,980 \\
\hline
\textbf{0,5}	& 0,691	& 0,709	& \textbf{2,1}	& 0,982	& 0,984 \\
\hline
\textbf{0,6}	& 0,726	& 0,742	& \textbf{2,2}	& 0,986	& 0,988 \\
\hline
\textbf{0,7}	& 0,758	& 0,773	& \textbf{2,3}	& 0,989	& 0,991 \\
\hline
\textbf{0,8}	& 0,788	& 0,802	& \textbf{2,4}	& 0,992	& 0,993 \\
\hline
\textbf{0,9}	& 0,816	& 0,829	& \textbf{2,5}	& 0,994	& 0,995 \\
\hline
\textbf{1,0}	& 0,841	& 0,853	& \textbf{2,6}	& 0,995	& 0,996 \\
\hline
\textbf{1,1}	& 0,864	& 0,875	& \textbf{2,7}	& 0,997	& 0,997 \\
\hline
\textbf{1,2}	& 0,885	& 0,894	& \textbf{2,8}	& 0,997	& 0,998 \\
\hline
\textbf{1,3}	& 0,903	& 0,911	& \textbf{2,9}	& 0,998	& 0,998 \\
\hline
\textbf{1,4}	& 0,919	& 0,926	& \textbf{3,0}	& 0,999	& 0,999 \\
\hline
\textbf{1,5}	& 0,933	& 0,939	& \textbf{3,1}	& 0,999	& 0,999 \\
\hline
\end{tabular}
\end{enumerate}

\clearpage
\section{Zmienne losowe dwuwymiarowe}
\begin{enumerate}
\item Z talii 52 kart wylosowano jedną kartę. Niech zmienna losowa $X$ przyjmuje wartość odpowiadającą liczbie wylosowanych waletów, zaś $Y$ odpowiadającą liczbie wylosowanych trefli.
\begin{enumerate}
\item Wyznacz rozkłady zmiennych losowych $X$ oraz $Y$. \ans{$P(X=0)=\frac{48}{52} P(X=1)=\frac{4}{52} P(Y=1)=\frac{13}{52} P(Y=0)=\frac{39}{52}$}
\item Wyznacz rozkład zmiennej losowej $(X,Y)$. \ans{$P(0,0)=\frac{36}{52} P(1,0)=\frac{3}{52} P(0,1)=\frac{12}{52} P(1,1)=\frac{1}{52}$}
\item Oblicz dystrybuantę zmiennej losowej $(X,Y)$. \ans{\[F(u,v)=\begin{cases} 
	0 & u\leq 0 \lor v\leq 0 \\ 
	\frac{36}{52} & 0<u\leq 1 \land 0<v\leq 1 \\
	\frac{39}{52} & u>1 \land 0<v\leq 1 \\
	\frac{48}{52} & 0<u\leq 1 \land v>1 \\
	1 & u>1 \land v>1
	\end{cases}\]}
\item Czy zmienne losowej $X$ i $Y$ są niezależne? Odpowiedź uzasadnij odpowiednim rachunkiem. \ans{Tak}
\item Oblicz moment zwykły mieszany rzędu 1+1 zmiennej losowej $(X,Y)$. \ans{$EXY=\frac{1}{52}$}
%\item Wiadomo, że wylosowano trefla. Oblicz prawdopodobieństwo, że jest to walet posługując się warunkowym rozkładem prawdopodobieństwa. \ans{$P(X=1|Y=1)=\frac{\frac{1}{52}}{\frac{13}{52}}=\frac{1}{13}$}
\end{enumerate}
\item Z talii 52 kart wylosowano jedną kartę. Niech zmienna losowa $X$ przyjmuje wartość odpowiadającą liczbie wylosowanych dam trefl, zaś $Y$ odpowiadającą liczbie wylosowanych trefli.
\begin{enumerate}
\item Wyznacz rozkłady zmiennych losowych $X$ oraz $Y$. \ans{$P(X=0)=\frac{51}{52} P(X=1)=\frac{1}{52} P(Y=1)=\frac{13}{52} P(Y=0)=\frac{39}{52}$}
\item Wyznacz rozkład zmiennej losowej $(X,Y)$. \ans{$P(0,0)=\frac{39}{52} P(1,0)=0 P(0,1)=\frac{12}{52} P(1,1)=\frac{1}{52}$}
\item Oblicz dystrybuantę zmiennej losowej $(X,Y)$.
\item Czy zmienne losowej $X$ i $Y$ są niezależne? Odpowiedź uzasadnij odpowiednim rachunkiem. \ans{Nie, $0=P(1,0)\neq P(X=1)P(Y=0)=\frac{39}{52^2}$}
\item Oblicz moment zwykły mieszany rzędu 1+1 zmiennej losowej $(X,Y)$. \ans{$EXY=\frac{1}{52}$}
%\item Wiadomo, że wylosowano trefla. Oblicz prawdopodobieństwo, że nie jest to dama posługując się warunkowym rozkładem prawdopodobieństwa. \ans{$P(X=0|Y=1)=\frac{\frac{12}{52}}{\frac{13}{52}}=\frac{12}{13}$}
\end{enumerate}
\item Doświadczenie polega na dwukrotnym rzucie kostką sześciościenną. Niech zmienna $X$ odpowiada liczbie rzutów, w~których wyrzucono parzystą liczbę oczek, natomiast zmienna $Y$ liczbie rzutów, w których wyrzucono co najmniej 5 oczek.
\begin{enumerate}
\item Podaj brzegowe rozkłady prawdopodobieństwa zmiennych losowych $X$ oraz $Y$.
\item Podaj łączny rozkład prawdopodobieństwa zmiennej losowej dwuwymiarowej $(X,Y)$.
\item Zbadaj, czy zmienne losowe $X$ oraz $Y$ są niezależne.
\item Wiadomo, że dwukrotnie wyrzucono parzystą liczbę oczek. Jakie jest prawdopodobieństwo, że były to dwie szóstki? Odpowiedź przedstaw wykorzystując rozkład prawdopodobieństwa zmiennej losowej $(X,Y)$.
\end{enumerate}
\item Pippin rzucił trzykrotnie sześciościenną, uczciwą kostką do gry. Niech zmienna $X$ odpowiada liczbie rzutów, w~których udało mu się wyrzucić mniej niż pięć
oczek, natomiast zmienna $Y$ liczbie rzutów, w których udało mu się wyrzucić przynajmniej dwa oczka.
\begin{enumerate}
\item Podaj brzegowe rozkłady prawdopodobieństwa zmiennych losowych $X$ oraz $Y$.
\item Podaj łączny rozkład prawdopodobieństwa zmiennej losowej dwuwymiarowej $(X,Y)$.
\item Zbadaj, czy zmienne losowe $X$ oraz $Y$ są zależne.
\item Oblicz moment zwykły mieszany rzędu 2+1 zmiennej losowej $(X,Y)$.
\end{enumerate}
\item Hobbici w skórzanym woreczku mają 7 kartek zapisanych atramentami w dwóch różnych kolorach: czerwonym i zielonym.
Na każdej z kartek jest inna cyfra od 1 do 7, przy czym cyfry 1, 2, 3, 6, 7 są zapisane kolorem czerwonym, a cyfry 4, 5 kolorem zielonym.
Pippin losuje ze zwracaniem z woreczka dwie kartki.
Niech $X$ będzie zmienną losową odpowiadającą liczbie wylosowanych kartek, na których była napisana parzysta liczba, natomiast $Y$ zmienną losową odpowiadającą liczbie kartek zapisanych czerwonym atramentem.

\begin{enumerate}
\item Podaj rozkład brzegowy zmiennej $X$.
\item Podaj rozkład brzegowy zmiennej $Y$.
\item Podaj łączny rozkład prawdopodobieństwa zmiennej losowej $(X,Y)$.
\item Zbadaj, czy zmienne losowej $X$ oraz $Y$ są niezależne.
\item Oblicz moment zwykły mieszany rzędu 1+1 zmiennej losowej $(X,Y)$.
\end{enumerate}

\clearpage
\item Dana jest funkcja $f(x,y)$ określona poniższym wzorem:
\[ f(x,y)=\begin{cases} cxy & 0\leq x \leq 4 \land 0\leq y\leq 2 \\ 0 & \text{wpp}\end{cases} \]
\begin{enumerate}
\item Wyznacz stałą $c$ taką, żeby $f(x,y)$ była funkcją gęstości prawdopodobieństwa pewnej zmiennej losowej $(X,Y)$. \ans{$c=\frac{1}{16}$}
\item Wyznacz gęstości prawdopodobieństwa rozkładów brzegowych. \ans{$f_X(u)=\frac{u}{8} f_Y(v)=\frac{v}{2}$}
\item Wyznacz dystrybuantę zmiennej losowej $(X,Y)$. \ans{\[F(u,v)=\begin{cases}
	0 & u\leq 0 \lor v\leq 0 \\
	\frac{u^2v^2}{64} & 0<u\leq 4 \land 0<v\leq 2 \\
	\frac{u^2}{16} & 0<u\leq 4 \land v>2 \\
	\frac{v^2}{4} & u> 4 \land 0<v\leq 2 \\
	1 & u>4\land v>2
	\end{cases}\]}
\item Wiadomo, że realizacja zmiennej losowej $Y$ zawiera się w przedziale $(1,2)$. Jakie jest prawdopodobieństwo, że wtedy realizacja zmiennej losowej $X$ zawiera się w przedziale $(0,2)$?
\ans{$P(0<X<2|1<Y<2)=\frac{P(0<X<2,1<Y<2)}{P(1<Y<2)}=\frac{F(2,2)-F(2,1)-F(0,2)+F(0,1)}{F_Y(2)-F_Y(1)}=\frac{\frac{1}{4}-\frac{1}{16}-0+0}{1-\frac{1}{4}}=\frac{3}{16}\frac{4}{3}=\frac{1}{4}$}
\end{enumerate}
\end{enumerate}
\clearpage
\section{Regresja liniowa}
\begin{enumerate}
\item Produkcja piwa składa się m.in. z zaszczepienia brzeczki piwnej odpowiednią liczbą komórek drożdżowych. Jednym z typowych źródeł drożdży jest zebranie ich z dna pojemnika, w którym odbywała się poprzednia fermentacja (zebranie tzw. gęstwy drożdżowej). Zdarza się, że trzeba zebrać gęstwę pewien czas przed przygotowaniem następnej partii brzeczki piwnej. Niestety, drożdże w trakcie przechowywania obumierają.

W laboratorium pewnego browaru przeprowadzono 50 eksperymentów, badających
procent przeżywalności drożdży w zależności od czasu ich przechowywania. W
poniższej tabeli w kolumnach podano wartości zmiennej losowej $X$,
odpowiadającej liczbie dni, przez które gęstwa drożdżowa była przechowywana,
natomiast w~wierszach podane są wartości zmiennej losowej $Y$, przedstawiającej
procent żywych komórek drożdżowych w gęstwie. W komórkach podane są liczby
eksperymentów, w których uzyskano daną parę warości $(X,Y)$.

\begin{tabular}{r|rrrr|r}
\diagbox{Y}{X} & 3 & 7 & 14 & 21 & $\sum$ \\
\hline
95 	& 8 & 1 & 0 & 0 & 9\\
85 	& 1 & 8 & 1 & 0 & 10 \\
70 	& 1 & 4 & 8 & 4 & 17 \\
40 	& 2 & 2 & 4 & 6 & 14 \\
\hline
$\sum$ 	& 12 & 15 & 13 & 10 & 50\\
\end{tabular}

\begin{enumerate}
\item Podaj (w formie tabeli dwudzielczej) rozkład prawdopodobieństwa zmiennej losowej dwuwymiarowej $(X,Y)$ wraz z~rozkładami brzegowymi.
\item Oblicz wartości średnie i wariancje zmiennych losowych jednowymiarowych $X$ oraz $Y$. \ans{$EX=10{,}66 EY=69{,}1 DX=6{,}51 DY=20{,}21$}
\item Oblicz współczynnik korelacji zmiennej losowej $(X,Y)$. Uzasadnij za pomocą obliczonego współczynnika, czy można spodziewać się zachodzenia związku liniowego między zmiennymi $X$ oraz $Y$? \ans{$cov(X,Y)=-74{,}21, \varrho=-0{,}56$}
\item Wykorzystaj regresję liniową do wyjaśnienia zmiennej losowej $Y$ w kategoriach zmiennej losowej $X$, tzn. oblicz współczynniki równania $Y=aX+b$. \ans{$a=-1{,}83 b=88{,}61$}
\item Jakiej przeżywalności drożdży należy się spodziewać po 10 dniach przechowywania gęstwy, a jakiej po 180 dniach? Wykorzystaj model obliczony w poprzednim punkcie i przedyskutuj sensowność otrzymanych wyników. \ans{$y(10)=70, y(180)<0$}
\end{enumerate}

\item W Minas Tirith dysponują setką piwnic, w~których przechowują żywność na
wypadek oblężenia. W~każdej z piwnic jest 1000 beczek z solonym mięsem.
Piwnice rozmieszczone są na poziomach od 0 do 3. Niech zmienna losowa $X$
oznacza poziom, na którym znajduje się piwnica, a $Y$ liczbę beczek z zepsutym
jedzeniem wykrytych podczas dorocznej kontroli. W tabeli poniżej znajduje się
rozkład prawdopodobieństwa zmiennej losowej dwuwymiarowej $(X,Y)$.

\begin{tabular}{|r|r|r|r|}
\hline
\diagbox{$X$}{$Y$} & \textbf{0} & \textbf{1} & \textbf{2} \\
\hline
\textbf{0} & 0 & 0,05 & 0,2 \\
\hline
\textbf{1} & 0,05 & 0,15 & 0,05\\
\hline
\textbf{2} & 0,15 & 0,1 & 0 \\
\hline
\textbf{3} & 0,2 & 0,05 & 0\\
\hline
\end{tabular}

\begin{enumerate}
\item Oblicz wartości średnie i wariancje zmiennych losowych jednowymiarowych $X$ oraz $Y$.
\item Oblicz moment zwykły mieszany rzędu $1+1$ zmiennej losowej $(X,Y)$.
\item Oblicz współczynnik korelacji zmiennej losowej $(X,Y)$. Uzasadnij za pomocą obliczonego współczynnika, czy można spodziewać się zachodzenia związku liniowego między zmiennymi $X$ oraz $Y$?
\item Wykorzystaj regresję liniową i oblicz współczynniki równania $Y=aX+b$.
\item Ilu beczek z zepsutym mięsem należałoby się spodziewać w piwnicach Białej Wieży położonej na siódmy poziomie miasta? Wykorzystaj model obliczony w poprzednim punkcie i przedyskutuj sensowność otrzymanych wyników.
\item Oblicz prawdopodobieństwo, że w piwnicy będą dwie beczki z zepsutym jedzeniem, jeżeli wiadomo, że piwnica znajduje się na pierwszym poziomie.
\end{enumerate}

\end{enumerate}



\end{document}
