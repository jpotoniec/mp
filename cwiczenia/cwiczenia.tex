\documentclass[twoside]{mwart}
\usepackage[utf8]{inputenc}
\usepackage{polski}
\usepackage{amsmath,amssymb}
\usepackage[margin=2cm]{geometry}
\usepackage{tikz}
\usepackage{diagbox}
\usepackage{qrcode}
\usepackage{url}
\usetikzlibrary{calc}
\usepackage[linesnumbered]{algorithm2e}
\DontPrintSemicolon
\pagestyle{empty}
\linespread{1.3}
\newcommand{\abs}[1]{\left|#1\right|}
%\let\odp
\ifx\odp\undefined
\usepackage{verbatim}
\newcommand{\ans}[1]{}
\newenvironment{ansenv}{\comment}{\endcomment}
\else
\newenvironment{ansenv}{\paragraph{Odpowiedź:}}{}
\newcommand{\ans}[1]{\begin{ansenv}#1\end{ansenv}}
\fi
\DeclareMathOperator{\cov}{cov}
\DeclareMathOperator{\corr}{\rho}
\colorlet{darkgreen}{green!50!black}
\begin{document}
\section{Przestrzeń probabilistyczna}
%[anchor=south west,inner sep=0pt]
\begin{tikzpicture}[remember picture,overlay]
 \node at ($(current page.north east)-(3cm,3cm)$) {\qrcode[hyperlink,height=3cm]{http://www.cs.put.poznan.pl/jpotoniec/mp/cwiczenia.pdf}};
\end{tikzpicture}

\begin{enumerate}
\item Na ile sposobów można ustawić 5 osób w kolejce? \ans{$5!$}
\item Ile słów pięcioliterowych (nawet tych bezsensownych) można utworzyć z liter \emph{A, B, C}? \ans{$3^5=243$}
\item Z partii towaru zawierającej sztuki dobre i niedobre losujemy 3 sztuki (\emph{próba}). Niech $A$ oznacza zdarzenie: \emph{dokładnie jedna sztuka dobra w próbie}, $B$ zdarzenie: \emph{co najwyżej jedna sztuka dobra w próbie}, $C$ zdarzenie: \emph{co najmniej jedna sztuka dobra w próbie}. Opisz słownie następujące zdarzenia:
\begin{enumerate}
\item $A'$, $B'$, $C'$
\item $A\cup B$
\item $A\cap B$
\item $B\cup C$
\item $B\cap C$
\item $B'\cap C'$
\end{enumerate}
\ans{
	W próbie może być (dokładnie) 0, 1, 2 lub 3 sztuki dobre. Zatem:
	\begin{enumerate}
		\item 
		\begin{itemize}
			\item $A'$ Ze wszystkich możliwości usuwamy odpowiedź 1 sztuka dobra, uzyskując: \emph{0 lub 2 lub 3 sztuki dobre w próbie}. Uwaga, odpowiedź \emph{Dokładnie 1 sztuka zła w próbie} nie jest poprawna, bo to znaczy to samo co \emph{dokładnie 2 sztuki dobre w próbie}, a to jest tylko jedna z pozostałych możliwości.
			\item $B'$  Ze wszystkich możliwości usuwamy odpowiedzi 0 i 1 sztuk dobrych, uzyskując: \emph{2 lub 3 sztuki dobre w próbie}.
			\item $C'$  Ze wszystkich możliwości usuwamy odpowiedzi 1, 2, 3 sztuki dobre, uzyskując: \emph{0 sztuk dobrych w próbie}.
		\end{itemize}
		\item $A\cup B$ zachodzi zdarzenie $A$ i/lub zachodzi zdarzenie $B$, zatem \emph{w próbie jest dokładnie jedna sztuka dobra i/lub w próbie jest co najwyżej jedna sztuka dobra}, co upraszcza się do \emph{w próbie jest co najwyżej jedna sztuka dobra}
		\item $A\cap B$ zachodzi zdarzenie $A$ i jednocześnie zachodzi zdarzenie $B$, zatem \emph{w próbie jest dokładnie jedna sztuka dobra i jednocześnie w próbie jest co najwyżej jedna sztuka dobra}, co upraszcza się do \emph{w próbie jest dokładnie jedna sztuka dobra}
		\item $B\cup C$ \emph{co najwyżej 1 sztuka dobra i/lub co najmniej 1 sztuka dobra}, czyli \emph{dowolna liczba sztuk dobrych} lub \emph{zdarzenie pewne}
		\item $B\cap C$ \emph{co najwyżej 1 sztuka dobra i jednocześnie co najmniej 1 sztuka dobra}, czyli \emph{dokładnie 1 sztuka dobra}
		\item $B'\cap C'$ \emph{2 lub 3 sztuki dobre w próbie i jednocześnie 0 sztuk dobrych w próbie}, czyli \emph{zdarzenie niemożliwe}.
	\end{enumerate}
}
\item Inżynier projektuje magazyn do przechowywania kartonów puszek żywności. Kartony mają kształt sześcianów o krawędzi 4 dm i masie 50 kg każdy. Zakłada się, że kartony nie mogą być ułożone w wieżę wyższą niż 24 dm. Zaproponować przestrzeń zdarzeń losowych dla następujących doświadczeń:
\begin{enumerate}
\item Obserwacja całkowitego obciążenia 16 $dm^2$ powierzchni pochodzącego z jednego stosu kartonów.
\item Obserwacja całkowitego obciążenia 16 $dm^2$ powierzchni pochodzącego z dwóch stosów kartonów, przy założeniu, że jest to obciążenie wywołane przezpolowę masy każdego z dwóch stosów.
\end{enumerate}
W obu przestrzeniach opisać następujące zdarzenia:
\begin{description}
\item[A] całkowite obciążenie wynosi co najmniej 150 kg;
\item[B] całkowite obciążenie wynosi nie więcej niż 200 kg;
\item[C] całkowite obciążenie przekracza 250 kg;
\end{description}
\ans{
	Istnieje wiele możliwych rozwiązań, poniżej zaprezentowane są przykłady. Kryterium jest takie: w danej przestrzeni zdarzeń elementarnych musi dać się opisać jako jej podzbiory zdarzenia $A$, $B$, $C$ i dla danej obserwacji stosu/stosów kartonów musi dać się przypisać dokładnie jedno zdarzenie elementarne (ale może być tak że wielu obserwacjom przypisujemy to samo zdarzenie elementarne).
	\begin{enumerate}
		\item Obserwacja całkowitego obciążenia 16 $dm^2$ powierzchni pochodzącego z jednego stosu kartonów.
		\begin{enumerate}
			\item Zdarzeniom elementarnym odpowiadają masy stosu i nie rozpatrujemy pustych stosów
			\begin{gather*}
				\Omega=\{50, 100, 150, 200, 250, 300\} \\
				A = \{150, 200, 250, 300\} \\
				B = \{50, 100, 150, 200\} \\
				C = \{300\}
			\end{gather*}
			\item J.w., ale rozpatrujemy puste stosy
			\begin{gather*}
				\Omega=\{0, 50, 100, 150, 200, 250, 300\} \\
				A = \{150, 200, 250, 300\} \\
				B = \{0, 50, 100, 150, 200\} \\
				C = \{300\}
			\end{gather*}
			\item Kodujemy zdarzenia elementarne jako wysokość stosu zamiast masy
			\begin{gather*}
			\Omega=\{0, 1, 2, 3, 4, 5, 6\} \\
			A = \{3, 4, 5, 6\} \\
			B = \{0, 1, 2, 3, 4\} \\
			C = \{6\}
			\end{gather*}
		\end{enumerate}
		\item Obserwacja całkowitego obciążenia 16 $dm^2$ powierzchni pochodzącego z dwóch stosów kartonów, przy założeniu, że jest to obciążenie wywołane przez połowę masy każdego z dwóch stosów. Wszystkie uwagi o kodowaniu i zerach z poprzedniego podpunktu nadal obowiązują.
		\begin{enumerate}
			\item Zdarzeniom elementarnym oznaczone są obserwowaną masą, dopuszczamy puste stosy. Należy zwrócić uwagę, że dopuszczalne jest stosowanie $\ldots$ jeżeli jest jasne co skraca. Musi się wtedy obowiązkowo pojawić ostatni element, bo inaczej zbiór wygląda na nieskończony.
			\begin{gather*}
			\Omega=\{0, 25, 50, 75, \ldots, 300\} \\
			A = \{150, 175, \ldots, 300\} \\
			B = \{0, 25, \ldots, 200\} \\
			C = \{275, 300\}
			\end{gather*}
			\item Możemy też kodować zdarzenia elementarne jako uporządkowane dwójki, w każdym przypadku kodując oddzielnie wysokość pierwszego i drugiego stosu. Dla ułatwienia zapisu można posłużyć się tzw. \emph{set-builder notation}.
			\begin{gather*}
			\Omega = \{ (i,j) \colon i,j\in\{0,1,2,3,4,5,6\}\}=\{(0,0), (0, 1), \ldots (0,6), (1,0), (1,1), \ldots, (6,6) \} \\
			A=\{(i,j)\in\Omega \colon \frac{i+j}{2}\cdot 50 \geq 150 \}	\\		
			B=\{(i,j)\in\Omega \colon \frac{i+j}{2}\cdot 50 \leq 200 \}\\
			C=\{(i,j)\in\Omega \colon \frac{i+j}{2}\cdot 50 > 250 \}
			\end{gather*}
			\item Zamiast dwójek uporządkowanych (oznaczanych nawiasami okrągłymi) możemy stosować dwójki nieuporządkowane (dwuelementowe multizbiory)
			\begin{gather*}
			\Omega = \{ \{i,j\} \colon i,j\in\{0,1,2,3,4,5,6\}\}=\{\{0,0\}, \{0, 1\}, \ldots \{0,6\}, \{1,1\}, \{1,2\}, \ldots, \{1,6\}, \ldots, \{5,6\}, \{6,6\} \} \\
			A=\{\{i,j\}\in\Omega \colon \frac{i+j}{2}\cdot 50 \geq 150 \}	\\		
			B=\{\{i,j\}\in\Omega \colon \frac{i+j}{2}\cdot 50 \leq 200 \} \\
			C=\{\{i,j\}\in\Omega \colon \frac{i+j}{2}\cdot 50 > 250 \}
			\end{gather*}
		\end{enumerate}
	\end{enumerate}
}
\item Na 10 kartkach napisano liczby od 1 do 10 i wrzucono do pudełka. Losujemy w sposób przypadkowy dwie kartki. Niech $A$ odpowiada zdarzeniu \emph{wylosowanie kartki z numerem 1}, a $B$ zdarzeniu \emph{wylosowanie pary liczb, których suma jest większa od 4}. Zakładamy, że wylosowanie każdej z kartek jest równoprawdopodobne.
\begin{enumerate}
\item Przedstaw przestrzeń zdarzeń elementarnych. Podaj jej rozmiar. \ans{$\Omega=\{\{i,j\}|i,j\in\{1,\ldots,10\} \land i\neq j\}, \left|\Omega\right|={10 \choose 2}=45$}
\item Zdefiniuj zdarzenia $A$ i $B$ jako zbiory zdarzeń elementarnych. \ans{$A=\{\{i,j\}\in\Omega|i=1 \lor j=1\}$, $B=\{\{i,j\}\in\Omega|i+j>4\}$}
\item Ile jest zdarzeń sprzyjających zdarzeniom $A$ i $B$. \ans{$\left|A\right|=9$, $\left|B\right|=45-4=41$}
\item Oblicz $P(A)$ i $P(B)$ \ans{$P(A)=\frac{9}{45}, P(B)=\frac{41}{45}$}

\ans{Alternatywne rozwiązanie korzysta z par uporządkowanych, wtedy wszystkie zbiory są dwa razy bardziej liczne, ale prawdopodobieństwa wychodzą takie same. Można też rozważyć przypadek losowania ze zwracaniem, wtedy $\Omega$ musi składać się z par \emph{uporządkowanych}, $\left|\Omega\right|=100$ no i liczności $A$ oraz $B$, a w konsekwencji prawdopodobieństwa wychodzą trochę inne.}
\end{enumerate}
\item W gospodzie \emph{Złoty okoń} zorganizowano loterię, w której do sprzedania było 100 biletów, a wygrywał tylko jeden. Każdy mógł kupić najwyżej jeden bilet. Niech zdarzenie $A$ odpowiada sytuacji, w której Estella posiada los wygrywający. Jakie ma szanse na wygraną?
\begin{enumerate}%
\item Przedstaw przestrzeń zdarzeń elementarnych. \ans{$\Omega=\{\omega_n|n=1,2,\ldots,100\}$}%
\item Czy w tej przestrzeni wszystkie zdarzenia elementarne są jednakowo prawdopodobne? \ans{Tak (przy czym jest to raczej kwestia założenia)}%
\item Jaki jest rozmiar przestrzeni zdarzeń elementarnych? \ans{$\left|\Omega\right|=100$}%
\item Zdefiniuj zdarzenie $A$ jako zbiór zdarzeń elementarnych. \ans{$A=\{\omega_1$\} (Wybieramy dowolny los jako wygrywający, powiedzmy ten, któremu odpowiada zdarzenie elementarne $\omega_1$)}%
\item Oblicz prawdpodobieństwo $P(A)$, dbając o to by jasno przedstawić tok rozumowania. \ans{
	Na podstawie odpowiedzi w punkcie b możemy posłużyć się prawdopodobieństwem klasycznym:
	$P(A)=\frac{\left|A\right|}{\left|\Omega\right|}=\frac{1}{100}$
}%
\end{enumerate}%
\item W gospodzie \emph{Pod Zielonym Smokiem} oferują sześć różnych dań obiadowych. Pięciu klientów wchodzi jeden po drugim do gospody
i~niezależnie od siebie zamawia posiłek. Niech zdarzenie $A$ odpowiada sytuacji, w~której pierwsze danie z menu zamówi dokładnie jedna osoba.%
\begin{enumerate}%
\item Przedstaw przestrzeń zdarzeń elementarnych. \ans{$\Omega=\{(i_1,\ldots,i_5)|i_j=1,2,\ldots,6\}$
	 Uporządkowane ciągi 5-ciu elementów, w których kolejne pozycje odpowiadają numerom dań zamówionych przez kolejnych klientów.
	 Musimy rozpatrywać wszystkich klientów jednocześnie, ponieważ nasze doświadczenie polega na obserwacji wszystkich klientów na raz (\emph{dokładnie jedna osoba}), a jednemu wynikowi obserwacji musi odpowiadac \emph{dokładnie jedno} zdarzenie elementarne.
}%
\item Czy w tej przestrzeni wszystkie zdarzenia elementarne są jednakowo prawdopodobne? \ans{Tak (zakładając, że klienci zamawiają dania niezależnie od siebie i }%
\item Jaki jest rozmiar przestrzeni zdarzeń elementarnych? \ans{$\left|\Omega\right|=6^5$}%
\item Zdefiniuj zdarzenie $A$ jako zbiór zdarzeń elementarnych. \ans{$A=\{(i_1,\ldots,i_5)|\exists j: i_j=1 \land \forall k\neq j: i_k\neq 1\}$, $\left|A\right|=5\cdot5^4$}%
\item Oblicz prawdpodobieństwo $P(A)$, dbając o to by jasno przedstawić tok rozumowania. \ans{$P(A)=\frac{5^5}{6^5}=\frac{5}{6}^5\approx 0{,}40$}%
\end{enumerate}%
\item Oblicz, ile jest liczb naturalnych sześciocyfrowych, w zapisie których występuje dokładnie trzy razy cyfra 0 i dokładnie raz występuje cyfra 5. %\ans{${8\choose 2}3\frac{5!}{3!}+8\cdot\left(\frac{5!}{3!2!}+\frac{5!}{3!}\right)=1920$}
\item  Drewniane pale mają losową długość $L$ nie przekraczającą 12 m. Pale są przeznaczone do wbijania w ziemię, której skalna warstwa stanowiąca opór znajduje się na losowej głębokości $H$, nie większej niż 10 m. Zaproponować przestrzeń zdarzeń elementarnych dla tak opisanego doświadczenia. Zdefiniować przez odpowiednie zdarzenia elementarne następujące doświadczenia:
\begin{description}
\item[A] długość losowo wybranego pala będzie większa od głębokości skalnej warstwy;
\item[B] głębkość skalnej warstwy przekroczy 8 metrów;
\item[C] długość losowo wybranego pala przekroczy 8 metrów;
\item[D] $B\cap C$
\item[E] $B\cup C$
\item[F] $(B\cup C)\cap A'$
\end{description}
\item Partia towaru składa się ze 100 elementów, wśród których 5 jest wadliwych. Poddajemy kontroli 50 elementów. Partię przyjmujemy, jeśli wśród kontrolowanych elementów jest nie więcej niż jeden wadliwy. Niech zdarzenie $A$ odpowiada przyjęciu partii.
\begin{enumerate}%
\item Przedstaw przestrzeń zdarzeń elementarnych. \ans{$\Omega=\{\omega_J|\left|J\right|=50 \land J\subset\{1,2,\ldots,100\}\}$}%
\item Czy w tej przestrzeni wszystkie zdarzenia elementarne są jednakowo prawdopodobne? \ans{Tak}%
\item Jaki jest rozmiar przestrzeni zdarzeń elementarnych? \ans{$\left|\Omega\right|={100\choose 50}$}%
\item Zdefiniuj zdarzenie $A$ jako zbiór zdarzeń elementarnych. \ans{$A=\{\omega_J|\left|J\cap \{1,2,\ldots,5\}\right|\leq 1\}$}%
\item Oblicz prawdpodobieństwo $P(A)$, dbając o to by jasno przedstawić tok rozumowania. \ans{$\left|A\right|={5\choose 1}{95\choose 49}+{95\choose 50}, P(A)\approx0{,}181$}%
\end{enumerate}%
\item Winda rusza z siedmioma pasażerami i zatrzymuje się na dziesięciu piętrach. Niech zdarzenie $A$ odpowiada sytuacji, w której żadnych dwóch pasażerów nie opuści windy na tym samym piętrze.
\begin{enumerate}%
\item Przedstaw przestrzeń zdarzeń elementarnych. \ans{$\Omega=\{\omega_{i_1,\ldots,i_7}|i_j=1,2,\ldots,10\}$}%
\item Czy w tej przestrzeni wszystkie zdarzenia elementarne są jednakowo prawdopodobne? \ans{Tak}%
\item Jaki jest rozmiar przestrzeni zdarzeń elementarnych? \ans{$\left|\Omega\right|=10^7$}%
\item Zdefiniuj zdarzenie $A$ jako zbiór zdarzeń elementarnych. \ans{$A=\{\omega_{i_1,\ldots,i_7}|\forall j,k\colon j\neq k\to i_j\neq i_k\}, \left|A\right|=\frac{10!}{3!}=604800$}%
\item Oblicz prawdpodobieństwo $P(A)$, dbając o to by jasno przedstawić tok rozumowania. \ans{$P(A)=\frac{604800}{10^7}=0{,}06$}%
\end{enumerate}%
\item Dwudziestoosobowa grupa studencka, w której jest 6 kobiet, otrzymała 5 biletów do teatru. Bilety rozdziela się drogą losowania. Niech zdarzenie $A$ odpowiada sytuacji, w której wśród posiadaczy biletów znajdą się dokładnie trzy kobiety.
\begin{enumerate}%
\item Przedstaw przestrzeń zdarzeń elementarnych. \ans{$\Omega=\{\omega_J|\left|J\right|=5 \land J\subset\{1,2,\ldots,20\}\}$}%
\item Czy w tej przestrzeni wszystkie zdarzenia elementarne są jednakowo prawdopodobne? \ans{Tak}%
\item Jaki jest rozmiar przestrzeni zdarzeń elementarnych? \ans{$\left|\Omega\right|={20\choose 5}=15504$}%
\item Zdefiniuj zdarzenie $A$ jako zbiór zdarzeń elementarnych. \ans{$A=\{\omega_J|\left|J\cap \{1,2,\ldots,6\}\right|=3\}, \left|A\right|={6\choose 3}{14\choose 2}=1820$}%
\item Oblicz prawdpodobieństwo $P(A)$, dbając o to by jasno przedstawić tok rozumowania. \ans{$P(A)=0{,}12$}%
\end{enumerate}%
\item Dane są dwa pojemniki. W pierwszym z nich znajduje się 11 kul: 7 białych i 4 czarne. W~drugim pojemniku jest 6 kul: 3 białe i 3 czarne. Z każdego pojemnika losujemy po dwie kule.
Rozpatrz poniższe pytania w dwóch wariantach: wszystkie kule w danym kolorze są identyczne oraz gdy kule są rozróżnialne (np. numerowane).
\begin{enumerate}
\item Ile jest możliwych układów? \ans{${11 \choose 2}{6 \choose 2}=825$}
\item Ile jest układów składających się wyłącznie z kul czarnych? \ans{${4 \choose 2}{3 \choose 2}=18$}
\item Ile jest układów składających się wyłącznie z kul białych? \ans{${7 \choose 2}{3 \choose 2}=63$}
\item Ile jest układów składających się z pary białej i pary czarnej? \ans{${7\choose 2}{3 \choose 2}+{4 \choose 2}{3 \choose 2}+7\cdot4\cdot3\cdot3=333$}
\item Ile jest układów składających się z jednej kuli białej i trzech czarnych? \ans{$7\cdot4\cdot{3\choose 2}+3\cdot3\cdot{4\choose 2}=138$}
\item Ile jest układów składających się z jednej kuli czarnej i trzech białych?  \ans{$4\cdot7\cdot{3\choose 2}+3\cdot3\cdot{7\choose 2}=273$}
\end{enumerate}
\end{enumerate}%
\clearpage
\section{Prawdopodobieństwo warunkowe i niezależność zdarzeń}
\begin{enumerate}
\item W sklepie są sprzedawane baterie dwóch firm A i B. Firma A dostarcza do sklepu dwa razy więcej baterii niż firma B. Braki wśród baterii (inaczej: niesprawne baterie) tych firm stanowią odpowiednio $0{,}9\%$ i $1{,}4\%$. Kupujemy jedną baterię. Jakie jest prawdopodobieństwo kupienia baterii dobrej?
\ans{
	Definiujemy zdarzenia:
	\begin{itemize}
		\item $A$ bateria pochodzi z firmy A
		\item $B$ bateria pochodzi z firmy B
		\item $D$ bateria jest dobra
		\item $Z$ bateria jest niesprawna
	\end{itemize}
	Z treści zadania budujemy układ równań:
	\[\begin{cases}
	P(A)=2P(B) \\
	P(A)+P(B)=1
	\end{cases} \]
	I obliczamy prawdopodobieństwa: \[P(A)=\frac{2}{3}\qquad P(B)=\frac{1}{3}\]
	Odczytujemy prawdopodobieństwa warunkowe:
	\[ P(Z|A)=0{,}9\% \qquad P(Z|B)=1{,}4\% \]
	Zauważamy, że $A$ i $B$ stanowią podział przestrzeni: oba są możliwe, sumują się do całej przestrzeni i są rozłączne. Zatem możemy zastosować twierdzenie o prawdopodobieństwie całkowitym:
	\[ P(Z)=P(Z|A)P(A)+P(Z|B)P(B)=0{,}9\%\cdot\frac{2}{3}+1{,}4\%\cdot\frac{1}{3}=\frac{3,2}{3}\%=\frac{32}{3000} \]
	Zauważamy, że $D=Z'$, a zatem:
	\[P(D)=1-P(Z)=\frac{2968}{3000}\approx 98,93\% \]
}
\item O pewnym roczniku studentów wiadomo, że dzielą się na grupy:
\begin{description}
\item[$G_1$] $5\%$ potrafiące odpowiedzieć na wszystkie pytania;
\item[$G_2$] $30\%$ potrafiące odpowiedzieć na 70\% pytań;
\item[$G_3$] $40\%$ potrafiące odpowiedzieć na 60\% pytań;
\item[$G_4$] $25\%$ potrafiące odpowiedzieć na 50\% pytań;
\end{description}
Wybrano w sposób przypadkowy jednego studenta. Obliczyć:
\begin{enumerate}
\item prawdopodobieństwo, że odpowie on na pytanie;
\item prawdopodobieństwo, że należy do grupy drugiej, jeżeli wiadomo, że odpowiedział na pytanie.
\end{enumerate}
\ans{
	Definiujemy zdarzenia:
	\begin{itemize}
		\item $O$ student umie odpowiedzieć na pytanie
		\item $G_1, G_2, G_3, G_4$ student należy do danej grupy
	\end{itemize}
	Odczytujemy prawdopodobieństwa z treści:
	\begin{gather*}
		P(G_1)=\frac{5}{100} \qquad P(O|G_1)=1 \\
		P(G_2)=\frac{30}{100} \qquad P(O|G_2)=\frac{7}{10} \\
		P(G_3)=\frac{40}{100} \qquad P(O|G_3)=\frac{6}{10} \\
		P(G_4)=\frac{25}{100} \qquad P(O|G_4)=\frac{5}{10} \\
	\end{gather*}
	Student należy do dokładnie jednej grupy, a zatem zdarzenia $G_1, \ldots, G_4$ stanowią podział przestrzeni i możemy zastosować twierdzenie o prawdopodobieństwie całkowitym, żeby obliczyć odpowiedź na pytanie z punktu a:
	\[ P(O) = P(O|G_1)P(G_1)+P(O|G_2)P(G_2)+P(O|G_3)P(G_3)+P(O|G_4)P(G_4)=\frac{625}{1000} \]
	
	Następnie korzystamy z twierdzenia Bayesa, żeby obliczyć prawdopodobieństwo z punktu b:
	\[ P(G_2|O) = \frac{P(O|G_2)P(G_2)}{P(O)} = \frac{\frac{7}{10}\cdot\frac{30}{100}}{\frac{625}{1000}} = \frac{210}{625}= 0{,}336 \]
}
\item Prawdopodobieństwo przekazania sygnału przez jeden przekaźnik jest równe $0{,}9$. Przekaźniki działają niezależnie, tzn. awaria jednego z nich nie ma wpływu na działanie pozostałych.
Obliczyć prawdopodobieństwo przekazania sygnału:
\begin{enumerate}
\item przy połączeniu szeregowym dwóch przekaźników (inaczej: muszą działać oba przekaźniki);
\item przy połączeniu równoległym dwóch przekaźników (inaczej: wystarczy, że chociaż jeden przekaźnik będzie działał);
\item przy połączeniu szeregowym trzech przekaźników;
\item przy połączeniu równoległym trzech przekaźników;
\end{enumerate}
\ans{
	Oznaczamy zdarzenia:
	\begin{itemize}
		\item $A$ działa 1. przekaźnik
		\item $B$ działa 2. przekaźnik
		\item $C$ działa 3. przekaźnik
	\end{itemize}
	Zdarzenia są niezależne i każde zachodzi z prawdopodobieństwem $0{,}9$.
	Zatem:
	\begin{enumerate}
		\item \[ P(A\cap B)\underset{\text{z niezależności}}{=}P(A)\cdot P(B)=0{,}9\cdot0{,}9=0{,}81 \]
		\item Interesuje nas $P(A\cup B)$. Możemy skorzystać z prawa de Morgana:
		\[P(A\cup B)=P([A'\cap B']')=1-P(A'\cap B') \]
		A następnie z twierdzenia o niezależności zdarzeń przeciwnych:
		\[P(A'\cap B') = P(A')P(B')=(1-0{,}9)^2=0{,}01\]
		Wracając do pierwszego równania otrzymujemy
		\[P(A\cup B)=1-0{,}01=0{,}99\]
		\item \[ P(A\cap B\cap C)\underset{\text{z niezależności}}{=}P(A)\cdot P(B)\cap P(C)=0{,}9^3=0{,}729 \]
		\item Analogicznie jak punkt b:
		\[ P(A\cup B\cup C)=1-P(A'\cap B'\cap C')=1-(1-0{,}1)^3=0{,}999 \]
	\end{enumerate}
}
\item Wiadomo, że 90\% produkcji spełnia wymagania techniczne. Przeprowadzono dodatkową kontrolę, przy której mogły być popełnione pewne błędy, a mianowicie: element wadliwy mógł zostać sklasyfikowany jako dobry z prawdopodobieństwem $0{,}05$, a element dobry mógł zostać sklasyfikowany jako wadliwy z prawdopodobieństwem $0{,}02$. Obliczyć prawdopodobieństwo tego, że element, który został sklasyfikowany jako dobry, faktycznie jest dobry.
\ans{
	Oznaczamy zdarzenia:
	\begin{itemize}
		\item $K$ element sklasyfikowany jako dobry
		\item $K'$ element sklasyfikowany jako zły
		\item $D$ element faktycznie dobry
		\item $D'$ element faktycznie zły
	\end{itemize}
	Odczytujemy prawdopodobieństwa z treści zadania i obliczamy prawdopodobieństwa zdarzeń przeciwnych:
	\begin{gather*}
		P(D)=0{,}9 \qquad P(D')=0{,}1 \\
		P(K|D')=0{,}05 \qquad P(K'|D')=0{,}95 \\
		P(K'|D)=0{,}02 \qquad P(K|D)=0{,}98
	\end{gather*}
	Mamy policzyć $P(D|K)$, a skoro znamy prawdopodobieństwa "w drugą stronę" (tj. np. $P(K|D)$), to korzystamy z tw. Bayesa:
	\begin{gather*}
		P(D|K)=  \frac{P(K|D)P(D)}{P(K)} = \frac{P(K|D)P(D)}{P(K|D)P(D)+P(K|D')P(D')} = 
		\frac{0{,}98\cdot 0{,}9}{0{,}98\cdot 0{,}9 + 0{,}05\cdot 0{,}1}=\frac{882}{887}
	\end{gather*}
} 

\item Merry wybrał się do Starego Lasu zbierać pewne rośliny o wielce pożądanych właściwościach na nadchodzącą Sobótkę.
Niestety, Merry nie jest zbyt biegły w rozpoznawaniu roślin.
Jedyne co wie, to że ma zbierać lejkowate kwiaty w jednym z trzech kolorów: białe, różowe lub niebieskie.
Czego Merry'emu nie powiedziano to, że wśród białych kwiatów tylko $30\%$ posiada pożądane właściwości, wśród różowych $55\%$, a wśród niebieskich
aż $90\%$.
Wracając do domu Merry zauważył, że wśród kwiatów, które zebrał liczba białych do liczby różowych ma się jak $3:2$, a~liczba różowych do
liczby niebieskich jak $2:1$. Merry dla zabawy zamknął oczy i na chybił-trafił wybrał jeden z~zebranych kwiatów.
 
\begin{enumerate}
	\item Jakie jest prawdopodobieństwo, że wybrany kwiat jest niebieski?
	\item Jakie jest prawdopodobieństwo, że wybrany kwiat jest biały lub różowy?
	\item Jakie jest prawdopodobieństwo, że wybrany kwiat ma pożądane właściwości, jeżeli wiadomo, że jest niebieski?
	\item Jakie jest prawdopodobieństwo, że wybrany kwiat jednocześnie ma pożądane właściwości i jest niebieski?
	\item Jakie jest prawdopodobieństwo, że wybrany kwiat ma pożądane właściwości, jeżeli nie wiadomo jaki ma kolor?
	\item Co by było gdyby Merry nie wiedział, jakiego koloru wylosował kwiat, ale wiedział, że wybrany kwiat nie ma pożądanych
		właściwości: jakie byłoby wtedy prawdopodobieństwo, że kwiat będzie biały lub niebieski?
\end{enumerate}

\ans{
	Zaczynamy od nazwania zdarzeń:
	\begin{itemize}
		\item $N$ niebieski kwiat
		\item $B$ biały kwiat
		\item $R$ różowy kwiat
		\item $W$ kwiat ma właściwości
	\end{itemize}

	Odczytujemy prawdopodobieństwa z treści:
	\[ P(W|B)=\frac{3}{10} \qquad P(W|R)=\frac{55}{100}=\frac{11}{20} \qquad P(W|N)=\frac{9}{10} \]
	
	Z proporcji budujemy układ równań:
	\[ \begin{cases}
		\frac{P(B)}{P(R)}=\frac{3}{2} \\
		\frac{P(R)}{P(N)}=2 \\
		P(B)+P(N)+P(R)=1
	\end{cases} \]
	
	Rozwiązujemy układ równań otrzymując:
	\[ P(B)=\frac{3}{6} \qquad P(R)=\frac{2}{6} \qquad P(N)=\frac{1}{6} \]
	
	\begin{enumerate}
		\item $P(N)=\frac{1}{6}$
		\item $P(B\cup R)=P(B)+P(R)=\frac{5}{6}$ (zdarzenia $B$ i $R$ są rozłączne)
		\item odczytujemy z treści zadania: $P(W|N)=\frac{9}{10}$
		\item korzystamy z definicji prawdopodobieństwa warunkowego: \[P(W\cap N)=P(W|N)P(N)=\frac{9}{10}\cdot\frac{1}{6}=\frac{9}{60}\]
		\item Zdarzenia $B, R, N$ stanowią podział przestrzeni, zatem korzystam z prawdopodobieństwa całkowitego:
		\[ P(W)=P(W|N)P(N)+P(W|R)P(R)+P(W|B)P(B)= 
		\frac{9}{10}\cdot\frac{1}{6} + \frac{11}{20}\cdot\frac{1}{3} + \frac{3}{10}\cdot\frac{3}{6} = \frac{9+11+9}{60}=\frac{29}{60} \]
		\item Korzystamy z faktu, że $R$ jest zdarzeniem przeciwnym dla $B\cup N$ i przepisujemy z twierdzenia Bayesa:
		\[ P(B\cup N|W')=1-P(R|W') = 1-\frac{P(W'|R)P(R)}{P(W')} \]
		Korzystając ze zdarzeń przeciwnych:
		\[1-\frac{P(W'|R)P(R)}{P(W')} = 1-\frac{[1-P(W|R)]P(R)}{1-P(W)} \]
		Podstawiamy wartości liczbowe:
		\[ 1-\frac{[1-P(W|R)]P(R)}{1-P(W)} = 1-\frac{\left(1-\frac{11}{20}\right)\cdot \frac{1}{3}}{1-\frac{29}{60}} = 1-\frac{\frac{9}{60}}{\frac{31}{60}}=\frac{22}{31}
		\]
	\end{enumerate}
}

\clearpage
\item Kuce w Oatbarton od czasu do czasu zapadają na Straszną Chorobę Kuców.
Na szczęście zmyślni, hobbicy farmerzy wymyślili sposób na wykrywanie choroby we wczesnym stadium i izolowanie chorych zwierząt.
Niestety, sposób nie jest idealny: prawdopodobieństwo, że zwierzę zostatnie uznane za chore, wynosi odpowiednio $0,85$ dla zwierzęcia chorego i $0,07$ dla zwierzęcia zdrowego.
Z danych historycznych wiadomo, że prawdopodobieństwo zapadnięcia na Straszną Chorobę Kuców wynosi $0,2$ dla każdego kuca.
Rozpatrujemy doświadczenie polegające na badaniu kuca Ostrouchego.

\begin{enumerate}
\item Zdefiniuj przestrzeń zdarzeń elementarnych. 
\item Jaki jest rozmiar przestrzeni zdarzeń elementarnych? 
\item Czy w tej przestrzeni wszystkie zdarzenia elementarne są równoprawdopodobne? Odpowiedź uzasadnij. 
\item Nazwij i zdefiniuj jako zbiory zdarzeń elementarnych następujące zdarzenia opisane słownie: 
\begin{itemize}
\item Ostrouchy jest chory.
\item Ostrouchy jest zdrowy.
\item Test wskazał, że Ostrouchy jest chory.
\item Test wskazał, że Ostrouchy jest zdrowy.
\end{itemize}
Wykorzystaj tak przypisane nazwy w dalszych obliczeniach!
\item Bez przeprowadzania testu, jakie jest prawdopodobieństwo, że Ostrouchy jest zdrowy? 
\item Jakie jest prawdopodobieństwo, że jednocześnie Ostrouchy jest zdrowy i test wskazał, że Ostrouchy jest zdrowy? 
\item Jakie jest prawdopodobieństwo, że test wskaże, że Ostrouchy jest zdrowy? 
\item Jakie jest prawdopodobieństwo, że Ostrouchy jest zdrowy, jeżeli test wskazał, że Ostrouchy jest zdrowy? 
\item Czy zdarzenia: \emph{Ostrouchy jest zdrowy} oraz \emph{test wskazał, że Ostrouchy jest zdrowy} są niezależne? Odpowiedź uzasadnij odpowiednim rachunkiem. 
\end{enumerate}
\item Telegraficzne przekazywanie informacji odbywa się metodą nadawania
sygnałów kropka-kreska. Statystyczne właściwości zakłóceń są takie, że błędy
następują przeciętnie w 2 przypadkach na 5 przy nadawaniu sygnału kropka i w 1
przypadku na 3 przy nadawaniu sygnału kreska. Wiadomo, że ogólny stosunek
liczby nadawanych sygnałów kropka do sygnałów kreska jest $\frac{5}{3}$.
\begin{enumerate}
\item Odebrano kropkę. Jakie jest prawdopodobieństwo, że nadano kropkę? \ans{$P(N_\cdot|O_\cdot)=\frac{P(O_\cdot|N_\cdot)P(N_\cdot)}{P(O_\cdot|N_\cdot)P(N_\cdot)+P(O_\cdot|N_-)P(N_-)}=\frac{\frac{3}{5}\frac{5}{8}}{\frac{3}{5}\frac{5}{8}+\frac{1}{3}\frac{3}{8}}=\frac{3}{4}$}
\item Odebrano kreskę. Jakie jest prawdopodobieństwo, że nadano kreskę?
\end{enumerate}

\item Brzeczka piwna za pomocą systemu pomp (na rysunku poniżej: czarne koła)
płynie z kadzi warzelnej (punkt $X$) do pojemnika fermentacyjnego (punkt $Y$)
	zgodnie ze schematem pokazanym na poniższym rysunku. Niestety,
	prawodpodobieństwo awarii każdej z pomp w trakcie przepompowywania brzeczki
	wynosi $0{,}1$ i jest stałe, i~niezależne od sprawności pozostałych pomp.
	Niech zdarzenie $A$ odpowiada przepompowaniu brzeczki z~kadzi do pojemnika,
	tzn. istnieniu jakiejkolwiek ścieżki pomiędzy punktami $X$ i $Y$ bez
	uszkodzonej pompy.
	\begin{tikzpicture}
	\draw (0,0) node[anchor=east] {$X$} -- (.5,0) |- (1,.5) -| (1.5,0) -- (2,0) |- (2.5,.5) -| (3,0) -- (3.5,0) node[anchor=west] {$Y$};
\draw (.5,0) |- (1,-.5) -| (1.5,0) -- (2,0) |- (2.5,-.5) -| (3,0);
\filldraw[black] (1,.5) circle (0.1);
\filldraw[black] (1,-.5) circle (0.1);
\filldraw[black] (2.5,.5) circle (0.1);
\filldraw[black] (2.5,-.5) circle (0.1);
\end{tikzpicture}
\begin{enumerate}
\item Przedstaw przestrzeń zdarzeń elementarnych. \ans{$\Omega=\{\omega_{i_1,\ldots,i_4}|i_j\in\{0,1\}\}$}
\item Czy w tej przestrzeni wszystkie zdarzenia elementarne są jednakowo prawdopodobne? \ans{Nie}
\item Jaki jest rozmiar przestrzeni zdarzeń elementarnych? \ans{$\abs{\Omega}=2^4$}
\item Zdefiniuj zdarzenie $A$ jako zbiór zdarzeń elementarnych. \ans{$A=\{
	\omega_{1111},\omega_{1110},\omega_{1101},
		\omega_{1011},\omega_{1010},\omega_{1001},
		\omega_{0111},\omega_{0110},\omega_{0101}
	\}$}
	\item Oblicz prawdpodobieństwo $P(A)$, dbając o to by jasno przedstawić tok rozumowania. \ans{$P(A)=0{,}9^4+4\cdot0{,}9^3\cdot0{,}1+4\cdot0{,}9^2\cdot0{,}1^2=0{,}9801=(0{,}9+0{,}9-0{,}9^2)^2$, np. $P(\omega_{1111})=P(A_1)P(A_2)P(A_3)P(A_4)$}
\end{enumerate}
\end{enumerate}

\clearpage
\section{Zmienne losowe jednowymiarowe typu skokowego}
\begin{enumerate}
\item Rozważamy doświadczenie polegające na jednokrotnym rzucie uczciwą kostką sześciościenną
\begin{enumerate}
\item Zaproponuj zmienną losową $X$ odpowiednią do tego doświadczenia. \ans{
	Oznaczmy zdarzenia elementarne za pomocą liczb rzymskich reprezentujących liczbę oczek wyrzuconych na kostce:
	\[\Omega = \{ I, II, III, IV, V, VI\} \]
	Zmienne losowe możemy wymyślić tu różne (zadanie jest mało precyzyjne), ale ograniczmy się do najprostszej, odwzorowujemy wynik w liczbę oczek:
	\[X=\{I\mapsto 1, II\mapsto 2, III\mapsto 3, IV\mapsto 4, V\mapsto 5, VI\mapsto 6\}\]
}
\item Podaj rozkład prawdopodobieństwa zmiennej losowej $X$. 
\ans{
\begin{tabular}{c|cccccc}
	$x_i$ & 1 & 2 & 3 & 4 & 5 & 6 \\
	\hline
	$P(X=x_i)$ & $\frac{1}{6}$ & $\frac{1}{6}$ & $\frac{1}{6}$ & $\frac{1}{6}$ & $\frac{1}{6}$ & $\frac{1}{6}$ \\
\end{tabular}

Możemy to też oczywiście zapisać inaczej, np. $P(X=i)=\frac{1}{6}$ dla $i\in\{1, 2, \ldots, 6\}$. Istotne, żeby pokazać przyporządkowanie prawdopodobieństw do poszczególnych punktów skokowych.
}
\item Podaj dystrybuantę zmiennej losowej $X$ i narysuj jej wykres.
\ans{
	\[ F(x)=P(X\leq x)=\sum_{x_i\leq x} P(X=x_i) = \begin{cases} 
	0 & x\in\left(-\infty,1\right) \\
	\frac{1}{6} & x\in\left[1, 2\right) \\
	\frac{2}{6} & x\in\left[2, 3\right) \\
	\frac{3}{6} & x\in\left[3, 4\right) \\
	\frac{4}{6} & x\in\left[4, 5\right) \\
	\frac{5}{6} & x\in\left[5, 6\right) \\
	1 & x\in\left[6, \infty\right) \end{cases} \]
	
	Prościej zapisać to samo w formie tabeli:\\
	\begin{tabular}{c|ccccccc}
		$x$ & $\left(-\infty, 1\right)$ & $\left[1, 2\right)$ & $\left[2, 3\right)$ & $\left[3, 4\right)$ & $\left[4, 5\right)$ & $\left[5, 6\right)$ & $\left[6, \infty\right)$ \\
		\hline
		$F(x)$ & $0$ & $\frac{1}{6}$ & $\frac{2}{6}$ & $\frac{3}{6}$ & $\frac{4}{6}$ & $\frac{5}{6}$ & $1$ \\
	\end{tabular}
}
\item Oblicz $P(X<4)$ korzystając z rozkładu prawdopodobieństwa. \ans{\[P(X<4)=P(X=1)+P(X=2)+P(X=3)=\frac{1}{6}+\frac{1}{6}+\frac{1}{6}=\frac{3}{6}\]}
\item Oblicz $P(X<4)$ korzystając z dystrybuanty. 
\ans{\[P(X<4)=\lim_{x\to 4-} F(x) = F(3) = \frac{1}{2}\]
	Przejście od $x\to 4-$ do $3$ wynika z tego, że de facto interesuje nas dowolna liczba z przedziału kończącego się w 4 otwartym, tj. przedziału $\left[3,4\right)$, a zatem np. $3$.
}
\item Oblicz $P(X>2)$ korzystając z rozkładu prawdopodobieństwa. \ans{\[P(X>2)=P(X=3)+P(X=4)+P(X=5)+P(X=6)=\frac{1}{6}+\frac{1}{6}+\frac{1}{6}+\frac{1}{6}=\frac{4}{6}\]}
\item Oblicz $P(X>2)$ korzystając z dystrybuanty. \ans{\[P(X>2)=1-P(X\leq 2)=1-F(2)=1-\frac{2}{6}=\frac{2}{3}\]}
\item Oblicz wartość średnią. \ans{
	\[EX=\sum_{x_i} x_i\cdot P(X=x_i) = 1\cdot\frac{1}{6} +2\cdot\frac{1}{6} +3\cdot\frac{1}{6} +4\cdot\frac{1}{6} +5\cdot\frac{1}{6} +6\cdot\frac{1}{6} +  \frac{21}{6}=3{,}5\]
}
\item Oblicz wariancję.
\ans{
	Dwa sposoby:
	\begin{enumerate}
		\item Bezpośrednio z definicji
		\begin{align*}
		 D^2X = & E\left[(X-EX)^2\right] = \sum_{x_i} \left(x_i-EX)^2\cdot P(X=x_i)\right) = \\ & (1 - 3{,}5)^2\cdot\frac{1}{6} (2 - 3{,}5)^2\cdot\frac{1}{6} (3 - 3{,}5)^2\cdot\frac{1}{6} (4 - 3{,}5)^2\cdot\frac{1}{6} (5 - 3{,}5)^2\cdot\frac{1}{6} (6 - 3{,}5)^2\cdot\frac{1}{6} = \\ &		 
		\frac{1}{6}((-2{,}5)^2+(-1{,}5)^2+(-0{,}5)^2+2{,}5^2+1{,}5^2+0{,}5^2)
		= 2\frac{11}{12}
		\end{align*}
		\item Posługując się przekształconym wzorem:
		\begin{align*}
			D^2X = & E(X^2)-(EX)^2 = \sum_{x_i} x_i^2P(X=x_i) - (EX)^2 = \\
			& 1^2\cdot\frac{1}{6} + 2^2\cdot\frac{1}{6} + 3^2\cdot\frac{1}{6} + 4^2\cdot\frac{1}{6} + 5^2\cdot\frac{1}{6} + 6^2\cdot\frac{1}{6} - (3,5)^2 = \frac{91}{6} - \frac{49}{4} = \frac{35}{12} = 2\frac{11}{12} \\
		\end{align*}
	\end{enumerate}
}
% \ans{$D^2X=\frac{1}{6}((-2{,}5)^2+(-1{,}5)^2+(-0{,}5)^2+2{,}5^2+1{,}5^2+0{,}5^2)$}
\end{enumerate}
\item Prawdopodobieństwo trafienia do celu w jednym strzale jest równie $\frac{1}{5}$. Niech $X$ przyjmuje wartość 1 jeżeli udało się trafić i 0 w przeciwnym przypadku.
\begin{enumerate}
\item Podaj rozkład zmiennej losowej $X$. \ans{$P(X=1)=p\quad P(X=0)=1-p \qquad p=\frac{1}{5}$}
\item Oblicz średnią liczbę celnych strzałów. \ans{$EX=1\cdot p+0\cdot (1-p)=p=\frac{1}{5}$}
\item Oblicz odchylenie standardowe zmiennej losowej $X$. \ans{$DX= \sqrt{D^2X} = \sqrt{E(X^2)-(EX)^2} = \sqrt{1^2\cdot p - p^2} = \sqrt{p(1-p)}=\sqrt{\frac{4}{25}} = \frac{2}{5}$}
\item Podaj najbardziej prawdopodobną wartość zmiennej losowej $X$. \ans{$0$, bo $P(X=0)>P(X=1)$, a mamy do wyboru tylko $0$ i $1$}
\end{enumerate}
\item Rozważamy doświadczenie polegające na obserwacji sumy oczek na dwóch uczciwych kostkach sześciościennych
\begin{enumerate}
\item Zaproponuj zmienną losową $X$ odpowiednią do tego doświadczenia. 
\ans{
	Niech pojedyncze zdarzenie elementarne to para uporządkowana: (wynik na 1. kostce, wynik na 2. kostce), gdzie wyniki są kodowane przez wyrzuconą liczbę oczek 1, 2, \ldots, 6. Wtedy przestrzeń zdarzeń elementarnych wygląda następująco:
	\[ \Omega = \{1, 2, \ldots, 6\}\times \{1, 2, \ldots, 6\} = \{ (1,1), (1,2), \ldots, (1,6), (2, 1), (2, 2), \ldots, (6, 6)\} \]
	
	Wtedy łatwo zdefiniować zmienną:
	\[ X((i, j)) = i+j \qquad \forall (i,j)\in \Omega \]
	
}
\item Podaj rozkład prawdopodobieństwa zmiennej losowej $X$. 
\begin{ansenv}
	\begin{tabular}{c|cccccccccccc}
		$x_i$ & 2 & 3 & 4 & 5 & 6 & 7 & 8 & 9 & 10 & 11 & 12 \\
		\hline
		$P(X=x_i)$ & $\frac{1}{36}$ & $\frac{2}{36}$ & $\frac{3}{36}$ & $\frac{4}{36}$ & $\frac{5}{36}$ & $\frac{6}{36}$ & $\frac{5}{36}$ & $\frac{4}{36}$ & $\frac{3}{36}$ & $\frac{2}{36}$ & $\frac{1}{36}$ \\
	\end{tabular}
\end{ansenv}
\item Podaj dystrybuantę zmiennej losowej $X$ i narysuj jej wykres.
\begin{ansenv}
	\begin{tabular}{c|cccccccccccc}
	$x$ & $\left(-\infty, 2\right)$ & $[2, 3)$ & $[3, 4)$ & $[4, 5)$ & $[5, 6)$ &	
	 $[6, 7)$ & $[7, 8)$ & $[8, 9)$ & $[9, 10)$ & $[10, 11)$ & $[11, 12)$ & $[12, \infty)$ \\
	 \hline
	$F(x)$ & 0 & $\frac{1}{36}$ & $\frac{3}{36}$ & $\frac{6}{36}$ & $\frac{10}{36}$ & $\frac{15}{36}$ & $\frac{21}{36}$ & $\frac{26}{36}$ & $\frac{30}{36}$ & $\frac{33}{36}$ & $\frac{35}{36}$ & $1$ \\
\end{tabular}
\end{ansenv}
\item Oblicz $P(X>5)$ korzystając z dystrybuanty.
\ans{\[ P(X>5)=1-P(X\leq 5)=1-F(5)=1-\frac{10}{36}=\frac{26}{36} \]}
\item Oblicz wartość średnią. \ans{
	\[ EX = \sum_{x_i} x_iP(X=x_i) = 2\cdot\frac{1}{36} + 3\cdot\frac{2}{36} + 4\cdot\frac{3}{36} + 5\cdot\frac{4}{36} + 6\cdot\frac{5}{36} + 7\cdot\frac{6}{36} + 8\cdot\frac{5}{36} + 9\cdot\frac{4}{36} + 10\cdot\frac{3}{36} + 11\cdot\frac{2}{36} + 12\cdot\frac{1}{36} = 7
	\]
}
\item Oblicz odchylenie standardowe. \ans{
	\[ EX^2 = 2^2\cdot\frac{1}{36} + 3^2\cdot\frac{2}{36} + 4^2\cdot\frac{3}{36} + 5^2\cdot\frac{4}{36} + 6^2\cdot\frac{5}{36} + 7^2\cdot\frac{6}{36} + 8^2\cdot\frac{5}{36} + 9^2\cdot\frac{4}{36} + 10^2\cdot\frac{3}{36} + 11^2\cdot\frac{2}{36} + 12^2\cdot\frac{1}{36} = \frac{1974}{36} \]
	
	\[	\sqrt{D^2X}=\sqrt{E(X^2)-(EX)^2} = \sqrt{\frac{1974}{36}-49}\approx 2{,}42\]}
\end{enumerate}
\item Kasyno w Bree wprowadziło następującą grę: gracz rozpoczyna z jednym punktem i rzuca uczciwą kostką sześciościenną.
Jeżeli wypadnie 1, gracz traci wszystkie punkty i kończy grę.
Jeżeli wypadnie 2, 3 lub 4, gracz podwaja liczbę punktów i kończy grę.
Jeżeli wypadnie 5 lub 6, gracz podwaja liczbę punktów i może ponownie rzucić kostką.
W ciągu gry można maksymalnie rzucić trzy razy kostką, tzn. w trzecim rzucie wyniki 5 lub 6 traktuje się tak samo jak wyniki 2, 3, 4.
Niech zmienna losowa $X$ oznacza liczbę punktów gracza na końcu gry.
\begin{enumerate}
\item Podaj zbiór punktów skokowych, tj. możliwych wartości, zmiennej losowej $X$.
\ans{$\{0, 2, 4, 8\}$}
\item Podaj rozkład prawdopodobieństwa zmiennej losowej $X$.
\ans{
\[ P(X=2)=\frac{3}{6}=\frac{1}{2} \qquad P(X=4)=\frac{2}{6}\frac{3}{6}=\frac{1}{6} \qquad P(X=8)=\frac{2}{6}\frac{2}{6}\frac{5}{6}=\frac{5}{54} \]
\[ P(X=0)=\frac{1}{6}+\frac{2}{6}\frac{1}{6}+\frac{2}{6}\frac{2}{6}\frac{1}{6}=\frac{13}{54}\]
}
\item Narysuj wykres dystrybuanty zmiennej losowej $X$.
\item Oblicz prawdopodobieństwo, że gracz zdobędzie niezerową liczbę punktów.
\ans{ \[P(X>0)=1-P(X=0)=\frac{41}{54} \]}
\item Oblicz średnią liczbę punktów gracza na końcu gry.
\ans{
\[ EX=2\cdot\frac{1}{2}+4\cdot\frac{1}{6}+8\cdot\frac{5}{54}=\frac{65}{27} \]
}
\item Obliczy odchylenie standardowe zmiennej losowej $X$.
\ans{
\[ EX^2 = 4\frac{1}{2}+16\frac{1}{6}+64\frac{5}{54}=\frac{286}{27} \qquad D^2X = \frac{286}{27}-\left(\frac{65}{27}\right)^2=\frac{3497}{27^2} \qquad DX=\frac{\sqrt{3497}}{27} \]
}
\item Jaka jest najbardziej prawdopodobna wartość zmiennej losowej $X$?
\ans{2}
\end{enumerate}
\item Ted Cotton po pracy udaje się do kasyna w Bree, które niedawno wprowadziło następującą grę: gracz rozpoczyna z jednym punktem i rzuca uczciwą kostką sześciościenną.
Jeżeli wypadnie 1, gracz traci wszystkie punkty i kończy grę.
Jeżeli wypadnie 2, 3 lub 4, gracz podwaja liczbę punktów i kończy grę.
Jeżeli wypadnie 5 lub 6, gracz podwaja liczbę punktów i może ponownie rzucić kostką.
Nie ma górnego ograniczenia na liczbę rzutów ani na wysokość wygranej.
Niech zmienna losowa $X$ oznacza liczbę punktów gracza na końcu gry.

Podpowiedź: \[
\sum_{n=0}^\infty q^n = 
\begin{cases} 
	\frac{1}{1-q} & \left|q\right|<1 \\
	\infty & \text{wpp}
\end{cases}
\]

\begin{enumerate}
\item Podaj zbiór punktów skokowych, tj. możliwych wartości, zmiennej losowej $X$.
\ans{$\{0, 2, 4, \ldots, 2^{n}, \ldots\}=\{0\}\cup\{2^n\colon n\in\mathbb{N}_{+} \}$}
\item Podaj funkcję prawdopodobieństwa $P$ zmiennej losowej $X$.
\ans{
\[
P(X=k)=\begin{cases} 
\left(\frac{1}{3}\right)^{n-1}\frac{1}{2} & k=2^n\, n\in\mathbb{N}_{+} \\
\frac{1}{4} & k=0\quad \text{(wyprowadzenie niżej)}
\end{cases} \]
}
\item Oblicz prawdopodobieństwo, że gracz skończy grę z niezerową liczbą punktów.
\ans{\[P(X>0)=\sum_{n=1}^\infty \left(\frac{1}{3}\right)^{n-1}\frac{1}{2}=
\frac{1}{2}\sum_{n=0}^\infty \left(\frac{1}{3}\right)^n = \frac{1}{2}\cdot \frac{1}{1-\frac{1}{3}}=\frac{3}{4} \]}
\item Oblicz średnią liczbę punktów gracza na końcu gry.
\ans{\[
EX = \sum_{n=1}^\infty 2^{n}\left(\frac{1}{3}\right)^{n-1}\frac{1}{2} = \sum_{n=1}^\infty \left(\frac{2}{3}\right)^{n-1} = \sum_{n=0}^\infty \left(\frac{2}{3}\right)^n = \frac{1}{1-\frac{2}{3}} = 3
\]}
\item Obliczy odchylenie standardowe zmiennej losowej $X$.
\ans{$DX=\sqrt{E(X^2)-(EX)^2}$ \\
\[
EX^2=\sum_{n=1}^\infty \left(2^{n}\right)^2\left(\frac{1}{3}\right)^{n-1}\frac{1}{2} =
\frac{1}{2} \sum_{n=1}^\infty 4^{n}\left(\frac{1}{3}\right)^{n-1} =
\frac{4}{2} \sum_{n=1}^\infty 4^{n-1}\left(\frac{1}{3}\right)^{n-1} =
\frac{1}{2} \sum_{n=0}^\infty \left(\frac{4}{3}\right)^n \to\infty
\]
Wyciągamy z tego wniosek, że nie można obliczyć odchylenia standardowego tej zmiennej losowej.
}
\item Jaka jest najbardziej prawdopodobna wartość zmiennej losowej $X$?
\ans{2}
\end{enumerate}
\item Bilbo bierze udział w grze, w której punkty zdobywa się za trafianie kamykami do celu. Każdemu zawodnikowi przysługuje maksymalnie pięć rzutów, przy czym:
\begin{itemize}
\item na początku każdy zawodnik ma jeden punkt;
\item każdy trafiony rzut powoduje podwojenie liczby posiadanych punktów;
\item rzut nietrafiony oznacza koniec gry dla danego zawodnika.
\end{itemize}
Prawdopodobieństwo, że Bilbo w pojedynczym rzucie trafi do celu wynosi $0{,}7$. Niech $X$ będzie zmienną losową oznaczającą liczbę punktów zdobytych przez
Bilba.
\begin{enumerate}
\item Podaj funkcję prawdopodobieństwa zmiennej losowej $X$.
\item Oblicz dystrybuantę zmiennej losowej $X$.
\item Oblicz $EX$.
\item Oblicz wariancję zmiennej losowej $X$.
\item Oblicz $P(5\leq X\leq 30)$.
\end{enumerate}


\item Dana jest następująca gra: gracz rzuca uczciwą kostką sześciościenną tak długo, dopóki nie wyrzuci piątki bądź
szóstki, ale nie więcej niż trzy razy. Jeżeli uda mu się wyrzucić założoną liczbę oczek w $k$-tym rzucie, wygrywa $5-k$
zł, w przeciwnym razie nie wygrywa nic. Niech zmienna losowa $X$ odpowiada wysokości wygranej. Podaj, dbając
by przedstawić tok rozumowania:
\begin{enumerate}
\item funkcję prawdopodobieństwa zmiennej losowej $X$;
\item dystrybuatnę zmiennej losowej $X$;
\item prawdopodobieństwo, że gracz wygra nie mniej niż 3, a nie więcej niż 4 zł;
\item średnią wartość wygranej;
\item odchylenie standardowe zmiennej losowej $X$.
\end{enumerate}
\item Linia technologiczna składająca samochody składa 5 sztuk w ciągu godziny. Przy okazji składania każdego
z samochodów istnieje prawdopodobieństwo $0{,}1$, że maszyna montująca drzwi kierowcy ulegnie rozregulowaniu
i będzie rysować lakier. Co gorsza, rozregulowanie jest trwałe w tym sensie, że dopóki technik nie wyreguluje
maszyny wszystkie montowane drzwi będą rysowane. W związku z kryzysem firma postanowiła wprowadzić
oszczędności i technik regulujący maszyny dokonuje kontroli tylko raz na godzinę. Porysowanie lakieru na jednych
drzwiach to koszt 600 zł. Niech zmienna $X$ odpowiada kwocie, która firma straciła w ciągu godziny w wyniku
porysowania lakieru przez maszynę montującą drzwi. Podaj, dbając by przedstawić tok rozumowania:
\begin{enumerate}
\item funkcję prawdopodobieństwa zmiennej losowej $X$;
\item dystrybuatnę zmiennej losowej $X$;
\item średnią stratę;
\item odchylenie standardowe zmiennej losowej $X$;
\item prawdopodobieństwo, że firma straci przynajmniej 1000 zł w ciągu godziny
\end{enumerate}



\end{enumerate}



\clearpage
\section{Dyskretne rozkłady prawdopodobieństwa}
\begin{enumerate}
\item Prawdopodobieństwo trafienia do celu w jednym strzale jest równie $\frac{1}{5}$. Niech $X$ oznacza liczbę strzałów celnych w wykonanej serii 5 niezależnych strzałów. 
\begin{enumerate}
\item Podaj rozkład zmiennej losowej $X$. \ans{
	Mamy do czynienia z powtarzającymi się próbami, niezależnymi od siebie, o stałym prawdopodobieństwie sukcesu i których liczba jest z góry ustalona. Zatem korzystamy z rozkładu dwumianowego:
	\[P(X=k)={5 \choose k}\left(\frac{1}{5}\right)^k\left(\frac{4}{5}\right)^{5-k} \qquad k\in\{0,1,\ldots,5\}\]
}
\item Oblicz prawdopodobieństwo, że liczba strzałów celnych będzie nie mniejsza niż 2. \ans{$P(X\geq 2)=1-P(X=0)-P(X=1)=1-\frac{4^5}{5^5}-5\cdot\frac{4^4}{5^5}=\frac{821}{3125}\approx0{,}263$}
\item Oblicz średnią liczbę celnych strzałów. \ans{$EX=np=1$}
\item Oblicz odchylenie standardowe zmiennej losowej $X$. \ans{$DX=\sqrt{np(1-p)}=\sqrt{\frac{4}{5}}$}
\item Podaj najbardziej prawdopodobną liczbę celnych strzałów. \ans{
	Patrzymy czy iloczyn $(n+1)p$ jest liczbą całkowitą. W tym przypadku nie jest ($(n+1)p=\frac{6}{5}$), a zatem jest jeden punkt najbardziej prawdopodobny:
	$\lfloor(n+1)p\rfloor=1$}
\end{enumerate}
\item Linia 64 jeżdżąca na trasie Literacka--Kacza jest obsługiwana przez 7 autobusów, które psują się przypadkowo i niezależnie od siebie. Każdy autobus może w ciągu całego dnia zepsuć się z~prawdopodobieństwem $0{,}25$. Niech $X$ oznacza liczbę autobusów, które w ciągu dnia uległy awarii i musiały zjechać do zajezdni.
\begin{enumerate}
\item Podaj rozkład zmiennej losowej $X$.
\ans{
	Mamy do czynienia z sytuacją analogiczną jak w poprzednim zadaniu, a zatem
	\[ P(X=k)={7\choose k}\left(0{,}25\right)^k\left(0{,}75\right)^{7-k} \qquad k\in\{0,1,\ldots,7\} \]
}
\item Oblicz prawdopodobieństwo, że w ciągu całego dnia zepsują się przynajmniej 3 autobusy.
\ans{
	\begin{align*}
	 P(X\geq 3) = & 1-P(X<3)=1-P(X=0)-P(X=1)-P(X=2) = \\ & 1 - (0{,}75)^7 - 7\cdot 0{,}25 \cdot (0{,}75)^6 - \frac{7\cdot 6}{2} \cdot (0{,}25)^2\cdot (0{,}75)^5 
	\end{align*}
}
\item Oblicz średnią liczbę zepsutych autobusów.
\ans{
	\[ EX=np=7\cdot 0{,}25 = 1{,}75 \]
}
\item Oblicz odchylenie standardowe zmiennej losowej $X$.
\ans{
	\[ DX=\sqrt{np(1-p)}=\sqrt{7\cdot 0{,}25\cdot 0{,}75}=\frac{\sqrt{21}}{4} \approx 1{,}15 \]
}
\item Podaj najbardziej prawdopodobną liczbę zepsutych autobusów. \ans{
	Iloczyn $(n+1)p=2$ jest liczbą całkowitą, zatem mamy dwa punkty skokowe, które są najbardziej prawdopodobne:
	\[(n+1)p-1=1, (n+1)p=2\]}
\end{enumerate}
\item W Minas Tirith gromadzą zapasy na wypadek oblężenia. Jeżeli mięso,
pakowane w beczki, jest źle wysuszone może się zepsuć. Prawdopodobieństwo, że
tak się stanie wynosi $0{,}0045$ niezależnie dla każdej beczki. W~piwnicach zgromadzono 1000 beczek z mięsem,
niech $X$ oznacza liczbę beczek z zepsutym mięsem.  \begin{enumerate}
\item Podaj rozkład zmiennej losowej $X$. \ans{
	Mamy podobną sytuację jak w poprzednich zadaniach, możemy więc posłużyć się rozkładem dwumianowym:
	\[P(X=k)={1000\choose k}p^k(1-p)^{1000-k}\]
	Ponieważ mamy do czynienia z dużą liczbą mało prawdopodobnych prób ($n>50, p<0{,}1, np<10$) możemy posłużyć się też przybliżeniem rozkładem Poissona:
	\[ P(X=k) = \frac{\lambda^k}{k!}e^{-\lambda} \quad \lambda=1000\cdot0{,}0045=4{,}5\]}
\item Oblicz prawdopodobieństwo, że zepsuje się nie więcej niż 5 beczek. \ans{
	Możemy skorzystać ze stablicowanego rozkładu Poissona (dostępny na końcu tego rozdziału), odczytujemy wartości z wiersza odpowiadającego $\lambda=4{,}5$ i kolumn odpowiadających punktom skokowym, które nas interesują:
	\begin{align*}
		P(X\leq 5)= & P(X=0)+P(X=1)+P(X=2)+P(X=3)+P(X=4)+P(X=5) = \\ &
		0{,}0111	+ 0{,}0500	+ 0{,}1125	+ 0{,}1687	+ 0{,}1898	+ 0{,}1708 =
	0{,}70290
	\end{align*}
}
\item Oblicz średnią liczbę zepsutych beczek. \ans{$EX=\lambda=4{,}5$}
\item Oblicz odchylenie standardowe zmiennej losowej $X$. \ans{$DX=\sqrt{\lambda} = \sqrt{4{,}5}$}
\end{enumerate}
\item Hobbici znani są ze swojej intensywnej i obfitej korespondencji. W Michael
Delving mieszka 500 hobbitów, a~w~ciągu jednego dnia do urzędu pocztowego
przychodzi $X$ hobbitów. Hobbici do urzędu pocztowego przychodzą niezależnie od
siebie i~z~równym prawdopodobieństwem $p$, a~każdego dnia jest ich tam
\emph{średnio} 75. Załóż, że zmienna losowa $X$ ma rozkład dwumianowy.
\begin{enumerate}
\item Oblicz wartość prawdopodobieństwa $p$ przyjścia do urzędu przez pojedynczego hobbita.
\ans{
	Hobbici przychodzą niezależnie i z równym prawdopodobieństwem, jest ich też stała liczba $n=500$, a zatem mamy do czynienia z modelem rozkładu dwumianowego. Znamy średnią, więc łatwo obliczyć prawdopodobieństwo:
	\[ p=\frac{EX}{n}=\frac{75}{500}=\frac{3}{20} \]
}
\item Podaj  (w formie funkcji prawdopodobieństwa) rozkład prawdopodobieństwa zmiennej losowej $X$.
\ans{
	\[ P(X=k)={500 \choose k}\left(\frac{3}{20}\right)^k\left(\frac{17}{20}\right)^{500-k} k\in\{0,1,\ldots,500\} \]
}
\item Ile różnych wartości może przyjąć dystrybuanta zmiennej losowej $X$? Odpowiedź uzasadnij.
\ans{
	Mamy 500 hobbitów, a zatem 501 punktów skokowych (0 do 500). Dystrybuanta zmiennej losowej typu skokowego rośnie w każdym punkcie skokowym, zatem mamy 501 wzrostów. Żeby mieć 501 wzrostów musimy mieć 502 różne wartości.
}
\item Jaką wartość przyjmie dystrybuanta zmiennej losowej $X$ w punkcie 1000?
\ans{
	\[F(1000)=P(X\leq 1000)=1\]
}
\item Oblicz prawdopodobieństwo, że przez cały dzień do urzędu nikt nie przyjdzie.
\ans{
	\[ P(X=0)=\left(\frac{17}{20}\right)^{500} \approx 0 \]
}
\item Podaj wartość $EX$.
\ans{
	\[ EX=75 \] (wystarczy przepisać z treści zadania)
}
\item Oblicz odchylenie standardowe zmiennej losowej $X$.
\ans{
	\[ DX=\sqrt{np(1-p)}=\sqrt{500\cdot\frac{3}{20}\cdot\frac{17}{20}}\approx 7{,}98 \]
}
\item Oblicz najbardziej prawdopodobną liczbę hobbitów, którzy przyjdą w ciągu dnia do urzędu.
\ans{
	Iloczyn $(n+1)p$ nie jest liczbą całkowitą, zatem bierzemy z niego podłogę:
	\[\lfloor(n+1)p\rfloor = 75 \]
}
\end{enumerate}
\item Słoń Nino przechodzi przez skład porcelany zawierający 4~regały, z~których każdy zawiera talerze warte 3000 zł.
	Każdy z regałów może zostać przewrócony przez Nina z~prawdopodobieństwem $0{,}2$.
	Niech zmienna $X$ oznacza wartość uszkodzonych talerzy (zakładając, że przewrócenie regału powoduje uszkodzenie się wszystkich zawartych na nim talerzy).
Podaj, dbając by przedstawić tok rozumowania:
\begin{enumerate}
\item funkcję prawdopodobieństwa zmiennej losowej $X$;
\ans{
	Zmienna losowa $X$ \textbf{nie} ma rozkładu dwumianowego i nie możemy bezpośrednio posłużyć się rozwiązaniami jak z poprzednich zadań. Na szczęście możemy posłużyć się zmienną pomocniczą o rozkładzie dwumianowym. Niech zmienna $Y$ oznacza liczbę przewróconych regałów:
	\[ P(Y=k)={4 \choose k}\left(0{,}2\right)^k\left(0{,}8\right)^{4-k} \qquad k\in\left\{0,1,2,3,4\right\} \]
	Na tej bazie zdefiniujemy zmienną $X$:
	\[ X=3000 Y \]
	I jej rozkład prawdopodobieństwa:
	\[ P(X=3000k) = P(Y=k) = {4 \choose k}\left(0{,}2\right)^k\left(0{,}8\right)^{4-k} \qquad k\in\left\{0,1,2,3,4\right\} \]
	albo:
	\[ P(X=k) = P(Y=\frac{k}{3000}) = {4 \choose \frac{k}{3000}}\left(0{,}2\right)^{\frac{k}{3000}}\left(0{,}8\right)^{4-\frac{k}{3000}} \qquad k\in\left\{0,3000,6000,9000,12000\right\} \]
}
\item dystrybuatnę zmiennej losowej $X$;
\begin{ansenv}
	\begin{tabular}{c|cccccc}
		$x$ & $\left(-\infty, 0\right)$ & $\left[0, 3000\right)$& $\left[ 3000, 6000\right)$& $\left[6000, 9000\right)$& $\left[9000, 12000\right)$& $\left[12000, \infty\right)$ \\
		\hline
		$F(x)$ & 0 & $\frac{256}{625}$ & $\frac{512}{625}$ & $\frac{608}{625}$ & $\frac{624}{625}$ & $1$
	\end{tabular}
\end{ansenv}
\item prawdopodobieństwo, że słoń stłucze talerzy za co najmniej $5{,}5$ tys. zł.
\ans{
	\[ P(X\geq 5500)=1-P(X<5500)=1-F(5499)=1-\frac{512}{625}=\frac{113}{625} \]
}
\item średnią wartość straty;
\ans{
	\[ EX = E(3000Y)=3000EY=3000\cdot 4\cdot 0{,}2=2400 \]
}
\item odchylenie standardowe zmiennej losowej $X$.
\ans{
	\[ DX = D(3000Y)=3000DY=3000\cdot\sqrt{4\cdot 0{,}2\cdot 0{,}8}=2400 \]
	Uwaga! Odpowiedź \textbf{nie} brzmi 1920.
	
	Gdyby pytanie było o wariancję, to wtedy trzeba uwzględnić, że tam jest kwadrat:
	\[ D^2X=D^2(3000Y)=3000^2D^2Y=3000^2\cdot 4\cdot 0{,}2\cdot 0{,}8 \]
}
\end{enumerate}
\newpage
\item W piwnicach Brandy Hallu zgromadzono 500 butelek soków na zimę.
Niestety, z poprzednich lat wiadomo, że każda butelka ma $0{,}5\%$ szans nie dotrwać zimy: spleśnieje, skwaśnieje itp.
Niech $X$ będzie zmienną losową oznaczającą liczbę zepsutych butelek soku.
Niech $Y$ będzie zmienną losową oznaczającą liczbę zdatnych do spożycia butelek soku, to znaczy $X+Y=500$.

\begin{enumerate}
\item Podaj funkcję prawdopodobieństwa zmiennej losowej $X$.
\item Podaj wartość średnią i odchylenie standardowe zmiennej losowej $X$.
\item Podaj funkcję prawdopodobieństwa zmiennej losowej $Y$.
\item Podaj wartość średnią i odchylenie standardowe zmiennej losowej $Y$.
\item Podaj najbardziej prawdopodobną liczbę zepsutych butelek.
\item Oblicz prawdopodobieństwo, że liczba przynajmniej 496 butelek soku będzie się nadawało do spożycia.
\item Oblicz prawdopodobieństwo, że liczba zepsutych butelek będzie różniła się od oczekiwanej liczby zepsutych butelek o mniej niż odchylenie standardowe.
\item Oszacuj (z dokładnością do $0{,}1$) prawdopodobieństwo, że liczba butelek nadających się do spożycia przekroczy 300. Odpowiedź uzasadnij.
\end{enumerate}
\item W piwnicy Bamfurlong znajduje się 10 słoików szczególnie cennej konfitury ze świecących gigantycznych pieczarek.
Właściciel przechowuje je na Święto Zimowe, ale nie wie, że ze względu na bliskość rzeki, każdy słoik ma 20\% szans spleśnieć do tego czasu!
Niech $X$ będzie zmienną losową oznaczającą liczbę słoików, które nie zapleśniały do Święta.
Niech $Y$ będzie zmienną losową oznaczającą liczbę słoików, które spleśniały.
Inaczej: $X+Y=10$.

\begin{enumerate}
\item Podaj funkcję prawdopodobieństwa zmiennej losowej $X$.
\item Podaj wartość średnią i odchylenie standardowe zmiennej losowej $X$.
\item Podaj funkcję prawdopodobieństwa zmiennej losowej $Y$.
\item Podaj wartość średnią i odchylenie standardowe zmiennej losowej $Y$.
\item Podaj najbardziej prawdopodobną liczbę zepsutych słoików.
\item Oblicz prawdopodobieństwo, że przynajmniej 8 słoików będzie się nadawało do spożycia.
\item Oblicz prawdopodobieństwo, że liczba zepsutych słoików będzie różniła się od oczekiwanej liczby zepsutych słoików o mniej niż odchylenie standardowe.
\end{enumerate}


\end{enumerate}
\vfill
{
	\footnotesize
\noindent\begin{tabular}{|r|r|r|r|r|r|r||r|r|r|r|r|r|r|}
\hline
\multicolumn{14}{|c|}{Rozkład Poissona} \\
\diagbox{$\lambda$}{$k$} & \textbf{0}	& \textbf{1}	& \textbf{2}	& \textbf{3}	& \textbf{4}	& \textbf{5} & \diagbox{$\lambda$}{$k$} & \textbf{0}	& \textbf{1}	& \textbf{2}	& \textbf{3}	& \textbf{4}	& \textbf{5}\\
\hline
\textbf{0,5}	& 0,6065	& 0,3033	& 0,0758	& 0,0126	& 0,0016	& 0,0002 &	\textbf{5,5}	& 0,0041	& 0,0225	& 0,0618	& 0,1133	& 0,1558	& 0,1714 \\
\hline                                                                                                          
\textbf{1,0}	& 0,3679	& 0,3679	& 0,1839	& 0,0613	& 0,0153	& 0,0031 &	\textbf{6,0}	& 0,0025	& 0,0149	& 0,0446	& 0,0892	& 0,1339	& 0,1606 \\
\hline                                                                                                          
\textbf{1,5}	& 0,2231	& 0,3347	& 0,2510	& 0,1255	& 0,0471	& 0,0141 &	\textbf{6,5}	& 0,0015	& 0,0098	& 0,0318	& 0,0688	& 0,1118	& 0,1454 \\
\hline                                                                                                          
\textbf{2,0}	& 0,1353	& 0,2707	& 0,2707	& 0,1804	& 0,0902	& 0,0361 &	\textbf{7,0}	& 0,0009	& 0,0064	& 0,0223	& 0,0521	& 0,0912	& 0,1277 \\
\hline                                                                                                          
\textbf{2,5}	& 0,0821	& 0,2052	& 0,2565	& 0,2138	& 0,1336	& 0,0668 &	\textbf{7,5}	& 0,0006	& 0,0041	& 0,0156	& 0,0389	& 0,0729	& 0,1094 \\
\hline                                                                                                          
\textbf{3,0}	& 0,0498	& 0,1494	& 0,2240	& 0,2240	& 0,1680	& 0,1008 &	\textbf{8,0}	& 0,0003	& 0,0027	& 0,0107	& 0,0286	& 0,0573	& 0,0916 \\
\hline                                                                                                          
\textbf{3,5}	& 0,0302	& 0,1057	& 0,1850	& 0,2158	& 0,1888	& 0,1322 &	\textbf{8,5}	& 0,0002	& 0,0017	& 0,0074	& 0,0208	& 0,0443	& 0,0752 \\
\hline                                                                                                         
\textbf{4,0}	& 0,0183	& 0,0733	& 0,1465	& 0,1954	& 0,1954	& 0,1563 &	\textbf{9,0}	& 0,0001	& 0,0011	& 0,0050	& 0,0150	& 0,0337	& 0,0607 \\
\hline                                                                                                         
\textbf{4,5}	& 0,0111	& 0,0500	& 0,1125	& 0,1687	& 0,1898	& 0,1708 &	\textbf{9,5}	& 0,0001	& 0,0007	& 0,0034	& 0,0107	& 0,0254	& 0,0483 \\
\hline                                                                                                        
\textbf{5,0}	& 0,0067	& 0,0337	& 0,0842	& 0,1404	& 0,1755	& 0,1755 &	\textbf{10,0}	& 0,0000	& 0,0005	& 0,0023	& 0,0076	& 0,0189	& 0,0378 \\
\hline
\end{tabular}
}

\clearpage
\section{Zmienne losowe jednowymiarowe typu ciągłego}

\begin{enumerate}
%\item Rozważmy funkcję \[f(x)=\begin{cases} Ax^{-4} & \left|x\right|\geq 1 \\ 0 & \text{wpp} \end{cases} \]
%\begin{enumerate}
%\item Dla jakiej wartości stałej $A$ ta funkcja jest gęstością prawdopodobieństwa pewnej zmiennej losowej $X$? \ans{$\frac{3}{2}$}
%\item Oblicz dystrybuantę zmiennej losowej $X$ i narysuj jej wykres. \ans{\[\begin{cases}-\frac{x^{-3}}{2} & x\leq-1\\\frac{1}{2} & -1<x\leq1\\1-\frac{x^{-3}}{2} & x>1 \end{cases}\]}
%\item Oblicz prawdopodobieństwo, że zmienna losowa $X$ przyjmie wartość większą niż 2. \ans{$\frac{1}{16}$}
%\item Oblicz prawdopodobieństwo, że zmienna losowa $X$ przyjmie wartość równą 2. \ans{$0$}
%\item Oblicz wartość średnią. \ans{$0$}
%\item Oblicz wariancję. \ans{$3$}
%\end{enumerate}
\item Dana jest funkcja 
\[ f(x)=\begin{cases} ax^2 & x\in \left<0,3\right> \\ 0 & \text{w przeciwnym przypadku} \end{cases} \]
\begin{enumerate}
\item Dla jakiej wartości parametru $a$ funkcja $f(x)$ jest funkcją gęstości prawdopodobieństwa pewnej zmiennej losowej?
\ans{
	Żeby funkcja była funkcją gęstości prawdopodobieństwa pewnej zmiennej losowej muszą być jednocześnie spełnione dwa warunki:
	\begin{enumerate}
		\item Całka oznaczona od $-\infty$ do $\infty$ z tej funkcji jest równa 1:
		\[ \int_{-\infty}^\infty f(x)\,dx=1\]
		\item Funkcja jest nieujemna dla dowolnego argumentu:
		\[ \forall x\in\mathbb{R}\colon f(x)\geq 0\]
	\end{enumerate}
	Zaczniemy od pierwszego warunku i będziemy szukali wartości $a$, dla której jest on spełniony:
	\[ \int_{-\infty}^\infty f(x)\,dx = \int_{-\infty}^0 0\,dx + \int_0^3 ax^2\,dx + \int_{3}^\infty 0\,dx =  \int_0^3 ax^2\,dx\]
	Obliczymy całkę nieoznaczoną z $ax^2$:
	\[ \int ax^2\, dx = a\int x^2\, dx = \frac{a}{3}x^3 + C \]
	Podstawiamy do całki oznaczonej:
	\[ \int_0^3 ax^2\,dx = \left. \frac{a}{3}x^3\right|_0^3 = \frac{a}{3}(3^3-0^3)=9a \]
 	W takim razie zostajemy z równaniem
	\[ 9a=1 \]
	I dochodzimy do
	\[ a=\frac{1}{9} \]
	Drugi warunek jest spełniony, bo ${x^2}{9}\geq 0$ dla dowolnej wartości $x\in\left<0,3\right>$, a $0\geq 0$.
	W takim razie odpowiedź brzmi: $a=\frac{1}{9}$.
}
\item Ile wynosi wartość średnia tej zmiennej losowej?
\ans{
	Posługujemy się definicją wartości średniej: \[ EX=\int_{-\infty}^\infty xf(x)\,dx \]
	Podstawiamy funkcję gęstości prawdopodobieństwa z treści zadania:
	\[ EX=\int_{-\infty}^0 x\cdot 0\, dx + \int_0^3 x\cdot \frac{x^2}{9}\, dx + \int_3^\infty x\cdot 0\, dx = \int_0^3 \frac{x^3}{9}\, dx \]
	Obliczamy całkę nieoznaczoną z $x^3$:
	\[ \int x^3\, dx=\frac{x^4}{4} +C \]
	Wracamy do całki oznaczonej:
	\[ \int_0^3 \frac{x^3}{9}\, dx = \frac{1}{9}\int_0^3 x^3\, dx = \frac{1}{9}\cdot \left.\frac{x^4}{4}\right|_0^3 = \frac{3^4-0^4}{9\cdot 4} = \frac{9}{4} \] 
	Zatem wartość średnia $EX$ zmiennej losowej $X$ wynosi $\frac{9}{4}$
}
\item Ile wynosi wariancja tej zmiennej losowej?
\ans{
	Rozwiążemy zadanie na dwa sposoby. Sposób 1, bezpośrednio z definicji wariancji $D^2X$:
	\[ D^2X = E\left[(X-EX)^2\right] = \int_{-\infty}^\infty (x-EX)^2 f(x)\, dx \]
	Z poprzedniego podpunktu wiemy, że $EX=\frac{9}{4}$. Dodatkowo, ponieważ funkcja gęstości prawdopodobieństwa jest niezerowa tylko w przedziale $\left(0, 3\right>$, iloczyn pod całką również tylko wtedy może być niezerowy. Zatem:
	\[ \int_{-\infty}^\infty (x-EX)^2 f(x)\, dx = \int_0^3 \left(x-\frac{9}{4}\right)^2 \cdot \frac{x^2}{9}\, dx \]
	Wyciągamy stałą $\frac{1}{9}$ przed całkę, a pod całką wykonujemy mnożenie:
	\[ \int_0^3 \left(x-\frac{9}{4}\right)^2 \cdot \frac{x^2}{9}\, dx = \frac{1}{9} \int_0^3 \left[ x^4-2\cdot\frac{9}{4}\cdot x^3 +\left(\frac{9}{4} \right)^2 x^2\right]\, dx
	\]
	Posługujemy się rodziną całek nieoznaczonych dla jednomianów o wykładniku $a>0$: \[ \int x^a\, dx = \frac{x^{a+1}}{a+1} + C \]
	i uzyskujemy:
	\[ \frac{1}{9} \int_0^3 \left[ x^4-2\cdot\frac{9}{4}\cdot x^3 +\left(\frac{9}{4} \right)^2 x^2\right]\, dx = \frac{1}{9} \left[ \frac{x^5}{5}-\frac{9}{2}\cdot\frac{x^4}{4} + \frac{3^4}{4^2}\cdot\frac{x^3}{3} \right]_0^3 = \frac{1}{9} \left[ \frac{3^5}{5} - \frac{3^6}{8} + \frac{3^6}{16} \right] = \frac{3^3}{5} -\frac{3^4}{16} = \frac{27}{80}  \]
	\par
	Sposób 2, posługując się przekształconym wzorem na wariancję:
	\[ D^2X = E\left(X^2\right) - \left(EX\right)^2 \]
	Obliczamy $EX^2$: 
	\[ E\left(X^2\right) = \int_{-\infty}^\infty x^2f(x)\,dx \]
	Wiemy, że iloczny pod całką jest niezerowy tylko tam, gdzie funkcja gęstości jest niezerowa, a zatem:
	\[ \int_{-\infty}^\infty x^2f(x)\,dx = \int_0^3 x^2\cdot \frac{x^2}{9}\,dx = \frac{1}{9} \int_0^3 x^4\,dx = \frac{1}{9}\cdot \frac{3^5-0^5}{5} = \frac{3^3}{5} \]
	Wracam do wzoru na wariancję pamiętając, że $EX=\frac{9}{4}$ z poprzedniego podpunktu w zadaniu:
	\[ D^2X = E\left(X^2\right) - \left(EX\right)^2 = \frac{3^3}{5} - \left(\frac{9}{4}\right)^2 =  \frac{3^3}{5} -\frac{3^4}{16} = \frac{27}{80} \]
	Czyli otrzymaliśmy dokładnie to samo co sposobem 1, ale szybciej.
}
\end{enumerate}
%\item Rozważmy funkcję \[f(x)=\begin{cases} 0 & x<0 \\ e^{-x} & x\geq 0\end{cases} \]
%\begin{enumerate}
%\item Udowodnij, że jest to funkcja gęstości prawdopodobieństwa pewnej zmiennej losowej $X$. \ans{nieujemna, $\int_{-\infty}^\infty f(x)dx=0+\int_0^\infty f(x)dx=1$}
%\item Oblicz dystrybuantę zmiennej losowej $X$ i narysuj jej wykres.
%\item Oblicz $P(X<\frac{1}{2})$
%\item Oblicz $P(1\leq X<2)$
%\item Oblicz wartość średnią
%\end{enumerate}
\item Autobusy linii 164 przyjeżdżają punktualnie co 20 minut na przystanek. Olga nie zna ich rozkładu jazdy, przychodzi więc na przystanek w całkowicie losowym momencie: jak jej się wyjdzie z domu. Niech $T$ oznacza czas oczekiwania Olgi na autobus.
\begin{enumerate}
\item Podaj rozkład zmiennej losowej $T$. \ans{
	Z tego, że Olga nie zna rozkładu jazdy wnioskujemy, że czas oczekiwania można modelować jako zmienną losową typu ciągłego o \emph{jednostajnym} rozkładzie prawdopodobieństwa, to znaczy o funkcji gęstości prawdopodobieństwa danej poniższym wzorem dla pewnych parametrów $a<b$:
	\[ f(t)=\begin{cases}\frac{1}{b-a} & a\leq t\leq b \\ 0 & \text{w przeciwnym przypadku} \end{cases} \]
	Wiemy, że Olga może czekać co najmniej 0 minut (przychodzi równo z autobusem), a co najwyżej 20 minut (autobus jeździ punktualnie i co 20 minut), wnioskujemy zatem, że $a=0$, $b=20$.	
	W takim razie możemy zapisać kompletną funkcję gęstości prawdopodobieństwa zmiennej losowej $T$:
		\[ f(t)=\begin{cases}\frac{1}{20} & 0\leq t\leq 20 \\ 0 & \text{w przeciwnym przypadku} \end{cases} \]
}
\item Oblicz prawdopodobieństwo, że Olga będzie czekała mniej niż 5 minut. \ans{
Podejdziemy do problemu na dwa sposoby. 
\paragraph{Sposób 1} posłużymy się wzorem, który pozwala obliczyć prawdopodobieństwo, że zmienna losowa przyjmie wartość z przedziału $\left<c, d\right>$ korzystając z funkcji gęstości prawdopodobieństwa:
\[ P(c\leq X \leq d)=\int_c^d f(x)\, dx \]
Z treści zadania wynika, że szukamy $P(X<5)$, czyli: $c=-\infty$, a $d=5$. Różnicę pomiędzy nierównością nieostrą $\leq$, a ostrą $<$ można zaniedbać w przypadku zmiennych losowych typu ciągłego, ponieważ prawdopodobieństwa w punkcie są równe 0.
W takim razie:
\[ P(X<5) = P(-\infty \leq X \leq 5) = \int_{-\infty}^5 f(x)\, dx \]
Ale wiemy, że $f(x)$ jest niezerowe tylko w przedziale $\left<0,20\right>$, zatem możemy zacząć całkowanie w $0$:
\[ P(X<5) = P(0 \leq X \leq 5) = \int_0^5 f(x)\, dx = \int_0^5 \frac{1}{20}\, dx = \left.\frac{x}{20}\right|^5_0=\frac{1}{4} \]
\paragraph{Sposób 2} Zaczniemy od obliczenia \emph{dystrybuanty}, czyli funkcji $F$ zmiennej rzeczywistej $u$ danej poniższym wzorem:
\[ F(u) = P(X\leq u) = \int_{-\infty}^u f(x)\, dx \]
Dla $u<0$ sprawa jest prosta: pod całką jest 0, zatem dystrybuanta też jest $0$.
Dla $u>20$ sprawa też jest prosta: na pewno Olga czeka nie więcej niż $u$ minut (bo $u>20$, a Olga czeka najwyżej 20 minut), zatem dystrybuanta musi wynosić $1$.
Pozostaje obszar $u\in\left<0,20\right>$:
\[ F(u) = \int_{-\infty}^0 0\, dx + \int_0^u \frac{1}{20}\, dx = \frac{u}{20} \]
Podsumowując, otrzymujemy następującą funkcję sklejaną:
\[ F(u) = P(T\leq u) = \begin{cases} 0 & u<0 \\ \frac{u}{20} & u\in\left<0,20\right> \\ 1 & u>20 \end{cases} \]
W takim razie podstawiając $u=5$ otrzymujemy:
\[ F(5) = P(T\leq 5) = \frac{5}{20}=\frac{1}{4} \]
}
\item Oblicz średni czas oczekiwania na autobus. 
\begin{ansenv}
	Oszukamy i zajrzymy do Wikipedii: \url{https://pl.wikipedia.org/w/index.php?title=Rozk%C5%82ad_jednostajny_ci%C4%85g%C5%82y&oldid=56001777}
	gdzie znajdziemy gotowy wzór na wartość średnią (oczekiwaną):
	\[ ET = \frac{a+b}{2} \]
	W takim razie: \[ ET = \frac{0+20}{2} = 10 \]
	
	Możemy też wzór wyprowadzić podobnie jak w zadaniu 1:
	\[ ET = \int_{-\infty}^\infty xf(x)\, dx \]
	Wiemy, że iloczyn pod całką będzie zerowy dla $x\not\in\left<a,b\right>$, zatem:
	\[ ET = \int_{-\infty}^\infty xf(x)\, dx = \int_a^b \frac{x}{b-a}\, dx = \left.\frac{x^2}{2(b-a)}\right|_a^b = \frac{b^2-a^2}{2(b-a)} \]
	Ze wzoru skróconego mnożenia:
	\[ ET = \frac{b^2-a^2}{2(b-a)} = \frac{(b-a)(b+a)}{2(b-a)} = \frac{b+a}{2} \]
	Co kończy wyprowadzenie.
\end{ansenv}
\item Oblicz odchylenie standardowe zmiennej losowej $T$. 
\begin{ansenv}
	Korzystamy z takiego samego oszustwa jak poprzednio i odkrywamy, że wariancja $D^2T$ dana jest wzorem:
	\[ D^2T=\frac{(b-a)^2}{12} \]
	a zatem odchylenie standardowe wynosi:
	\[ DT = \sqrt{\frac{(b-a)^2}{12}} = \frac{b-a}{\sqrt{12}} = \frac{20}{\sqrt{12}} \approx 5{,}77 \]
	
	Żeby wyprowadzić wzór skorzystamy z przekształconego wzoru jak w poprzednim zadaniu:
	\[ D^2T = E\left(T^2\right) - \left(ET\right)^2 \]
	Odjemnik znamy z poprzedniego punktu, zatem potrzebujemy obliczyć odjemną:
	\[ E\left(T^2\right) = \int_{-\infty}^\infty t^2f(t)\,dt \]
	Iloczyn pod całką jest niezerowy najwyżej dla $t\in\left<a,b\right>$, zatem:
	\[ E\left(T^2\right) = \int_{-\infty}^\infty t^2f(t)\,dt = \int_a^b \frac{t^2}{b-a} = \left.\frac{t^3}{3(b-a)}\right|_a^b = \frac{b^3-a^3}{3(b-a)} \]
	Ze wzoru skróconego mnożenia:
	\[ E\left(T^2\right) = \frac{b^3-a^3}{3(b-a)} = \frac{(b-a)(a^2+ab+a^2)}{3(b-a)} = \frac{a^2+ab+b^2}{3} \]
	Wracamy do wzoru na wariancję:
	\[ D^2T = E\left(T^2\right) - \left(ET\right)^2 = \frac{a^2+ab+b^2}{3} - \left(\frac{a+b}{2}\right)^2 = \frac{a^2+ab+b^2}{3} - \frac{a^2+2ab+b^2}{4} = \frac{a^2-2ab+b^2}{12} = \frac{(a-b)^2}{12} \]
\end{ansenv}
\end{enumerate}
\item Prawdopodobieństwo, że aparat fotograficzny nie zepsuje się w ciągu pierwszych pięciu miesięcy użytkowania wynosi $0{,}9$. Niech $X$ oznacza liczbę miesięcy bezawaryjnej pracy aparatu. Zakładamy, że zmienna $X$ ma rozkład wykładniczy.
\begin{enumerate}
\item Oblicz parametry tego rozkładu. 
\begin{ansenv}
	Zaczniemy od analizy zadania. Wiemy, że zmienna losowa reprezentująca czas \emph{bezawaryjnej pracy} ma rozkład wykładniczy. Czas bezawaryjnej pracy, to innymi słowy czas do pierwszej awarii, czyli do zepsucia się aparatu.
	W takim razie \emph{Prawdopodobieństwo, że aparat fotograficzny nie zepsuje się w ciągu pierwszych pięciu miesięcy} możemy rozumieć jako \emph{Prawdopodobieństwo, że do pierwszej awarii upłynie więcej niż pięć miesięcy}.
	W takim razie uzyskujemy: $P(X>5)=0{,}9$.
	Z treści zadania wiemy, że zmienna losowa ma rozkład wykładniczy, czyli opisana jest funkcją gęstości prawdopodobieństwa:
	\[ f(x) = \begin{cases} \lambda e^{-\lambda x} & x\geq 0 \\ 0 & \text{w przeciwnym przypadku} \end{cases} \]
	gdzie $\lambda$ jest poszukiwanym parametrem.
	
	Wyprowadzić wzór na dystrybuantę jest prosto (patrz niżej), ale na razie oszukamy i zajrzymy do Wikipedii: \url{https://pl.wikipedia.org/w/index.php?title=Rozk%C5%82ad_wyk%C5%82adniczy&oldid=59056371}
	\[ F(x) = P(X\leq x) = \begin{cases} 1-e^{-\lambda x} & x\geq 0 \\ 0 & \text{w przeciwnym przypadku} \end{cases} \]
	
	Skorzystamy ze zdarzenia przeciwnego do $X>5$:
	\[ P(X>5)=1-P(X\leq 5)=1-F(5) = 1-\left(1-e^{-\lambda\cdot 5}\right) = e^{-5\lambda} = 0{,}9 \]
	Rozwiązujemy równanie ze względu na zmienną $\lambda$ zaczynając od logarytmowania stronami:
	\[ \lambda = -\frac{\ln(0{,}9)}{5} \approx 0{,}021 \]
	
	Wyprowadzenie dystrybuanty (dla $u\geq 0$):
	\[ F(u) = \int_{-\infty}^u f(x)\, dx \]
	Wiemy, że dla $x\in(-\infty,0)$ pod całką jest 0, zatem:
	\[ F(u) = \int_0^u \lambda e^{-\lambda x}\, dx \]
	Obliczamy całkę nieoznaczoną stosując podstawienie $z=-x\lambda$, wtedy $dz=-\lambda dx$, a zatem $dx=\frac{-dz}{\lambda}$
	\[ \int \lambda e^{-x\lambda}\, dx = \int \lambda e^z\cdot\frac{-dz}{\lambda} = -\int e^z\, dz = -e^z + C = -e^{-x\lambda} +C \]
	Wracamy do całki nieoznaczonej:
	\[ F(u) = \int_0^u \lambda e^{-\lambda x}\, dx = \left. -e^{-x\lambda} \right|_0^u = -e^{-u\lambda} - \left(-e^{0\lambda}\right) = 1-e^{-u\lambda} \]
	Co kończy wyprowadzenie.
\end{ansenv}
\item Oblicz prawdopodobieństwo bezawaryjnej pracy aparatu w ciągu 24 miesięcy.
\begin{ansenv}
	$X$ to liczba miesięcy bezawaryjnej pracy, a zatem pytanie jest o $P(X>24)$.
	Korzystamy znowu ze zdarzenia przeciwnego i wzoru na dystrybuantę:
	\[ P(X>24) = 1-P(X\leq 24)=1-F(24) = e^{-\lambda\cdot 24} = e^{-\lambda = -\frac{\ln(0{,}9)}{5}\cdot 24} \]
	Korzystając z własności logarytmu naturalnego: $e^{\ln(x)} = x$ otrzymujemy:
	\[ P(X>24) = \left(0{,}9\right)^\frac{24}{5} \approx 0{,}603 \]
\end{ansenv}
\item Oblicz prawdopodobieństwo awarii aparatu w ciągu 36 miesięcy, jeżeli wiadomo, że przepracował bez awarii już 12 miesięcy. %\ans{$P(X<36|X>12)=P(X<36-12)=F(24)$}
\begin{ansenv}
	Skoro pojawia się słowo \emph{jeżeli}, to mamy do czynienia z prawdopodobieństwem warunkowym $P(X\leq 36|X>12)$. Korzystamy ze zdarzenia przeciwnego i definicji prawdopodobieństwa warunkowego:
	\[ P(X\leq 36|X>12) = 1-P(X>36|X>12) = 1-\frac{P(X>36 \land X>12)}{P(X>12)} = 1-\frac{P(X>36)}{P(X>12)} = 1-\frac{1-F(36)}{1-F(12)} \]
	Korzystając z definicji dystrybuanty i własności potęgowania:
	\[ 1-\frac{1-F(36)}{1-F(12)} = 1-\frac{e^{-36\lambda}}{e^{-12\lambda}} = 1-e^{-36\lambda-(-12\lambda)} = 1-e^{-24\lambda} = F(24) \approx 0{,}397 \]
	
	Powyższe można uogólnić do \emph{własności braku pamięci} (dla rozkładów typu ciągłego działa tylko w rozkładzie wykładniczym!):
	\[ \forall a,b>0\colon P(X>a+b|X>b) = P(X>a) \]
	W naszym przypadku $a=24$, a $b=12$ i już od $1-P(X>36|X>12)$ mogliśmy przejść do $1-P(X>24)=F(24)$.
\end{ansenv}
\end{enumerate}
\item W Minas Tirith gromadzą zapasy na wypadek oblężenia. Losowo wybrany worek zawiera $X$ kg mąki, gdzie $X$ jest zmienną losową o rozkładzie $N(20,2^2)$.
\begin{enumerate}
\item Utwórz zmienną losową $Y$ będącą standaryzowaną postacią zmiennej losowej $X$.
\begin{ansenv}
	$N(20, 2^2)$ należy rozumieć jako \emph{rozkład normalny o wartości średniej $\mu=20$ i wariancji $\sigma^2=2^2$}.
	Standaryzacja zawsze polega na odjęciu wartości średniej i podzieleniu przez odchylenie standardowe. Zatem zmienna $Y$ będąca standaryzowaną postacią zmiennej losowej $X$:
	\[ Y=\frac{X-20}{2} \]
\end{ansenv}
\item Wyraź dystrybuantę zmiennej $X$ w zależności od dystrybuanty zmiennej $Y$. \ans{$F_X(x)=F_Y(\frac{x-20}{2})$}
\item Oblicz wartość średnią i odchylenie standardowe zmiennej losowej $Y$.
\begin{ansenv}
	Jeżeli mamy do czynienia ze standaryzowaną zmienną losową to zawsze $EY=0$ i $DY=1$. Dlaczego?
	
	\[ EY=E\left[\frac{X-20}{2}\right] = \frac{E(X-20)}{2} = \frac{EX-20}{2}=\frac{20-20}{2} = 0 \]
	
	\[ D^2Y=D^2\left[\frac{X-20}{2}\right] = \frac{D^2(X-20)}{2^2} = \frac{D^2X}{2^2} =\frac{2^2}{2^2} = 1\]
	
	Posłużyliśmy się przy tym następującymi własnościami wartości średniej i wariancji, prawdziwymi dla dowolnej wartości $a\in\mathbb{R}$:
	\begin{gather*}
	E(aX) = aEX \\
	D^2(aX) = a^2D^2X \\
	E(X-a) = EX-a \\
	D^2(X-a) = D^2X \\
	\end{gather*}
\end{ansenv}
\item Oblicz prawdopodobieństwo, że w worku jest mniej niż 18 kg mąki.
\begin{ansenv}
	Do obliczania prawdopodobieństw będziemy posługiwać się tablicą zawierającą wartości dystrybuanty standaryzowanego rozkładu normalnego. Taka tablica jest dostępna na odwrocie kartki z ćwiczeniami, na egzaminie, jest też łatwa do znalezienia w Internecie.
	Zaczynamy od sprowadzenia polecenia do dystrybuanty zmiennej losowej $Y$ o standaryzowanym rozkładzie normalnym $N(0,1)$:
	\[ P(X<18) = F_X(18) = F_Y\left(\frac{18-20}{2}\right) = F_Y(-1) \]
	Następnie korzystamy z własności dystrybuanty standaryzowanego rozkładu normalnego:
	\[ F_Y(u) = 1-F_Y(-u) \]
	I w takim razie:
	\[ P(X<18) = F_Y(-1)=1-F_Y(1) \]
	$1$ możemy zapisać jako $1{,}00$ -- w tabeli odnajdujemy wiersz podpisany $1{,}0$ i kolumnę odpowiadającą cyfrze setek, więc $0$, a następnie odczytujemy wartość na przecięciu:
	\[ F_Y(1) = 0{,}8413 \]
	W takim razie:
	\[P(X<18)=F_X(18)=F_Y(-1)=1-F_Y(1)=1-0{,}8413\approx 0{,}16\]
\end{ansenv}
\item Oblicz prawdopodobieństwo, że w worku jest więcej niż $21{,}5$ kg mąki.
\begin{ansenv}
	Korzystamy ze zdarzenia przeciwnego, żeby przejść do dystrybuanty i sprowadzamy do dystrybuanty standaryzowanego rozkładu normalnego:
	\[ P(X>21{,}5)=1-F_X(21{,}5)=1-F_Y\left(\frac{21{,}5-20}{2}\right)=1-F_Y(0{,}75) \]
	W tabeli odnajdujemy wiersz podpisany $0{,}7$ i kolumnę podpisaną $0{,}05$:
	\[ F_Y(0{,}75)=0{,}7734 \]
	Wracamy do poprzedniej równości:
	\[ P(X>21{,}5)=1-F_Y(0{,}75) =1-0{,}7734\approx 0{,}23 \]
\end{ansenv}
\item Oblicz prawdopodobieństwo, że ilość mąki w worku różni się od wartości oczekiwanej o nie więcej niż 1~kg.
\begin{ansenv}
	Wartość oczekiwana (średnia) zmiennej losowej $X$ to 20, w takim razie pytanie jest o $X$ pomiędzy 19, a 21 kg. Potem postępujemy jak wcześniej:
	\begin{gather*}
	 P(\left|X-20\right|<1) = P(19<X<21) = F_X(21) - F_X(19) = F_Y\left(\frac{21-20}{2}\right)-F_Y\left(\frac{19-20}{2}\right) = \\ 
	 F_Y(0{,}5) - F_Y(-0{,}5) = F_Y(0{,}5) - (1-F_Y(0{,}5)) = 2F_Y(0{,}5) - 1
	\end{gather*}
	Z tablicy standaryzowanego rozkładu normalnego (wiersz $0{,}5$, kolumna $0$) odczytujemy:
	\[ F_Y(0{,}5)=0{,}6915 \]
	Zatem:
	\[ P(\left|X-20\right|<1) = 2F_Y(0{,}5) - 1 = 2\cdot0{,}6915-1=0{,}38 \]
\end{ansenv}
% \ans{$P(19<X<21)=F_X(21)-F_X(19)=F_Y(0{,}5)-F_Y(-0{,}5)=2F_Y(0{,}5)-1=2\cdot0{,}6915-1=0{,}38$}
\end{enumerate}
%\item Pociągi Kolei Wielkopolskich odjeżdżają w~kierunku Wągrowca co 60 min. Niestety, nastąpiła zmiana rozkładu, strona
%internetowa padła, telefony nie działają, a~plakaty z~rozkładem porwał wiatr. Zakładając, że rozkład czasu przybycia pasażera na stację jest
%jednostajny, a pociągi jeżdżą punktualnie, obliczyć:
%\begin{enumerate}
%	\item prawdopodobieństwo, że czas oczekiwania na pociąg przekroczy 10 min, ale będzie nie większe niż 55 min;
%	\item średni czas oczekiwania na pociąg.
%\end{enumerate}
%\item Zakładając, że wysokość losowo wybranego psa rasy Husky pochodzi z~rozkładu normalnego o średniej $53{,}5$ cm i odchyleniu standardowym $1{,}5$ cm (inaczej: $N(53{,}5;1{,}5)$), oblicz jakie jest prawdodpodobieństwo, że taki pies ma wysokość powyżej $56$ cm.
%\item Taczki wyprodukowane w zakładzie Hamfasta psują się średnio po trzech latach użytkowania.
%Fredegar zakupił jedną z taczek, niech $X$ będzie zmienną losową o rozkładzie wykładniczym, reprezentującą czas w~latach bezawaryjnej pracy taczki Fredegara.
%
%\begin{enumerate}
%\item Podaj wartość średnią i odchylenie standardowe zmiennej losowej $X$.
%\item Podaj dystrybuantę zmiennej losowej $X$.
%\item Oblicz prawdopodobieństwo, że taczka zepsuje się przed upływem dwóch lat. 
%\item Oblicz prawdopodobieństwo, że taczka będzie pracowała bez awarii co najmniej 6 lat, jeżeli wiadomo, że jest używana już od trzech lat.
%\end{enumerate}
\item W hobbickiej wsi Oatbarton znajduje się wielki spichlerz na owies, w~którym okoliczni farmerzy gromadzą swoje plony.
Masa w tonach $X$ owsa zgromadzonego w spichlerzu bezpośrednio po zbiorach ma rozkład normalny $N(100, 20^2)$.
Spichlerz ma pojemność 130 ton, a~roczne potrzeby wsi wynoszą 85 ton.

W poniższych zadaniach wyniki podawaj z dokładnością do przynajmniej dwóch miejsc po przecinku.
\begin{enumerate}
\item Podaj $EX$ oraz $DX$.
\ans{$EX=100 \quad DX=20$}
\item Niech $Y$ będzie standaryzowaną postacią zmiennej losowej $X$. Podaj $EY$ oraz $DY$.
\ans{$EY=0 \quad DY=1$}
\item Oblicz prawdopodobieństwo, że hobbici wyprodukują dokładnie 100 ton owsa.
\ans{$P(X=100)=0$, bo dla zmiennych losowych typu ciągłego prawdopodobieństwo w punkcie jest zawsze równe 0}
\item Oblicz prawdopodobieństwo, że hobbici wychodują najwyżej tyle owsa ile są w stanie zmagazynować.
\ans{$P(X<130)=F_X(130)=F_Y(1{,}5)=0{,}933$}
\item Oblicz prawdopodobieństwo, że hobbici nie wyprodukują dostatecznie dużo owsa i będą musieli go dokupić.
\ans{$P(X<85)=F_X(85)=F_Y(-0{,}75)=1-F_Y(0{,}75)=1-0{,}773=0{,}227$}
\item Oblicz prawdopodobieństwo, że masa owsa wyprodukowanego przez hobbitów będzie się różnić od wartości oczekiwanej o nie więcej niż pół odchylenia standardowego.
\ans{\begin{gather*}
	P\left(\left|X-EX\right|\leq \frac{DX}{2}\right) = P(\left|X-100\right|\leq 10) = P(90\leq X\leq 110) = F_X(110)-F_X(90) =\\
	F_Y(0{,}5)-F_Y(-0{,}5) = F_Y(0{,}5)-\left(1-F_Y(0{,}5)\right) = 2F_Y(0{,}5)-1 = 2\cdot0{,}6915-1=0{,}38
	\end{gather*}
}
\item Oblicz prawdopodobieństwo, że masa owsa wyprodukowanego przez hobbitów będzie się różnić od wartości oczekiwanej o nie więcej niż dwie wariancje.
\ans{\begin{gather*}
	P(\left|X-EX\right|<2D^2X) = P(\left|X-100\right|<800) = P(-700<X<900) = F_X(900)-F_X(-700)=2F_Y(40)-1=1
	\end{gather*}	
}
\end{enumerate}

\item W Bucklandzie na jesieni przygotowują sery na zimę. W każdym ze 100 magazynów
zgromadzonych jest po 1000 serów. Prawdopodobieństwo zepsucia się pojedynczego
krążka sera wynosi $4\cdot 10^{-3}$ i~nie zależy od pozostałych serów. Niech
$i=1,2,\ldots,100$ oznacza numer magazynu i niech zmienna losowa $X_i$ oznacza
liczbę zepsutych serów składowanych w $i$-tym magazynie. Ponadto, niech zmienna
losowa $Y=X_1+X_2+\ldots+X_{100}$ odpowiada sumarycznej liczbie zepsutych serów
we wszystkich magazynach. Załóż, że:
\begin{itemize}
\item wszystkie zmienne $X_i$ mają taki sam rozkład;
\item zmienne $X_i$ są niezależne;
\item zmienna $Y$ ma rozkład normalny o wartości średniej 100 razy większej niż wartość średnia dowolnej ze zmiennych $X_i$;
\item zmienna $Y$ ma rozkład normalny o wariancji 100 razy większej niż wariancja dowolnej ze zmiennych $X_i$.
\end{itemize}

\begin{enumerate}
\item Podaj  (w formie funkcji prawdopodobieństwa) rozkład prawdopodobieństwa zmiennej losowej $X_1$.
\item Oblicz (z dokładnością do dwóch miejsc po przecinku) prawdopodobieństwo, że w magazynie nr 10 zepsują się mniej niż trzy sery.
\item Podaj średnią liczbę zepsutych serów w magazynie nr 15.
\item Podaj wartości $EY$ oraz $DY$.
\item Utwórz zmienną losową $Z$ będącą standaryzowaną postacią zmiennej losowej $Y$.
\item Oblicz (z dokładnością do dwóch miejsc po przecinku) prawdopodobieństwo, że w Bucklandzie zepsują się więcej niż 424 sery.
\end{enumerate}

\item Ted Cotton, jeden z pracowników urzędu pocztowego w Hobbitonie, poczynił następującą obserwację: w 48 przypadkach na 50 po wyjściu klienta z urzędu następny klient przyjdzie do urzędu przed upływem 2 minut.
Co więcej, jeżeli przez 2 minuty nikt nie przyjdzie, to sytuacja się powtarza: w 48 przypadkach na 50 przez kolejne 2 minuty przyjdzie klient itd.
Z urzędu pocztowego właśnie wyszedł klient.
Niech $T$ będzie zmienną losową \emph{typu ciągłego} charakteryzującą się \emph{brakiem pamięci}, a~oznaczającą czas w minutach oczekiwania na przyjście następnego klienta.

\begin{enumerate}
\item Podaj rozkład prawdopodobieństwa zmiennej losowej $T$: jego nazwę, parametry i~dystrybuatnę.
\ans
{
Rozkład wykładniczy
\begin{gather*}
F(x) = \begin{cases} 1-e^{-\lambda x} & x>0 \\ 0 & \text{wpp} \end{cases} \\
P(T<2)=F(2)=\frac{48}{50} \\
1-e^{-2\lambda}=\frac{24}{25} \\
e^{-2\lambda}=\frac{1}{25} \\ 
-2\lambda = \ln \frac{1}{25} = -2 \ln 5 \\
\lambda = \ln 5 \approx 1{,}609
\end{gather*}
}
\item Podaj średni czas oczekiwania na kolejnego klienta.
\ans{ $ET = \frac{1}{\lambda} = \frac{1}{\ln 5} \approx 0{,}621$ }
\item Podaj odchylenie standardowe zmiennej losowej $T$.
\ans{$DT=ET=\frac{1}{\lambda} = \frac{1}{\ln 5} \approx 0{,}621$ }
\item Jakie jest prawdopodobieństwo, że czas oczekiwania na klienta będzie pomiędzy 1, a 3 minut?
\ans{$P(1\leq T\leq 3) = F(3)-F(1) = 1-e^{-3\lambda}-(1-e^{-\lambda}) = e^{-\lambda}-e^{-3\lambda}=5^{-1}-5^{-3}=\frac{5^2-1}{5^3}=\frac{24}{125}=0{,}192$}
\item Ted twierdzi, że kiedyś przez godzinę nikt nie przyszedł do urzędu pocztowego. Jakie jest prawdopodobieństwo takiego zdarzenia? Jakie są jego możliwe przyczyny?
\ans{$P(T>60)=1-F(60)=1-(1-e^{-60\lambda}) = 5^{-60} \approx 0$}
\end{enumerate}

{\small
\begin{tabular}{|r|r|r||r|r|r|}
\hline
\multicolumn{6}{|c|}{Rozkład normalny} \\
	& \textbf{0}	& \textbf{0,05}	& 	& \textbf{0}	& \textbf{0,05}\\
\hline
\textbf{0,0}	& 0,500	& 0,520	& \textbf{1,6}	& 0,945	& 0,951 \\
\hline
\textbf{0,1}	& 0,540	& 0,560	& \textbf{1,7}	& 0,955	& 0,960 \\
\hline
\textbf{0,2}	& 0,579	& 0,599	& \textbf{1,8}	& 0,964	& 0,968 \\
\hline
\textbf{0,3}	& 0,618	& 0,637	& \textbf{1,9}	& 0,971	& 0,974 \\
\hline
\textbf{0,4}	& 0,655	& 0,674	& \textbf{2,0}	& 0,977	& 0,980 \\
\hline
\textbf{0,5}	& 0,691	& 0,709	& \textbf{2,1}	& 0,982	& 0,984 \\
\hline
\textbf{0,6}	& 0,726	& 0,742	& \textbf{2,2}	& 0,986	& 0,988 \\
\hline
\textbf{0,7}	& 0,758	& 0,773	& \textbf{2,3}	& 0,989	& 0,991 \\
\hline
\textbf{0,8}	& 0,788	& 0,802	& \textbf{2,4}	& 0,992	& 0,993 \\
\hline
\textbf{0,9}	& 0,816	& 0,829	& \textbf{2,5}	& 0,994	& 0,995 \\
\hline
\textbf{1,0}	& 0,841	& 0,853	& \textbf{2,6}	& 0,995	& 0,996 \\
\hline
\textbf{1,1}	& 0,864	& 0,875	& \textbf{2,7}	& 0,997	& 0,997 \\
\hline
\textbf{1,2}	& 0,885	& 0,894	& \textbf{2,8}	& 0,997	& 0,998 \\
\hline
\textbf{1,3}	& 0,903	& 0,911	& \textbf{2,9}	& 0,998	& 0,998 \\
\hline
\textbf{1,4}	& 0,919	& 0,926	& \textbf{3,0}	& 0,999	& 0,999 \\
\hline
\textbf{1,5}	& 0,933	& 0,939	& \textbf{3,1}	& 0,999	& 0,999 \\
\hline
\end{tabular}
}
\end{enumerate}

\clearpage
\section{Zmienne losowe dwuwymiarowe}
\begin{enumerate}
\item Z talii 52 kart wylosowano jedną kartę. Niech zmienna losowa $X$ przyjmuje wartość odpowiadającą liczbie wylosowanych waletów, zaś $Y$ odpowiadającą liczbie wylosowanych trefli.
\begin{enumerate}
\item Wyznacz rozkłady zmiennych losowych $X$ oraz $Y$. 
\begin{ansenv}
	Postępujemy tak samo przy zadaniach o zmiennych losowych jednowymiarowych typu skokowego. Obie zmienne mogą przyjmować wyłącznie wartości ze zbioru $\{0, 1\}$, zatem:
	\[ P(X=1) = P(\text{wylosowanie waleta}) = \frac{4}{52}=\frac{1}{13} \qquad P(X=0)=1-P(X=1)=\frac{48}{52}=\frac{12}{13} \]
	\[ P(Y=1) = P(\text{wylosowanie trefla}) = \frac{13}{52}=\frac{1}{4} \qquad P(Y=0)=1-P(Y=1)=\frac{39}{52}=\frac{3}{4} \]
\end{ansenv}
\item Wyznacz rozkład zmiennej losowej $(X,Y)$. 
\begin{ansenv}
	Musimy rozważyć wszystkie możliwe pary punków skokowych zmiennej losowej $(X,Y)$: $\{0,1\}\times\{0,1\} = \{(0,0), (0, 1), (1, 0), (1,1)\}$, za każdym razem rozważając jednocześnie oba warunki wyznaczone przez zmienne losowe.
	Rozważania dotyczą tego samego zdarzenia elementarnego, tzn. $X=1, Y=1$ odpowiada sytuacji, w której \emph{ta sama wylosowana karta} okazuje się waletem ($X=1$) oraz treflem ($Y=1$).
	\[ P(X=1, Y=1) = P(\text{wylosowanie waleta i wylosowanie trefla}) = P(\text{wylosowanie waleta trefl}) = \frac{1}{52} \]
	\begin{align*}
	P(X=1, Y=0) = & P(\text{wylosowanie waleta i wylosowanie nie-trefla}) = \\ & P(\text{wylosowanie waleta pik, kier lub karo}) = \frac{3}{52}
	\end{align*}
	\begin{align*}
	P(X=0, Y=1) = & P(\text{wylosowanie nie-waleta i wylosowanie trefla}) = \\ & P(\text{wylosowanie trefla niebędącego waletem}) = \frac{12}{52} 
	\end{align*}
	\begin{align*}
	 P(X=0, Y=0) = & P(\text{wylosowanie nie-waleta i wylosowanie nie-trefla}) = \\  & P(\text{dowolna z 13 kart poza waletem z kolorów pik, kier, karo}) = \frac{3\cdot (13-1)}{52}=\frac{36}{52}
	 \end{align*}
	 Wygodnie jest zapisać tak uzyskane prawdopodobieństwa w formie tablicy dwuwymiarowej: \\
	 \begin{tabular}{c|cc|c}
	 	\diagbox{y}{x} & 0 & 1 & $\sum$ \\
	 	\hline
	 	0 & $\frac{36}{52}$ & $\frac{3}{52}$ & $\frac{39}{52}$ \\
	 	1 & $\frac{12}{52}$ & $\frac{1}{52}$ & $\frac{13}{52}$ \\
	 	\hline
	 	$\sum$ & $\frac{48}{52}$ & $\frac{4}{52}$ & 1
 	\end{tabular}
 
 	Wiersz i kolumna podpisane $\sum$ reprezentują, odpowiednio, sumy w kolumnach i w wierszach, i jednocześnie \emph{rozkłady brzegowe}, tzn. rozkłady prawdopodobieństwa zmiennych losowych $Y$ i $X$, takie same jak te, które wyznaczyliśmy w poprzednim podpunkcie.
\end{ansenv}
\item Oblicz dystrybuantę zmiennej losowej $(X,Y)$. 
\begin{ansenv}
	Dystrybuanta $F(u, v)$ w ogólności dana jest wzorem:
	\[ F(u, v) = P(X\leq u, Y\leq v) \]
	Punkty skokowe w każdym z wymiarów wyznaczają przedziały, w których dystrybuanta przyjmuje identyczne wartości. Zatem musimy rozważyć 9 takich par przedziałów:
	\[ \{(-\infty, 0), \left[0, 1\right), \left[1, \infty\right)\} \times \{(-\infty, 0), \left[0, 1\right), \left[1, \infty\right)\} \]
	Najwygodniej przedstawić to w postaci tabeli: 
	
    \begin{tabular}{c|ccc}
		\diagbox{v}{u} & $(-\infty, 0)$ & $[0, 1)$ & $(1, \infty)$ \\
		\hline
		$(-\infty, 0)$ & 0 & 0 & 0 \\
		$[0, 1)$ & 0 &  $\frac{36}{52}$ & $\frac{39}{52}$ \\
		$[1, \infty)$ & 0 & $\frac{48}{52}$ & 1 
	\end{tabular}

	Pierwszy wiersz i pierwsza kolumna zawierają same 0, bo zawsze co najmniej jeden z wymiarów jest przed pierwszym punktem skokowym, czyli jest niemożliwe żeby dana zmienna losowa przyjęła wartość nie większą niż dana.
	W prawym dolnym rogu zawsze 1, bo w obu wymiarach pytamy o sytuację gdzie zmienne losowe są nie większe niż największy z punktów skokowych - czyli zachodzi zdarzenie pewne.
	Pozostałe pola wypełniamy sumując kolejne fragmenty rozkładu prawdopodobieństwa, który uzyskaliśmy w poprzednim punkcie, np.:
	\[ F(0{,}5, 2) = P(X\leq 0{,}5, Y \leq 2) = P(X=0, Y=0) + P(X=0, Y=1) = \frac{36}{52}+\frac{12}{52} = \frac{48}{52} \]
	
	Dystrybuantę możemy zapisać też w formie funkcji sklejanej, np.
	\[F(u,v)=\begin{cases} 
	0 & u< 0 \lor v< 0 \\ 
	\frac{36}{52} & 0\leq u< 1 \land 0\leq v< 1 \\
	\frac{39}{52} & 1\leq u \land 0\leq v< 1 \\
	\frac{48}{52} & 0\leq u< 1 \land 1\leq v\\
	1 & 1\leq u \land 1\leq v
	\end{cases}\]
\end{ansenv}
\item Czy zmienne losowej $X$ i $Y$ są niezależne? Odpowiedź uzasadnij odpowiednim rachunkiem. 
\begin{ansenv}
	Posługujemy się definicją niezależności zmiennych losowych $X$ i $Y$:
	\[ \left[\forall x,y\in\mathbb{R}\colon P(X=x, Y=y)=P(X=x)\cdot P(Y=y) \right] \iff \text{$X$ i $Y$ są niezależne} \]
	W takim razie musimy sprawdzić dla wszystkich par liczb rzeczywistych czy powyższy iloczyn zachodzi.
	Dla par $(x, y)$ niebędących punktami skokowymi zachodzi trywialnie, bo po obu stronach mamy 0.
	Pozostają nam zatem 4 pary:  $\{(0,0), (0, 1), (1, 0), (1,1)\}$
	
	\begin{align*}
		P(X=0, Y=0)=\frac{36}{52} \qquad & P(X=0)\cdot P(Y=0)=\frac{48}{52}\cdot\frac{39}{52}=\frac{36}{52} & \checkmark \\
		P(X=0, Y=1)=\frac{12}{52} \qquad & P(X=0)\cdot P(Y=1)=\frac{48}{52}\cdot\frac{13}{52}=\frac{12}{52} & \checkmark \\
		P(X=1, Y=0)=\frac{3}{52} \qquad & P(X=1)\cdot P(Y=0)=\frac{4}{52}\cdot\frac{39}{52}=\frac{3}{52} & \checkmark \\
		P(X=1, Y=1)=\frac{1}{52} \qquad & P(X=1)\cdot P(Y=1)=\frac{4}{52}\cdot\frac{13}{52}=\frac{1}{52} & \checkmark \\
	\end{align*}

	Skoro dla wszystkich par się zgadza, to znaczy, że zmienne losowe $X$ i $Y$ są niezależne.
\end{ansenv}
\item Oblicz moment zwykły mieszany rzędu 1+1 zmiennej losowej $(X,Y)$. \ans{\[EXY=E(X\cdot Y)=\sum_{x}\sum_{y} x\cdot y\cdot P(X=x, Y=y) = 
	0\cdot 0\cdot \frac{36}{52} + 0\cdot 1\cdot \frac{12}{52} + 1\cdot 0\cdot \frac{3}{52} + 1\cdot 1\cdot \frac{1}{52} = \frac{1}{52}
	\]}
\item Oblicz kowariancję zmiennych losowych $X$ i $Y$.
\ans{
	\[\cov(X,Y)= EXY-EX\cdot EY = \frac{1}{52} - \left(0\cdot \frac{48}{52}+1\cdot \frac{4}{52}\right)\cdot\left(0\cdot\frac{39}{52}+1\cdot\frac{13}{52}\right) = 0 \]
	Oczywiście otrzymujemy 0, ponieważ zmienne losowe są niezależne, a zatem ich kowariancja musi wynosić 0.
}
\item Oblicz współczynnik korelacji zmiennych losowych $X$ i $Y$.
\ans{
	\[ \corr(X, Y) = \frac{\cov(X,Y)}{DX\cdot DY} = \frac{0}{DX\cdot DY} = 0 \]
}
%\item Wiadomo, że wylosowano trefla. Oblicz prawdopodobieństwo, że jest to walet posługując się warunkowym rozkładem prawdopodobieństwa. \ans{$P(X=1|Y=1)=\frac{\frac{1}{52}}{\frac{13}{52}}=\frac{1}{13}$}
\end{enumerate}
\item Z talii 52 kart wylosowano jedną kartę. Niech zmienna losowa $X$ przyjmuje wartość odpowiadającą liczbie wylosowanych dam trefl, zaś $Y$ odpowiadającą liczbie wylosowanych trefli.
\begin{enumerate}
\item Wyznacz rozkłady zmiennych losowych $X$ oraz $Y$. \ans{
	Zadanie jest prawie identyczne jak poprzednie jeżeli chodzi o technikę rozwiązywania.
	\[P(X=0)=\frac{51}{52} \qquad P(X=1)=\frac{1}{52} \]
	\[ P(Y=1)=\frac{13}{52} \qquad P(Y=0)=\frac{39}{52} \]
}
\item Wyznacz rozkład zmiennej losowej $(X,Y)$. 
\begin{ansenv}
 	\begin{tabular}{c|cc|c}
		\diagbox{y}{x} & 0 & 1 & $\sum$ \\
		\hline
		0 & $\frac{39}{52}$ & $0$ & $\frac{39}{52}$ \\
		1 & $\frac{12}{52}$ & $\frac{1}{52}$ & $\frac{13}{52}$ \\
		\hline
		$\sum$ & $\frac{51}{52}$ & $\frac{1}{52}$ & 1
	\end{tabular}
\end{ansenv}
\item Oblicz dystrybuantę zmiennej losowej $(X,Y)$.
\begin{ansenv}
	    \begin{tabular}{c|ccc}
		\diagbox{v}{u} & $(-\infty, 0)$ & $[0, 1)$ & $(1, \infty)$ \\
		\hline
		$(-\infty, 0)$ & 0 & 0 & 0 \\
		$[0, 1)$ & 0 &  $\frac{39}{52}$ & $\frac{39}{52}$ \\
		$[1, \infty)$ & 0 & $\frac{51}{52}$ & 1 
	\end{tabular}
\end{ansenv}
\item Czy zmienne losowej $X$ i $Y$ są niezależne? Odpowiedź uzasadnij odpowiednim rachunkiem. 
\begin{ansenv}
	Żeby stwierdzić, że nie są wystarczy pokazać jedną parę $(x, y)$ dla której równość $P(X=x, Y=y)=P(X=x)P(Y=y)$ nie zachodzi. Zatem niech $x=1$ i $y=0$:
	\[ P(X=1, Y=0)=0 \neq P(X=1)P(Y=0)=\frac{39}{52}\cdot\frac{1}{52}=\frac{1}{208} \]
	W takim razie zmienne losowe $X$ i $Y$ nie są niezależne.
	
	Można też na to popatrzeć intuicyjnie: czy z wartości zmiennej losowej $Y$ możemy się czegoś dowiedzieć o wartościach zmiennej losowej $X$?
	Przykładowo, jeżeli $Y=0$ (nie wylosowano trefla), to na pewno $X=0$ (nie wylosowano damy trefl), bo nie istnieją damy trefl niebędące treflami.
	W takim razie zmienne są zależne.
\end{ansenv}
\item Oblicz moment zwykły mieszany rzędu 1+1 zmiennej losowej $(X,Y)$. \ans{\[EXY=E(X\cdot Y)=\sum_{x}\sum_{y} x\cdot y\cdot P(X=x, Y=y) = 
	0\cdot 0\cdot \frac{39}{52} + 0\cdot 1\cdot \frac{12}{52} + 1\cdot 1\cdot \frac{1}{52} = \frac{1}{52}
	\]}
\item Oblicz kowariancję zmiennych losowych $X$ i $Y$.
\begin{ansenv}
	\begin{gather*}
	 EX = 0\cdot\frac{51}{52} + 1\cdot\frac{1}{52} = \frac{1}{52}  \\
	 EY = 0\cdot\frac{39}{52} + 1\cdot\frac{13}{52} = \frac{1}{4}  \\
	 \cov(X,Y) = EXY-EX\cdot EY = \frac{1}{52} - \frac{1}{52}\cdot \frac{1}{4} = \frac{3}{208} 
	\end{gather*}
\end{ansenv}
\item Oblicz współczynnik korelacji zmiennych losowych $X$ i $Y$.
\begin{ansenv}
	\begin{gather*}
	E(X^2) = \sum_x x^2P(X=x) = 0^2\cdot\frac{51}{52} + 1^2\cdot\frac{1}{52} = \frac{1}{52} \\
	E(Y^2) = \sum_y y^2P(Y=y) = 0^2\cdot\frac{39}{52} + 1^2\cdot\frac{13}{52} = \frac{13}{52} = \frac{1}{4} \\
	D^2X = E(X^2)-(EX)^2 = \frac{1}{52} - \left(\frac{1}{52}\right)^2 = \frac{51}{52^2} \\
	D^2Y = E(Y^2)-(EY)^2 = \frac{1}{4} - \left(\frac{1}{4}\right)^2 = \frac{3}{4^2} \\
	\corr(X,Y) = \frac{\cov(X,Y)}{DX\cdot DY} = \frac{\frac{3}{208}}{\sqrt{\frac{51}{52^2}}\cdot\sqrt{\frac{3}{4^2}}} = \frac{3}{\sqrt{51\cdot 3}} = \sqrt{\frac{3}{51}} \approx 0{,}2425
	\end{gather*}
\end{ansenv}
%\item Wiadomo, że wylosowano trefla. Oblicz prawdopodobieństwo, że nie jest to dama posługując się warunkowym rozkładem prawdopodobieństwa. \ans{$P(X=0|Y=1)=\frac{\frac{12}{52}}{\frac{13}{52}}=\frac{12}{13}$}
\end{enumerate}

\item W Shire szaleje burza śnieżna, a Merry i Pippin, zamknięci w czterech ścianach, grają w pewną nietypową grę, korzystając z dziwnej kostki sześciościennej, której każda ścianka wypada z równym prawdopodobieństwem, ale ścianki nie są ponumerowane tak jak zwykle.
Mianowicie, na ściankach znajdują się wyłącznie cyfry 1,2 oraz 3 w kolorach czerwonym i zielonym: dwie czerwone 1, jedna czerwona 2, jedna zielona 2 oraz dwie zielone 3.
Merry w sekrecie rzuca dwukrotnie kostką, mnoży wyniki i zlicza liczbę czerwonych wyników.
Niech zmienna losowa $X$ odpowiada iloczynowi, a $Y$ liczbie czerwonych wyników.
Merry podaje Pippinowi liczbę czerwonych wyników (tj. wartość, którą przyjęła zmienna losowa $Y$), a zadaniem Pippina jest zgadnąć iloczny wyrzuconych liczb (tj. wartość, którą przyjęła zmienna losowa $X$).

\begin{enumerate}
	\item Podaj brzegowy rozkład prawdopodobieństwa zmiennej losowej $X$. 
	\item Podaj brzegowy rozkład prawdopodobieństwa zmiennej losowej $Y$. 
	\item Podaj łączny rozkład prawdopodobieństwa zmiennej losowej dwuwymiarowej $(X,Y)$. 
	\begin{ansenv}
		Pierwsze trzy punkty rozwiązujemy łącznie budując łączny rozkład prawdopodobieństwa. Zaczniemy od stworzenia pomocniczej tabeli zawierającej pary wyników wraz z odpowiadającymi im kolorami oraz wartości iloczynów:
		
		\begin{tabular}{c|ccc|ccc}
			\diagbox{kostka 1}{kostka 2} & \textcolor{red}{1} & \textcolor{red}{1} & \textcolor{red}{2} & \textcolor{darkgreen}{2} & \textcolor{darkgreen}{3} & \textcolor{darkgreen}{3} \\
			\hline
			\textcolor{red}{1} & 1 & 1 & 2 & 2 & 3 & 3 \\
			\textcolor{red}{1} & 1 & 1 & 2 & 2 & 3 & 3 \\
			\textcolor{red}{2} & 2 & 2 & 4 & 4 & 6 & 6 \\
			\hline
			\textcolor{darkgreen}{2} & 2 & 2 & 4 & 4 & 6 & 6 \\
			\textcolor{darkgreen}{3} & 3 & 3 & 6 & 6 & 9 & 9 \\
			\textcolor{darkgreen}{3} & 3 & 3 & 6 & 6 & 9 & 9 \\
		\end{tabular}
	
		Lewy górny kwadrat odpowiada iloczynom, które powstają z dwóch czerwonych wyników (czyli $Y=2$). Prawy dolny kwadrat iloczynom, które powstają z dwóch zielonych wyników (czyli $Y=0$).
		Pozostałe dwa kwadraty są mieszane: jeden wynik czerwony, jeden wynik zielony (czyli $Y=1$).
		
		Rzuty kostką się niezależne od siebie, a każdy wynik równoprawdopodobny, zatem wystarczy zliczać liczbę wyników spełniających kryteria wyznaczone przez zmienne losowe i dzielić przez 36:
		
		\begin{tabular}{c|cccccc|c}
			& \multicolumn{7}{c}{zmienna $X$ -- iloczyn} \\
			\diagbox{y}{x} & 1 & 2 & 3 & 4 & 6 & 9 & $\sum$ \\
			\hline
			0 & 0 & 0 & 0 & $\frac{1}{36}$ & $\frac{4}{36}$ & $\frac{4}{36}$ & $\frac{9}{36}$ \\
			1 & 0 & $\frac{4}{36}$ & $\frac{8}{36}$ & $\frac{2}{36}$ & $\frac{4}{36}$ & 0 & $\frac{18}{36}$ \\
			2 & $\frac{4}{36}$ & $\frac{4}{36}$ & 0 & $\frac{1}{36}$ & 0 & 0 & $\frac{9}{36}$ \\
			\hline
			$\sum$ & $\frac{4}{36}$ & $\frac{8}{36}$ & $\frac{8}{36}$ & $\frac{4}{36}$ & $\frac{8}{36}$ & $\frac{4}{36}$ & $1$ \\
		\end{tabular}
	
		Wiersz oznaczony $\sum$ stanowi jednocześnie odpowiedź na podpunkt a, a kolumna oznaczona $\sum$ odpowiedź na podpunkt b.
	\end{ansenv}
	\item Czy zmienne losowe $X$ oraz $Y$ są niezależne? Odpowiedź uzasadnij.
	\ans{Zmienne są niezależne, bo np. $P(X=1,Y=0)=0\neq P(X=1)\cdot P(Y=0)=\frac{9}{36}\cdot\frac{4}{36}=\frac{1}{36}$}
	\item Oblicz wartości średnie zmiennych losowych $X$ oraz $Y$.
	\begin{ansenv}
		\[
		EX = 1\cdot \frac{4}{36} + 2\cdot\frac{8}{36} + 3\cdot\frac{8}{36} + 4\cdot\frac{4}{36} + 6\cdot\frac{8}{36} + 9\cdot\frac{4}{36} = \frac{144}{36} = 4
		\]
		\[
		EY = 0\cdot \frac{9}{36} + 1\cdot\frac{18}{36} + 2\cdot\frac{9}{36} = 1
		\]
	\end{ansenv}
	\item Oblicz kowariancję zmiennej losowej $(X,Y)$ (poniższe wyliczenia pomijają składniki sumy gdzie prawdopodobieństwo jest równe 0, co oczywiście nie wpływa na wynik)
	\begin{ansenv}
		\begin{align*}
		EXY = & \sum_x\sum_y x\cdot y\cdot P(X=x, Y=y) = \\ & 4\cdot 0\cdot \frac{1}{36} + 6\cdot 0\cdot \frac{4}{36} + 9\cdot 0\cdot \frac{4}{36} + \\ & 2\cdot 1\cdot \frac{4}{36} + 3\cdot 1\cdot \frac{8}{36} + 4\cdot 1\cdot \frac{2}{36} + 6\cdot 1\cdot \frac{4}{36} + \\ & 1\cdot 2\cdot \frac{4}{36} + 2\cdot 2\cdot \frac{4}{36} + 4\cdot 2\cdot \frac{1}{36} = \frac{96}{36} = \frac{8}{3}
		\end{align*}
		\[ \cov(X,Y)=EXY-EX\cdot EY = \frac{8}{3} - 4\cdot 1 = -\frac{4}{3} \]
	\end{ansenv}
	\item Oblicz odchylenia standardowe zmiennych losowych $X$ oraz $Y$.
	\begin{ansenv}
		\begin{gather*}
		EX^2 = 1^2\cdot \frac{4}{36} + 2^2\cdot\frac{8}{36} + 3^2\cdot\frac{8}{36} + 4^2\cdot\frac{4}{36} + 6^2\cdot\frac{8}{36} + 9^2\cdot\frac{4}{36} = \frac{784}{36} = \frac{196}{9} \\
		EY^2 = 0^2\cdot \frac{9}{36} + 1^2\cdot\frac{18}{36} + 2^2\cdot\frac{9}{36} = \frac{3}{2} \\
		DX = \sqrt{EX^2 - (EX)^2} = \sqrt{\frac{196}{9} - 4^2} = \sqrt{\frac{52}{9}} = \frac{2\sqrt{13}}{3} \\
		DY = \sqrt{EY^2- (EY)^2} = \sqrt{\frac{3}{2}-1^2} = \sqrt{\frac{1}{2}} = \frac{\sqrt{2}}{2}
		\end{gather*}
	\end{ansenv}
	\item Oblicz współczynnik korelacji zmiennej losowej $(X,Y)$.
	\begin{ansenv}
		\[\corr(X,Y)=\frac{\cov(X,Y)}{DX\cdot DY} = \frac{-\frac{4}{3}}{\frac{2\sqrt{13}}{3}\cdot \frac{\sqrt{2}}{2}} =
		\frac{-4}{\sqrt{13}\cdot\sqrt{2}} = \frac{-4}{\sqrt{26}} \approx -0{,}784 \]
	\end{ansenv}
\end{enumerate}

\item Hobbici w skórzanym woreczku mają 7 kartek zapisanych atramentami w dwóch różnych kolorach: czerwonym i zielonym.
Na każdej z kartek jest inna cyfra od 1 do 7, przy czym cyfry 1, 2, 3, 6, 7 są zapisane kolorem czerwonym, a cyfry 4, 5 kolorem zielonym.
Pippin losuje ze zwracaniem z woreczka dwie kartki.
Niech $X$ będzie zmienną losową odpowiadającą liczbie wylosowanych kartek, na których była napisana parzysta liczba, natomiast $Y$ zmienną losową odpowiadającą liczbie kartek zapisanych czerwonym atramentem.

\begin{enumerate}
\item Podaj rozkład brzegowy zmiennej $X$.
\item Podaj rozkład brzegowy zmiennej $Y$.
\item Podaj łączny rozkład prawdopodobieństwa zmiennej losowej $(X,Y)$.
\begin{ansenv}
	Podobnie jak w poprzednim zadaniu budujemy pomocniczą macierz, której nagłówki odpowiadają możliwym wartościom kartek, a w komórkach znajdują się odpowiadające im wartości zmiennej losowej $X$, natomiast wartości zmiennej losowej $Y$ będziemy widzieli na podstawie prostokątów wyznaczonych w macierzy:
	
	\begin{tabular}{c|ccc|cc|cc}
		\diagbox{II kartka}{I kartka} & \textcolor{red}{1} & \textcolor{red}{2} & \textcolor{red}{3} & \textcolor{darkgreen}{4} & \textcolor{darkgreen}{5} & \textcolor{red}{6} & \textcolor{red}{7} \\	
		\hline	
		 \textcolor{red}{1} & 0 & 1 & 0 & 1 & 0 & 1 & 0 \\
		 \textcolor{red}{2} & 1 & 2 & 1 & 2 & 1 & 2 & 1 \\
		 \textcolor{red}{3} & 0 & 1 & 0 & 1 & 0 & 1 & 0 \\
		 \hline
		 \textcolor{darkgreen}{4} & 1 & 2 & 1 & 2 & 1 & 2 & 1 \\
		 \textcolor{darkgreen}{5} & 0 & 1 & 0 & 1 & 0 & 1 & 0 \\
		 \hline
		 \textcolor{red}{6} & 1 & 2 & 1 & 2 & 1 & 2 & 1\\
		 \textcolor{red}{7} & 0 & 1 & 0 & 1 & 0 & 1 & 0 \\
	\end{tabular}

	Prostokąty w narożnikach odpowiadają dwóm kartkom zapisanym na czerwono, a więc $Y=2$.
	Kwadrat w środku odpowiada dwóm kartkom zapisanym na zielono, a więc $Y=0$.
	Pozostałe 4 prostokąty odpowiadają $Y=1$.
	Podobnie jak poprzednio: 49 równoprawdopodobnych możliwości, wystarczy zliczać i uzupełniać rozkład prawdopodobieństwa:
	
	\begin{tabular}{c|ccc|c}
		\diagbox{y}{x} & 0 & 1 & 2 & $\sum$ \\
		\hline
		0 & $\frac{1}{49}$ & $\frac{2}{49}$ & $\frac{1}{49}$ & $\frac{4}{49}$ \\
		1 & $\frac{6}{49}$ & $\frac{10}{49}$ & $\frac{4}{49}$ & $\frac{20}{49}$ \\
		2 & $\frac{9}{49}$ & $\frac{12}{49}$ & $\frac{4}{49}$ & $\frac{25}{49}$ \\
		\hline
		$\sum$ & $\frac{16}{49}$ & $\frac{24}{49}$ & $\frac{9}{49}$ & 1
	\end{tabular}
\end{ansenv}
\item Zbadaj, czy zmienne losowej $X$ oraz $Y$ są niezależne.
\ans{Zmienne nie są niezależne, ponieważ \[P(X=0, Y=0)=\frac{1}{49}\neq P(X=0)\cdot P(Y=0) = \frac{16}{49}\cdot\frac{4}{49}\]}
\item Oblicz moment zwykły mieszany rzędu 1+1 zmiennej losowej $(X,Y)$.
\ans{
	W obliczeniach $EXY$ (czyli momentu zwykłego mieszanego rzędu 1+1) pominięte są składniki wynoszące $0$.
	\begin{gather*}
	EXY = \sum_{x}\sum_{y} x\cdot y\cdot P(X=x, Y=y) = 1\cdot 1\cdot \frac{10}{49} + 2\cdot 1\cdot \frac{4}{49} + 1\cdot 2\cdot \frac{12}{49} + 2\cdot 2\cdot \frac{4}{49} = \frac{58}{49}
	\end{gather*}
}
\item Oblicz kowariancję zmiennych losowych $X$ i $Y$.
\begin{ansenv}
	\begin{gather*}
		EX = 0\cdot\frac{16}{49} + 1\cdot\frac{24}{49} + 2\cdot\frac{9}{49} = \frac{42}{49} = \frac{6}{7} \\
		EY = 0\cdot\frac{4}{49} + 1\cdot\frac{20}{49} + 2\cdot\frac{25}{49} = \frac{70}{49} = \frac{10}{7} \\
		\cov(X,Y) = EXY-EX\cdot EY = \frac{58}{49}-\frac{6}{7}\cdot\frac{10}{7}=-\frac{2}{49}
	\end{gather*}
\end{ansenv}
\item Oblicz współczynnik korelacji zmiennych losowych $X$ i $Y$.
\begin{ansenv}
	\begin{gather*}
	E(X^2) = 0^2\cdot\frac{16}{49} + 1^2\cdot\frac{24}{49} + 2^2\cdot\frac{9}{49} = \frac{60}{49} \\
	E(Y^2) = 0^2\cdot\frac{4}{49} + 1^2\cdot\frac{20}{49} + 2^2\cdot\frac{25}{49} = \frac{120}{49} \\
	DX = \sqrt{E(X^2)-(EX)^2} = \sqrt{\frac{60}{49}-\left(\frac{6}{7}\right)^2} = \sqrt{\frac{24}{49}} = \frac{2\sqrt{6}}{7} \\
	DY = \sqrt{E(Y^2)-(EY)^2} = \sqrt{\frac{120}{49}-\left(\frac{10}{7}\right)^2} = \sqrt{\frac{20}{49}} = \frac{2\sqrt{5}}{7} \\
	\corr(X,Y)=\frac{\cov(X,Y)}{DX\cdot DY} = \frac{-\frac{2}{49}}{\frac{2\sqrt{6}}{7}\cdot \frac{2\sqrt{5}}{7}} = \frac{-2}{4\sqrt{30}} = -\frac{1}{2\sqrt{30}} \approx -0{,}0912
	\end{gather*}
\end{ansenv}
\end{enumerate}

\item Hobbici w skórzanym woreczku mają 7 kartek zapisanych atramentami w dwóch różnych kolorach: czerwonym i zielonym.
Na każdej z kartek jest inna cyfra od 1 do 7, przy czym cyfry 1, 2, 3, 6, 7 są zapisane kolorem czerwonym, a cyfry 4, 5 kolorem zielonym.
Pippin losuje ze zwracaniem z woreczka dwie kartki.
Niech $X$ będzie zmienną losową odpowiadającą reszcie z~dzielenia przez 3 sumy liczb na wylosowanych kartkach, natomiast $Y$~zmienną losową odpowiadającą liczbie wylosowanych kartek zapisanych czerwonym atramentem.

\begin{enumerate}
\item Podaj rozkład brzegowy zmiennej $X$. (2 punkt)
\item Podaj rozkład brzegowy zmiennej $Y$. (2 punkt)
\item Podaj łączny rozkład prawdopodobieństwa zmiennej losowej $(X,Y)$. (3 punkty)
\item Zbadaj, czy zmienne losowej $X$ oraz $Y$ są niezależne. (2 punkty)
\item Oblicz moment zwykły mieszany rzędu 1+1 zmiennej losowej $(X,Y)$. (1 punkt)
\end{enumerate}

\item Pippin rzucił trzykrotnie sześciościenną, uczciwą kostką do gry. Niech zmienna $X$ odpowiada liczbie rzutów, w~których udało mu się wyrzucić mniej niż pięć
oczek, natomiast zmienna $Y$ liczbie rzutów, w których udało mu się wyrzucić przynajmniej dwa oczka.
\begin{enumerate}
\item Podaj brzegowe rozkłady prawdopodobieństwa zmiennych losowych $X$ oraz $Y$.
\item Podaj łączny rozkład prawdopodobieństwa zmiennej losowej dwuwymiarowej $(X,Y)$.
\item Zbadaj, czy zmienne losowe $X$ oraz $Y$ są zależne.
\item Oblicz moment zwykły mieszany rzędu 1+1 zmiennej losowej $(X,Y)$.
\item Oblicz kowariancję zmiennych losowych $X$ i $Y$.
\item Oblicz współczynnik korelacji zmiennych losowych $X$ i $Y$.
\end{enumerate}

\item Dana jest funkcja $f(x,y)$ określona poniższym wzorem:
\[ f(x,y)=\begin{cases} cxy & 0\leq x \leq 4 \land 0\leq y\leq 2 \\ 0 & \text{wpp}\end{cases} \]
\begin{enumerate}
\item Wyznacz stałą $c$ taką, żeby $f(x,y)$ była funkcją gęstości prawdopodobieństwa pewnej zmiennej losowej $(X,Y)$. \ans{$c=\frac{1}{16}$}
\item Wyznacz gęstości prawdopodobieństwa rozkładów brzegowych. \ans{$f_X(u)=\frac{u}{8} f_Y(v)=\frac{v}{2}$}
\item Wyznacz dystrybuantę zmiennej losowej $(X,Y)$. \ans{\[F(u,v)=\begin{cases}
	0 & u\leq 0 \lor v\leq 0 \\
	\frac{u^2v^2}{64} & 0<u\leq 4 \land 0<v\leq 2 \\
	\frac{u^2}{16} & 0<u\leq 4 \land v>2 \\
	\frac{v^2}{4} & u> 4 \land 0<v\leq 2 \\
	1 & u>4\land v>2
	\end{cases}\]}
\item Wiadomo, że realizacja zmiennej losowej $Y$ zawiera się w przedziale $(1,2)$. Jakie jest prawdopodobieństwo, że wtedy realizacja zmiennej losowej $X$ zawiera się w przedziale $(0,2)$?
\ans{$P(0<X<2|1<Y<2)=\frac{P(0<X<2,1<Y<2)}{P(1<Y<2)}=\frac{F(2,2)-F(2,1)-F(0,2)+F(0,1)}{F_Y(2)-F_Y(1)}=\frac{\frac{1}{4}-\frac{1}{16}-0+0}{1-\frac{1}{4}}=\frac{3}{16}\frac{4}{3}=\frac{1}{4}$}
\end{enumerate}
\end{enumerate}
\clearpage
\section{Regresja liniowa}
\begin{enumerate}
\item Produkcja piwa składa się m.in. z zaszczepienia brzeczki piwnej odpowiednią liczbą komórek drożdżowych. Jednym z typowych źródeł drożdży jest zebranie ich z dna pojemnika, w którym odbywała się poprzednia fermentacja (zebranie tzw. gęstwy drożdżowej). Zdarza się, że trzeba zebrać gęstwę pewien czas przed przygotowaniem następnej partii brzeczki piwnej. Niestety, drożdże w trakcie przechowywania obumierają.

W laboratorium pewnego browaru przeprowadzono 50 eksperymentów, badających
procent przeżywalności drożdży w zależności od czasu ich przechowywania. W
poniższej tabeli w kolumnach podano wartości zmiennej losowej $X$,
odpowiadającej liczbie dni, przez które gęstwa drożdżowa była przechowywana,
natomiast w~wierszach podane są wartości zmiennej losowej $Y$, przedstawiającej
procent żywych komórek drożdżowych w gęstwie. W komórkach podane są liczby
eksperymentów, w których uzyskano daną parę warości $(X,Y)$.

\begin{tabular}{r|rrrr|r}
\diagbox{Y}{X} & 3 & 7 & 14 & 21 & $\sum$ \\
\hline
95 	& 8 & 1 & 0 & 0 & 9\\
85 	& 1 & 8 & 1 & 0 & 10 \\
70 	& 1 & 4 & 8 & 4 & 17 \\
40 	& 2 & 2 & 4 & 6 & 14 \\
\hline
$\sum$ 	& 12 & 15 & 13 & 10 & 50\\
\end{tabular}

\begin{enumerate}
\item Podaj (w formie tabeli dwudzielczej) rozkład prawdopodobieństwa zmiennej losowej dwuwymiarowej $(X,Y)$ wraz z~rozkładami brzegowymi.
\item Oblicz wartości średnie i wariancje zmiennych losowych jednowymiarowych $X$ oraz $Y$. \ans{$EX=10{,}66 EY=69{,}1 DX=6{,}51 DY=20{,}21$}
\item Oblicz współczynnik korelacji zmiennej losowej $(X,Y)$. Uzasadnij za pomocą obliczonego współczynnika, czy można spodziewać się zachodzenia związku liniowego między zmiennymi $X$ oraz $Y$? \ans{$cov(X,Y)=-74{,}21, \varrho=-0{,}56$}
\item Wykorzystaj regresję liniową do wyjaśnienia zmiennej losowej $Y$ w kategoriach zmiennej losowej $X$, tzn. oblicz współczynniki równania $Y=aX+b$. \ans{$a=-1{,}83 b=88{,}61$}
\item Jakiej przeżywalności drożdży należy się spodziewać po 10 dniach przechowywania gęstwy, a jakiej po 180 dniach? Wykorzystaj model obliczony w poprzednim punkcie i przedyskutuj sensowność otrzymanych wyników. \ans{$y(10)=70, y(180)<0$}
\end{enumerate}

\item W Minas Tirith dysponują setką piwnic, w~których przechowują żywność na
wypadek oblężenia. W~każdej z piwnic jest 1000 beczek z solonym mięsem.
Piwnice rozmieszczone są na poziomach od 0 do 3. Niech zmienna losowa $X$
oznacza poziom, na którym znajduje się piwnica, a $Y$ liczbę beczek z zepsutym
jedzeniem wykrytych podczas dorocznej kontroli. W tabeli poniżej znajduje się
rozkład prawdopodobieństwa zmiennej losowej dwuwymiarowej $(X,Y)$.

\begin{tabular}{|r|r|r|r|}
\hline
\diagbox{$X$}{$Y$} & \textbf{0} & \textbf{1} & \textbf{2} \\
\hline
\textbf{0} & 0 & 0,05 & 0,2 \\
\hline
\textbf{1} & 0,05 & 0,15 & 0,05\\
\hline
\textbf{2} & 0,15 & 0,1 & 0 \\
\hline
\textbf{3} & 0,2 & 0,05 & 0\\
\hline
\end{tabular}

\begin{enumerate}
\item Oblicz wartości średnie i wariancje zmiennych losowych jednowymiarowych $X$ oraz $Y$.
\item Oblicz moment zwykły mieszany rzędu $1+1$ zmiennej losowej $(X,Y)$.
\item Oblicz współczynnik korelacji zmiennej losowej $(X,Y)$. Uzasadnij za pomocą obliczonego współczynnika, czy można spodziewać się zachodzenia związku liniowego między zmiennymi $X$ oraz $Y$?
\item Wykorzystaj regresję liniową i oblicz współczynniki równania $Y=aX+b$.
\item Ilu beczek z zepsutym mięsem należałoby się spodziewać w piwnicach Białej Wieży położonej na siódmy poziomie miasta? Wykorzystaj model obliczony w poprzednim punkcie i przedyskutuj sensowność otrzymanych wyników.
\item Oblicz prawdopodobieństwo, że w piwnicy będą dwie beczki z zepsutym jedzeniem, jeżeli wiadomo, że piwnica znajduje się na pierwszym poziomie.
\end{enumerate}

\end{enumerate}

\clearpage
\section{Zmienne losowe dyskretne}
\begin{enumerate}
\item Rozważamy doświadczenie polegające na jednokrotnym rzucie uczciwą kostką sześciościenną
\begin{enumerate}
\item Zaproponuj zmienną losową $X$ odpowiednią do tego doświadczenia. \ans{$X\colon\Omega\to\{1,\ldots,6\}$}
\item Podaj rozkład prawdopodobieństwa zmiennej losowej $X$. \ans{$P(X=i)=\frac{1}{6}$}
\item Podaj dystrybuantę zmiennej losowej $X$ i narysuj jej wykres.
\item Oblicz $P(X<4)$ korzystając z rozkładu prawdopodobieństwa. \ans{$P(X<4)=P(X=1)+P(X=2)+P(X=3)$}
\item Oblicz $P(X<4)$ korzystając z dystrybuanty. \ans{$P(X<4)=F_X(4)$}
\item Oblicz $P(X>2)$ korzystając z rozkładu prawdopodobieństwa. \ans{$P(X>2)=P(X=3)+\ldots+P(X=6)$}
\item Oblicz $P(X>2)$ korzystając z dystrybuanty. \ans{$P(X>2)=1-P(X\leq 2)=1-P(X<3)=1-F(3)$}
\item Oblicz wartość średnią. \ans{$EX=\frac{21}{6}=3{,}5$}
\item Oblicz wariancję. \ans{$D^2X=\frac{1}{6}((-2{,}5)^2+(-1{,}5)^2+(-0{,}5)^2+2{,}5^2+1{,}5^2+0{,}5^2)$}
\end{enumerate}
\item Prawdopodobieństwo trafienia do celu w jednym strzale jest równie $\frac{1}{5}$. Niech $X$ przyjmuje wartość 1 jeżeli udało się trafić i 0 w przeciwnym przypadku.
\begin{enumerate}
\item Podaj rozkład zmiennej losowej $X$. \ans{$P(X=1)=p\quad P(X=0)=1-p$}
\item Oblicz średnią liczbę celnych strzałów. \ans{$EX=p=0{,}2$}
\item Oblicz odchylenie standardowe zmiennej losowej $X$. \ans{$DX=\sqrt{p(1-p)}=0{,}4$}
\item Podaj najbardziej prawdopodobną wartość zmiennej losowej $X$. \ans{$0$}
\end{enumerate}
\item Rozważamy doświadczenie polegające na obserwacji sumy oczek na dwóch uczciwych kostkach sześciościennych
\begin{enumerate}
\item Zaproponuj zmienną losową $X$ odpowiednią do tego doświadczenia. \ans{$X\colon\Omega\to\{1,\ldots,12\}$}
\item Podaj rozkład prawdopodobieństwa zmiennej losowej $X$. \ans{$(1,2,3,4,5,6,5,4,3,2,1)/36$}
\item Podaj dystrybuantę zmiennej losowej $X$ i narysuj jej wykres.
\item Oblicz $P(X>5)$ korzystając z dystrybuanty.
\item Oblicz wartość średnią. \ans{$7$}
\item Oblicz odchylenie standardowe. \ans{$\sqrt{D^2X}=\sqrt{54{,}83-49}=2{,}42$}
\end{enumerate}
\item Bilbo bierze udział w grze, w której punkty zdobywa się za trafianie kamykami do celu. Każdemu zawodnikowi przysługuje maksymalnie pięć rzutów, przy czym:
\begin{itemize}
\item na początku każdy zawodnik ma jeden punkt;
\item każdy trafiony rzut powoduje podwojenie liczby posiadanych punktów;
\item rzut nietrafiony oznacza koniec gry dla danego zawodnika.
\end{itemize}
Prawdopodobieństwo, że Bilbo w pojedynczym rzucie trafi do celu wynosi $0{,}7$. Niech $X$ będzie zmienną losową oznaczającą liczbę punktów zdobytych przez
Bilba.
\begin{enumerate}
\item Podaj funkcję prawdopodobieństwa zmiennej losowej $X$.
\item Oblicz dystrybuantę zmiennej losowej $X$.
\item Oblicz $EX$.
\item Oblicz wariancję zmiennej losowej $X$.
\item Oblicz $P(5\leq X\leq 30)$.
\end{enumerate}
\item Z talii 52 kart wylosowano jedną kartę. Niech zmienna losowa $X$ przyjmuje wartość odpowiadającą liczbie wylosowanych waletów, zaś $Y$ odpowiadającą liczbie wylosowanych trefli.
\begin{enumerate}
\item Wyznacz rozkłady zmiennych losowych $X$ oraz $Y$. \ans{$P(X=0)=\frac{48}{52} P(X=1)=\frac{4}{52} P(Y=1)=\frac{13}{52} P(Y=0)=\frac{39}{52}$}
\item Wyznacz rozkład zmiennej losowej $(X,Y)$. \ans{$P(0,0)=\frac{36}{52} P(1,0)=\frac{3}{52} P(0,1)=\frac{12}{52} P(1,1)=\frac{1}{52}$}
\item Oblicz dystrybuantę zmiennej losowej $(X,Y)$. \ans{\[F(u,v)=\begin{cases} 
	0 & u\leq 0 \lor v\leq 0 \\ 
	\frac{36}{52} & 0<u\leq 1 \land 0<v\leq 1 \\
	\frac{39}{52} & u>1 \land 0<v\leq 1 \\
	\frac{48}{52} & 0<u\leq 1 \land v>1 \\
	1 & u>1 \land v>1
	\end{cases}\]}
\item Czy zmienne losowej $X$ i $Y$ są niezależne? Odpowiedź uzasadnij odpowiednim rachunkiem. \ans{Tak}
\end{enumerate}

\clearpage
\item Z talii 52 kart wylosowano jedną kartę. Niech zmienna losowa $X$ przyjmuje wartość odpowiadającą liczbie wylosowanych dam trefl, zaś $Y$ odpowiadającą liczbie wylosowanych trefli.
\begin{enumerate}
\item Wyznacz rozkłady zmiennych losowych $X$ oraz $Y$. \ans{$P(X=0)=\frac{51}{52} P(X=1)=\frac{1}{52} P(Y=1)=\frac{13}{52} P(Y=0)=\frac{39}{52}$}
\item Wyznacz rozkład zmiennej losowej $(X,Y)$. \ans{$P(0,0)=\frac{39}{52} P(1,0)=0 P(0,1)=\frac{12}{52} P(1,1)=\frac{1}{52}$}
\item Oblicz dystrybuantę zmiennej losowej $(X,Y)$.
\item Czy zmienne losowej $X$ i $Y$ są niezależne? Odpowiedź uzasadnij odpowiednim rachunkiem. \ans{Nie, $0=P(1,0)\neq P(X=1)P(Y=0)=\frac{39}{52^2}$}
\end{enumerate}
\item Doświadczenie polega na dwukrotnym rzucie kostką sześciościenną. Niech zmienna $X$ odpowiada liczbie rzutów, w~których wyrzucono parzystą liczbę oczek, natomiast zmienna $Y$ liczbie rzutów, w których wyrzucono co najmniej 5 oczek.
\begin{enumerate}
\item Podaj brzegowe rozkłady prawdopodobieństwa zmiennych losowych $X$ oraz $Y$.
\item Podaj łączny rozkład prawdopodobieństwa zmiennej losowej dwuwymiarowej $(X,Y)$.
\item Zbadaj, czy zmienne losowe $X$ oraz $Y$ są niezależne.
\item Wiadomo, że dwukrotnie wyrzucono parzystą liczbę oczek. Jakie jest prawdopodobieństwo, że były to dwie szóstki? Odpowiedź przedstaw wykorzystując rozkład prawdopodobieństwa zmiennej losowej $(X,Y)$.
\end{enumerate}

\item Dana jest następująca gra: gracz rzuca uczciwą kostką sześciościenną tak długo, dopóki nie wyrzuci piątki bądź
szóstki, ale nie więcej niż trzy razy. Jeżeli uda mu się wyrzucić założoną liczbę oczek w $k$-tym rzucie, wygrywa $5-k$
zł, w przeciwnym razie nie wygrywa nic. Niech zmienna losowa $X$ odpowiada wysokości wygranej. Podaj, dbając
by przedstawić tok rozumowania:
\begin{enumerate}
\item funkcję prawdopodobieństwa zmiennej losowej $X$;
\item dystrybuatnę zmiennej losowej $X$;
\item prawdopodobieństwo, że gracz wygra nie mniej niż 3, a nie więcej niż 4 zł;
\item średnią wartość wygranej;
\item odchylenie standardowe zmiennej losowej $X$.
\end{enumerate}
\item Linia technologiczna składająca samochody składa 5 sztuk w ciągu godziny. Przy okazji składania każdego
z samochodów istnieje prawdopodobieństwo $0{,}1$, że maszyna montująca drzwi kierowcy ulegnie rozregulowaniu
i będzie rysować lakier. Co gorsza, rozregulowanie jest trwałe w tym sensie, że dopóki technik nie wyreguluje
maszyny wszystkie montowane drzwi będą rysowane. W związku z kryzysem firma postanowiła wprowadzić
oszczędności i technik regulujący maszyny dokonuje kontroli tylko raz na godzinę. Porysowanie lakieru na jednych
drzwiach to koszt 600 zł. Niech zmienna $X$ odpowiada kwocie, która firma straciła w ciągu godziny w wyniku
porysowania lakieru przez maszynę montującą drzwi. Podaj, dbając by przedstawić tok rozumowania:
\begin{enumerate}
\item funkcję prawdopodobieństwa zmiennej losowej $X$;
\item dystrybuatnę zmiennej losowej $X$;
\item średnią stratę;
\item odchylenie standardowe zmiennej losowej $X$;
\item prawdopodobieństwo, że firma straci przynajmniej 1000 zł w ciągu godziny
\end{enumerate}

\item Hobbici w skórzanym woreczku mają 7 kartek zapisanych atramentami w dwóch różnych kolorach: czerwonym i zielonym.
Na każdej z kartek jest inna cyfra od 1 do 7, przy czym cyfry 1, 2, 3, 6, 7 są zapisane kolorem czerwonym, a cyfry 4, 5 kolorem zielonym.
Pippin losuje ze zwracaniem z woreczka dwie kartki.
Niech $X$ będzie zmienną losową odpowiadającą liczbie wylosowanych kartek, na których była napisana parzysta liczba, natomiast $Y$ zmienną losową odpowiadającą liczbie kartek zapisanych czerwonym atramentem.
\begin{enumerate}
\item Podaj rozkład brzegowy zmiennej $X$.
\item Podaj rozkład brzegowy zmiennej $Y$.
\item Podaj łączny rozkład prawdopodobieństwa zmiennej losowej $(X,Y)$.
\item Zbadaj, czy zmienne losowej $X$ oraz $Y$ są niezależne.
\end{enumerate}
\end{enumerate}

\clearpage
\section{Powtórka}
\begin{enumerate}
\item Hobbici w skórzanym woreczku mają 7 kartek zapisanych atramentami w dwóch różnych kolorach: czerwonym i zielonym.
Na każdej z kartek jest inna cyfra od 1 do 7, przy czym cyfry 1, 2, 3, 6, 7 są zapisane kolorem czerwonym, a cyfry 4, 5 kolorem zielonym.
Pippin losuje ze zwracaniem z woreczka dwie kartki.
Niech $X$ będzie zmienną losową odpowiadającą liczbie wylosowanych kartek, na których była napisana parzysta liczba, natomiast $Y$ zmienną losową odpowiadającą liczbie kartek zapisanych czerwonym atramentem.
\begin{enumerate}
\item Podaj rozkład brzegowy zmiennej $X$.
\item Podaj rozkład brzegowy zmiennej $Y$.
\item Podaj łączny rozkład prawdopodobieństwa zmiennej losowej $(X,Y)$.
\item Zbadaj, czy zmienne losowej $X$ oraz $Y$ są niezależne.
\end{enumerate}
\item Pippin rzucił trzykrotnie sześciościenną, uczciwą kostką do gry. Niech zmienna $X$ odpowiada liczbie rzutów, w~których udało mu się wyrzucić mniej niż pięć
oczek, natomiast zmienna $Y$ liczbie rzutów, w których udało mu się wyrzucić przynajmniej dwa oczka.
\begin{enumerate}
\item Podaj brzegowe rozkłady prawdopodobieństwa zmiennych losowych $X$ oraz $Y$.
\item Podaj łączny rozkład prawdopodobieństwa zmiennej losowej dwuwymiarowej $(X,Y)$.
\item Zbadaj, czy zmienne losowe $X$ oraz $Y$ są zależne.
\end{enumerate}
\item W Bucklandzie na jesieni przygotowują sery na zimę. W każdym ze 100 magazynów
zgromadzonych jest po 1000 serów. Prawdopodobieństwo zepsucia się pojedynczego
krążka sera wynosi $4\cdot 10^{-3}$ i~nie zależy od pozostałych serów. Niech
$i=1,2,\ldots,100$ oznacza numer magazynu i niech zmienna losowa $X_i$ oznacza
liczbę zepsutych serów składowanych w $i$-tym magazynie. Ponadto, niech zmienna
losowa $Y=X_1+X_2+\ldots+X_{100}$ odpowiada sumarycznej liczbie zepsutych serów
we wszystkich magazynach. Załóż, że:
\begin{itemize}
\item wszystkie zmienne $X_i$ mają taki sam rozkład;
\item zmienne $X_i$ są niezależne;
\item zmienna $Y$ ma rozkład normalny o wartości średniej 100 razy większej niż wartość średnia dowolnej ze zmiennych $X_i$;
\item zmienna $Y$ ma rozkład normalny o wariancji 100 razy większej niż wariancja dowolnej ze zmiennych $X_i$.
\end{itemize}

\begin{enumerate}
\item Podaj  (w formie funkcji prawdopodobieństwa) rozkład prawdopodobieństwa zmiennej losowej $X_1$.
\item Oblicz (z dokładnością do dwóch miejsc po przecinku) prawdopodobieństwo, że w magazynie nr 10 zepsują się mniej niż trzy sery.
\item Podaj średnią liczbę zepsutych serów w magazynie nr 15.
\item Podaj wartości $EY$ oraz $DY$.
\item Utwórz zmienną losową $Z$ będącą standaryzowaną postacią zmiennej losowej $Y$.
\item Oblicz (z dokładnością do dwóch miejsc po przecinku) prawdopodobieństwo, że w Bucklandzie zepsują się więcej niż 424 sery.
\end{enumerate}

\item Stary Tom, kiedy nikt nie widzi, udaje się w przebraniu do jednego z kasyn w Bree na kilka partii pewnej gry losowej.
W jego sakiewce znajduje się 100 koron.
Tom jest bardzo przewidywalnym graczem: zawsze gra dokładnie 15 gier i wychodzi.
W każdej z~gier Tom ma szansę wygranej $p=0{,}25$.
Niech $X$ będzie zmienną losową oznaczającą liczbę gier, w~których Tom wygrał.
Każda gra kosztuje 5 koron, w przypadku przegranej Tom nie otrzymuje nic, a w przypadku wygranej otrzymuje 10 koron.
W związku z tym Tom może maksymalnie stracić 75 koron, a zyskać również maksymalnie 75 koron.
Niech $Y$ będzie zmienną losową oznaczającą ile koron znajduje się w sakiewce Toma po wyjściu z kasyna.

\begin{enumerate}
	\item Podaj funkcję prawdopodobieństwa zmiennej losowej $X$.
	\item Oblicz wartość średnią i odchylenie standardowe zmiennej losowej $X$.
	\item Podaj funkcję prawdopodobieństwa zmiennej losowej $Y$.
	\item Oblicz wartość średnią i odchylenie standardowe zmiennej losowej $Y$.
	\item Oblicz z dokładnością do co najmniej dwóch miejsc po przecinku prawdopodobieństwo, że Tom wyjdzie z~mniej niż 45 koronami.
	\item Jaka jest najbardziej prawdopodobna kwota, która znajduje się w sakiewce Toma przy wyjściu? Uwaga: jeżeli jest więcej
		niż jedna taka kwota, to podaj wszystkie.
\end{enumerate}

\item Sam Gamgee i Tom Cotton w ramach wieczornych rozrywek grają w kości.
Każda z kostek ma 10 ścian ponumerowanych liczbami od 1 do 10.
Każdy z hobbitów ma po 3 kostki, każda w nieco innym kolorze, przy czym kostki Sama są niebieskie, a Toma żółte.
Pojedyncza partia polega na wykonaniu jednoczesnego rzutu wszystkimi kostkami przez obu hobbitów.
Niech zdarzenie A polega na wyrzuceniu przez Sama pary liczb (to znaczy na dwóch kostkach jest taka sama wartość, a na trzeciej inna), a przez Toma układu $(6,7,8)$ (to znaczy na którejś z kostek jest 6, na innej 7, a na trzeciej 8).
\begin{enumerate}
\item Zdefiniuj przestrzeń zdarzeń elementarnych.
\item Jaki jest rozmiar przestrzeni zdarzeń elementarnych?
\item Przestaw zdarzenie A jako zbiór zdarzeń elementarnych.
\item Oblicz prawdopodobieństwo A.
\item Czy zdarzenia \emph{Sam wyrzucił parę} oraz \emph{Tom wyrzucił układ $(6,7,8)$} są zdarzeniami niezależnymi? Odpowiedź uzasadnij.
\end{enumerate}

\item Pippin zaangażował się w grę hazardową, która składa się z trzech rund.
W każdej rundzie Pippin może wygrać z prawdopodobieństwem $0{,}55$ i podwoić liczbę posiadanych monet albo przegrać i stracić $\frac{3}{4}$ pieniędzy.
Początkowo Pippin ma 64 monety.
Niech $W$ będzie zmienną losową oznaczającą liczbę wygranych przez Pippina rund, a $X$ zmienną losową oznaczającą liczbę monet posiadanych przez Pippina po zakończeniu gry.
\begin{enumerate}
\item Podaj rozkład prawdopodobieństwa zmiennej losowej $W$.
\item Oblicz wartość średnią zmiennej losowej $W$.
\item Oblicz odchylenie standardowe zmiennej losowej $W$.
\item Podaj wzór matematyczny łączący $X$ i $W$.
\item Podaj rozkład prawdopodobieństwa zmiennej losowej $X$.
\end{enumerate}

\item Grupa 20 hobbitów dzieli się do gry w piłkę losowo na zespół czerwony i zespół zielony, każdy z nich po dziesięciu hobbitów.
Wśród tych 20 hobbitów, czworo z nich jest bardzo dobrych w tej grze, trzynaścioro dobrych, a trzech przeciętnych.
Niech zdarzenie $A$ odpowiada sytuacji, w której w zespole czerwonym są trzej bardzo dobrzy gracze i tylko jeden przeciętny, a zdarzenie $B$ sytuacji, w której w zespole zielonym jest siedmiu graczy dobrych.
\begin{enumerate}
\item Zdefiniuj przestrzeń zdarzeń elementarnych.
\item Jaki jest rozmiar przestrzeni zdarzeń elementarnych?
%\item Czy w tej przestrzeni wszystkie zdarzenia elementarne są równoprawdopodobne? (1 punkt)
\item Przestaw zdarzenie $A$ jako zbiór zdarzeń elementarnych.
\item Oblicz prawdopodobieństwo $A$.
\item Czy zdarzenia $A$ i $B$ są zdarzeniami niezależnymi? Odpowiedź uzasadnij odpowiednim rachunkiem.
\end{enumerate}
\end{enumerate}

\clearpage
\section{Dowody}
Udowodnij, że:
\begin{enumerate}
\item zdarzenie pewne i dowolne zdarzenie $A$ są niezależne;
\item zdarzenie niemożliwe i dowolne zdarzenie $A$ są niezależne;
\item zdarzenie pewne i niemożliwe są niezależne;
\item jeżeli zdarzenia $A$ i $B$ są niezależne, to zdarzenia $A$ i $B'$ są niezależne;\ans{$P(A\cap B')=P(A\cap[\Omega\backslash B])=P([A\cap \Omega]\backslash[A\cap B])=P(A\backslash[A\cap B])=P(A)-P(A\cap B)=P(A)-P(A)P(B)=P(A)(1-P(B))=P(A)P(B')$}
\item jeżeli zdarzenia $A$, $B$ i $C$ są niezależne, to zdarzenia $A\cup B$ i $C$ są niezależne;\ans{$P([A\cup B]\cap C)=P([A\cap C]\cup[B\cap C])=P(A\cap C)+P(B\cap C)-P(A\cap B\cap C)=P(A)P(C)+P(B)P(C)-P(A)P(B)P(C)=P(C)(P(A)+P(B)-P(A)P(B))=P(C)P(A\cup B)$}
\item prawdopodobieństwo warunkowe spełnia pierwszy aksjomat;
\item prawdopodobieństwo warunkowe spełnia drugi aksjomat;
\item prawdopodobieństwo warunkowe spełnia trzeci aksjomat;
\item zachodzi twierdzenie o prawdopodobieństwie całkowitym: \[P(B)=\sum_{i=1}^\infty P(B|A_i)P(A_i) \]
\item zachodzi twierdzenie Bayesa \[P(A_i|B)=\frac{P(A_i)P(B|A_i)}{\sum_{j=1}^\infty P(A_j)P(B|A_j)} \]
\item jeżeli zdarzenia $A$, $B$ i $C$ są niezależne oraz $C$ nie jest zdarzeniem niemożliwym, to \[P(A\cap B|C)=P(A|C)P(B|C)\]
\item dla dowolnej zmiennej losowej $X$ i dowolnej stałej $c$ \[D^2(X+c)=D^2(X)\]
\item dla dowolnej zmiennej losowej $X$ \[D^2(X)=E(X^2)-E^2(X)\]
\item dla dowolnej zmiennej losowej $X$ i dowolnej stałej $c$ różnej od $EX$ zachodzi \[D^2(X)<E(X-c)^2\]
\item dla dowolnej zmiennej losowej $X$ zmienna losowa $Y$ dana poniższym wzorem jest unormowana \[Y=\frac{X-E(X)}{D(X)}\]
\item wartość średnia iloczynu dowolnej, skończonej liczby niezależnych zmiennych losowych o skończonych wartościach średnich jest równa iloczynowi wartości średnich tych zmiennych (indukcyjnie, oddzielnie dla zmiennych każdego typu);
\item wariancja sumy dowolnej skończonej liczby niezależnych zmiennych losowych o skończonych wariancjach jest równa sumie wariancji tych zmiennych (indukcyjnie);
\item wartość średnia i wariancja zmiennej losowej $Y=X_1+\ldots+X_n$ są dane poniższym wzorem, gdzie $X_i$ są niezależnymi zmiennymi o rozkładzie zero-jedynkowym z parametrem $p$
\[E(Y)=np \qquad D^2(Y)=np(1-p) \]
\end{enumerate}

\clearpage


\section{Truncated exponential backoff}
W sieciach opartych na współdzielonym medium (np. Ethernet przy stosowaniu koncentratorów) dochodzi do kolizji, tzn. sytuacji, gdy dwóch nadawców zaczyna jednocześnie nadawać.
Jedną z technik rozwiązywania tego problemu jest algorytm \emph{truncated exponential backoff}, przedstawiony poniżej.
Algorytm ten jest uruchamiany jeżeli pierwsze podejście do wysłania pakietu się nie powiodło, zmienna $n$ numeruje kolejne próby retransmisji pakietu.
(W Ethernecie zamiast liczb $8$ i $5$ używane są, odpowiednio, $16$ i $10$.
Nie ma to istotne znaczenia przy analizie poza większą ilością obliczeń do wykonania.)
\begin{algorithm}
\For{$n\gets 1\, \KwTo\, 8$}
{
	\lIf{$n\leq 5$}
	{
		$k \gets $ wylosuj liczbę z $\{0, 1 \ldots, 2^n-1\}$ korzystając z rozkładu jednostajnego \label{backoff:delay1}
	}
	\lElse
	{
		$k \gets $ wylosuj liczbę z $\{0, 1 \ldots, 2^5-1\}$  korzystając z rozkładu jednostajnego \label{backoff:delay2}
	}
	poczekaj $k$ jednostek czasu \label{backoff:wait}\;
	nadaj pakiet \;
	\If{udało się nadać pakiet bez kolizji}
	{
		zakończ algorytm z sukcesem: pakiet został wysłany
	}
}
zakończ algorytm z porażką: nie udało się nadać pakietu
\end{algorithm}

Rozważ dwóch nadawców $A$ i $B$, którzy próbują nadać pakiet.
Nastąpiła kolizja i oboje uruchomili powyższy algorytm, ale \emph{niezależnie} od siebie wybierając losowe opóźnienie w liniach \ref{backoff:delay1} i \ref{backoff:delay2}.
W tym kontekście rozpatrz zadania poniżej.
W miarę możliwości wszystkie obliczenia wykonuj na ułamkach zwykłych, bo będziesz miał do czynienia z dużymi liczbami i ułamki dziesiętne liczone na zwykłym kalkulatorze szybko mogą zaprowadzić Cię na manowce.
\begin{enumerate}
\item Niech $A_1$ będzie zmienną losową odpowiadającą opóźnieniu wylosowanemu przez nadawcę $A$ w linii \ref{backoff:delay1} przed pierwszą retransmisją ($n=1$).
Uzupełnij poniższy rozkład prawdopodobieństwa zmiennej losowej $A_1$.
\[ P(A_1=0)=\ldots\ldots\ldots \qquad P(A_1=1)=\ldots\ldots\ldots \]
\ans{\[P(A_1=0)=P(A_1=1)=\frac{1}{2}\]}
\item Uogólnij powyższą analizę na zmienne $A_n$ i $B_n$, tzn. opóźnienie wylosowane przed $n$-tą retransmisją: przez nadawcę $A$ dla $A_n$ i przez nadawcę $B$ dla $B_n$.
\[ P(A_n=k)=\ldots\ldots\ldots\ldots \qquad P(B_n=k)=\ldots\ldots\ldots\ldots \qquad n=1, 2, 3, 4, 5\, k\in\{0, 1 \ldots, 2^n-1\} \]
\[ P(A_n=k)=\ldots\ldots\ldots\ldots \qquad P(B_n=k)=\ldots\ldots\ldots\ldots \qquad n=6, 7, 8\, k\in\{0, 1 \ldots, 2^5-1\} \]
\ans{
	\[P(A_n=k)=P(B_n=k)=\frac{1}{2^n} \quad n\leq 5 \]
	\[P(A_n=k)=P(B_n=k)=\frac{1}{2^5} \quad n> 5 \]
}
\item Niech $K_n$ będzie zmienną losową przyjmującą wartość $1$ jeżeli przy $n$-tej retransmisji nastąpiła kolizja, tzn. obaj nadawcy wylosowali dokładnie to samo opóźnienie, a $0$ jeżeli kolizja nie wystąpiła.
Uzupełnij poniższy rozkład prawdopodobieństwa:
\[ P(K_n=1)=P(A_n=B_n)=\ldots\ldots\ldots\ldots \qquad P(K_n=0)=\ldots\ldots\ldots\ldots \qquad n=1, 2, \ldots, 5\]
\[ P(K_n=1)=P(A_n=B_n)=\ldots\ldots\ldots\ldots \qquad P(K_n=0)=\ldots\ldots\ldots\ldots \qquad n=6, 7, 8\]
\ans{
	\[ P(K_n=1)=P(A_n=B_n) = \sum_{k=0}^{2^n-1} P(A_n=k, B_n=k) = \sum_{k=0}^{2^n-1} P(A_n=k) P(B_n=k) = \frac{1}{2^n} \]
}
\item Niech $N$ będzie zmienną losową oznaczającą ile retransmisji zostało wykonane przez algorytm (niezależnie od tego czy zakończył się sukcesem czy porażką).
Żeby $n$-ta retransmisja została wykonana we wszystkich wcześniejszych musiała zajść kolizja ($K_1=\ldots=K_{n-1}=1$).
Uzupełnij poniższy rozkład prawdopodobieństwa (zwróć szczególną uwagę na ostatnią kolumnę!):\\
\begin{tabular}{l|p{1.3cm}|p{1.3cm}|p{1.3cm}|p{1.3cm}|p{1.3cm}|p{1.3cm}|p{1.3cm}|p{1.3cm}}
$n$ & 1 & 2 & 3 & 4 & 5 & 6 & 7 & 8 \\
\hline
$P(N=n)$ & & & & & & & & 
\end{tabular}
\vspace{1cm}\\
\ans{~\\
	\begin{tabular}{l|p{1.3cm}|p{1.3cm}|p{1.3cm}|p{1.3cm}|p{1.3cm}|p{1.3cm}|p{1.3cm}|p{1.3cm}}
		$n$ & 1 & 2 & 3 & 4 & 5 & 6 & 7 & 8 \\
		\hline
		$P(N=n)$ & $\frac{1}{2}$ & $\frac{3}{2^3}$ & $\frac{7}{2^6}$ & $\frac{15}{2^{10}}$ & $\frac{31}{2^{15}}$ & $\frac{31}{2^{20}}$ & $\frac{31}{2^{25}}$ & $\frac{1}{2^{25}}$ 
	\end{tabular}
}
%Podaj wzór ogólny: $ P(N=n)=\ldots\ldots\ldots $
\item Ile wynosi suma prawdopodobieństw w rozkładzie prawdopodobieństwa zmiennej losowej $N$? \ldots\ldots\ldots\ldots
\ans{$1$}
\item Jakie jest prawdopodobieństwo, że pakiet nie zostanie wysłany (tzn. że wszystkie retransmisje spowodują kolizję)?
\[ P(K_1=1, K_2=1, \ldots, K_8=1) = \ldots\ldots\ldots\ldots\ldots\ldots \]
\ans{
	\[ P(K_1=1, K_2=1, \ldots, K_8=1) = \frac{1}{2^{30}} \]
}
\item Ile \emph{średnio} jest wykonywanych retransmisji w toku działania algorytmu?
\[ EN = \ldots\ldots\ldots\ldots\ldots \]
\ans{
	\[ EN = 1\cdot\frac{1}{2} + 2\cdot\frac{3}{2^2} + 3\cdot\frac{7}{2^6} + 4\cdot\frac{15}{2^{15}} + 5\cdot\frac{31}{2^{15}} + 6\cdot\frac{31}{2^{20}} + 7\cdot\frac{31}{2^{25}} + 8\cdot\frac{1}{2^{25}} = \frac{55\,084\,065}{2^{25}}\approx 1{,}642 \]
}
\item Uzupełnij poniższą tabelę prawdopodobieństwem, że zostanie wykonane co najmniej $n$ retransmisji.
Zaobserwuj, że jest to równoważne stwierdzeniu, że w ciągu pierwszych $n-1$ retransmisji wystąpiła kolizja.
\begin{tabular}{l|p{1.3cm}|p{1.3cm}|p{1.3cm}|p{1.3cm}|p{1.3cm}|p{1.3cm}|p{1.3cm}|p{1.3cm}}
$n$ & 1 & 2 & 3 & 4 & 5 & 6 & 7 & 8 \\
\hline
$P(N\geq n)$ & & & & & & & & 
\end{tabular}
\vspace{1cm}\\
\begin{ansenv}
	\begin{tabular}{l|p{1.3cm}|p{1.3cm}|p{1.3cm}|p{1.3cm}|p{1.3cm}|p{1.3cm}|p{1.3cm}|p{1.3cm}}
		$n$ & 1 & 2 & 3 & 4 & 5 & 6 & 7 & 8 \\
		\hline
		$P(N\geq n)$ & 1 & $\frac{1}{2}$ & $\frac{1}{2^3}$ & $\frac{1}{2^6}$ & $\frac{1}{2^{10}}$ & $\frac{1}{2^{15}}$ & $\frac{1}{2^{20}}$ & $\frac{1}{2^{25}}$ 
	\end{tabular}
\end{ansenv}
\item Oblicz średni czas oczekiwania w poszczególnych krokach algorytmu, tj. wartości średnie $EA_n=EB_n$.
\begin{tabular}{p{1.3cm}|p{1.3cm}|p{1.3cm}|p{1.3cm}|p{1.3cm}|p{1.3cm}|p{1.3cm}|p{1.3cm}}
	$EA_1$ & $EA_2$ & $EA_3$ & $EA_4$ & $EA_5$ & $EA_6$ & $EA_7$ & $EA_8$ \\
	\hline
	& & & & & & & 
\end{tabular}
\vspace{1cm}\\
\begin{ansenv}
	\begin{tabular}{p{1.3cm}|p{1.3cm}|p{1.3cm}|p{1.3cm}|p{1.3cm}|p{1.3cm}|p{1.3cm}|p{1.3cm}}
		$EA_1$ & $EA_2$ & $EA_3$ & $EA_4$ & $EA_5$ & $EA_6$ & $EA_7$ & $EA_8$ \\
		\hline
		$\frac{1}{2}$ & $\frac{3}{2}$ & $\frac{7}{2}$ & $\frac{15}{2}$ & $\frac{31}{2}$ & $\frac{31}{2}$ & $\frac{31}{2}$ & $\frac{31}{2}$ 
	\end{tabular}
\end{ansenv}
\item Niech $W_n$ będzie czasem oczekiwania przed $n$-tą retransmisją (czyli $W_n=A_n$) o ile do $n$-tej retransmisji doszło (tzn. gdy $N\geq n$) i $0$ w przeciwnym wypadku (tzn. gdy $N<n$).
Ile wynosi \emph{średnia} wartość $W_n$?
Zauważ, że prawdopodobieństwo $P(W_n=0)$ nie gra roli w obliczaniu średniej.\\
\begin{tabular}{p{1.3cm}|p{1.3cm}|p{1.3cm}|p{1.3cm}|p{1.3cm}|p{1.3cm}|p{1.3cm}|p{1.3cm}}
$EW_1$ & $EW_2$ & $EW_3$ & $EW_4$ & $EW_5$ & $EW_6$ & $EW_7$ & $EW_8$ \\
\hline
 & & & & & & & 
\end{tabular}
\vspace{1cm}\\
Podpowiedź: dla dowolnych zmiennych losowych typu skokowego $X, Y$ zachodzi 
\[EX = \sum_{y_j} E\left(X|Y=y_j\right)P(Y=y_j) \]
\begin{ansenv}
	\[ EW_n = E(W_n|N\geq n)P(N\geq n) + E(W_n|N<n)P(N<n) = EA_nP(N\geq n)\]
	\begin{tabular}{p{1.3cm}|p{1.3cm}|p{1.3cm}|p{1.3cm}|p{1.3cm}|p{1.3cm}|p{1.3cm}|p{1.3cm}}
		$EW_1$ & $EW_2$ & $EW_3$ & $EW_4$ & $EW_5$ & $EW_6$ & $EW_7$ & $EW_8$ \\
		\hline
		$\frac{1}{2}$ & $\frac{3}{2^2}$ & $\frac{7}{2^4}$ & $\frac{15}{2^7}$ & $\frac{31}{2^{11}}$ & $\frac{31}{2^{16}}$ & $\frac{31}{2^{21}}$ & $\frac{31}{2^{26}}$
	\end{tabular}
\end{ansenv}
\item Ile \emph{średnio} jednostek czasu algorytm spędza na oczekiwaniu w linii \ref{backoff:wait}?
Innymi słowy: ile wynosi średnia sumy zmiennych losowych $W_n$:
\[ E(W_1+W_2+\ldots+W_8)=\ldots\ldots\ldots\ldots\ldots\]
\ans{\[ E(W_1+W_2+\ldots+W_8) = \frac{1}{2^{26}}\left(2^{25} + 3\cdot 2^{24} + 7\cdot 2^{22} + 15\cdot 2^{19} + 31\cdot 2^{15} + 31\cdot 2^{10} + 31\cdot 2^{5} + 31 \right) \]}
\end{enumerate}

\cleardoublepage
\section{Twierdzenia graniczne}

\begin{enumerate}
	\item W Bucklandzie na jesieni przygotowują sery na zimę. W każdym ze 100 magazynów
	zgromadzonych jest po 1000 serów. Prawdopodobieństwo zepsucia się pojedynczego
	krążka sera wynosi $4\cdot 10^{-3}$ i~nie zależy od pozostałych serów. Niech
	$i=1,2,\ldots,100$ oznacza numer magazynu i niech zmienna losowa $X_i$ oznacza
	liczbę zepsutych serów składowanych w $i$-tym magazynie. Ponadto, niech zmienna
	losowa $Y$ odpowiada sumarycznej liczbie zepsutych serów
	we wszystkich magazynach.
	
	\begin{enumerate}
		\item Podaj  (w formie funkcji prawdopodobieństwa) rozkład prawdopodobieństwa zmiennej losowej $X_1$.
		\item Oblicz (z dokładnością do dwóch miejsc po przecinku) prawdopodobieństwo, że w magazynie nr 10 zepsują się mniej niż trzy sery.
		\item Podaj średnią liczbę zepsutych serów w magazynie nr 15.
		\item Oblicz średnią liczbę zepsutych serów łącznie we wszystkich magazynach w Bucklandzie.
		\item Oblicz wartość $DY$.
		\item Oblicz prawdopodobieństwo, że w Bucklandzie zepsują się więcej niż 424 sery.
		\item Oblicz prawdopodobieństwo, że w Bucklandzie zepsuje się pomiędzy 385, a 421 serów.
	\end{enumerate}

	\ans{
		\begin{enumerate}
			\item \[ P(X_1=k) = {1000 \choose k} (4\cdot 10^{-3})^k(1-4\cdot 10^{-3})^{1000-k} \]
			\item Przybliżenie rozkładem Poissona $\lambda=1000\cdot4\cdot 10^{-3}=4$
			\[ P(X_{10}<3) = P(X_{10}\in\{0,1,2\}) = 0{,}0183+0{,}0733+0{,}1465=0{,}2381 \]
			\item $EX_{15}=1000\cdot4\cdot 10^{-3}=4$
			\item $Y$ jest sumą 100 zmiennych o rozkładzie $B(1000, 4\cdot 10^{-3})$, zatem samo ma rozkład dwumianowy: \mbox{$B(100\cdot 1000, 4\cdot 10^{-3})$}
			\[ EY=100000\cdot4=400 \]
			\item \[ DY = \sqrt{100000\cdot 4\cdot 10^{-3} \cdot (1-4\cdot 10^{-3})} \approx 20  \]
			\item Korzystamy z twierdzenia Moivre'a-Laplace'a i przybliżamy $Y$ rozkładem normalnym $N(400, 20^2 )$. $\Phi$ oznacza dystrybuantę $N(0,1)$
			\[ P(Y>424) = 1-F_{Y}(424) = \text{standaryzacja} = 1-\Phi(\frac{424-400}{20}) = 1-\Phi(1{,}2) = 1-0{,}885=0{,}115  \]
			\item Analogicznie; Z oznacza zmienną z $N(0,1)$
			\begin{align*}
			P(385\leq Y\leq 421) = P(\frac{385-400}{20} \leq Z \leq \frac{421-400}{20} ) = P(-0{,}75 \leq Z \leq 1{,}05) = \\ \Phi(1{,}05) - (1-\Phi(0{,}75)) = \Phi(1{,}05) + \Phi(0{,}75) - 1 = 0{,}8513 + 0{,}7734 - 1 = 0{,}625
			\end{align*}
		\end{enumerate}
	}

	\item Dane są dwie kostki sześciościenne, na każdej znajdują się cyfry $1, 2, \ldots, 6$.
	Jedna z kostek jest uczciwa, tzn. każdy wynik jest równoprawdopodobny.
	Druga kostka ma różne prawdopodobieństwa uzyskania poszczególnych wyników, ale identyczną wartość średnią i odchylenie standardowe jak pierwsza kostka.
	Ile minimalnie należy wykonać rzutów obiema kostkami, żeby z prawdopodobieństwem przynajmniej $95\%$ różnica między średnimi arytmetycznymi uzyskanych wyników była nie większa niż $0{,}1$?
	
	\ans{
		Niech ciąg $(A_i)$ oznacza wyniki w $i$-tym rzucie na uczciwej kostce, a ciąg $(B_i)$ wyniki w $i$-tym rzucie na nieuczciwej kostce.
		Z konstrukcji doświadczenia (kolejne rzuty kostkami) wnioskujemy, że zmienne $A_i$ i $B_i$ są niezależne (i w obrębie ciągu, i między ciągami).
		Obliczamy wartości średnie i wariancje:
		\[ EA_i=EB_i=\frac{7}{2} \qquad D^2A_i=D^2B_i=\frac{35}{12} \]
		Niech $\overline{A_n}$ i $\overline{B_n}$ oznacza średnie arytmetyczne odpowiednio na uczciwej i nieuczciwej kostce aż do $n$-tego rzutu:
		\[ \overline{A_n} = \frac{1}{n}\sum_{i=1}^n A_i \qquad  \overline{B_n} = \frac{1}{n} \sum_{i=1}^n B_i \]
		Dochodzimy do ciągu zmiennych opisujących różnicę między średnimi arytmetycznymi:
		\[ \overline{X_n} = \overline{A_n}-\overline{B_n} = \frac{1}{n}\sum_{i=1}^n A_i-\frac{1}{n}\sum_{i=1}^n B_i \]
		Przekształcamy łącząc sumy:
		\[ \overline{X_n} = \frac{1}{n}\sum_{i=1}^n \left(A_i-B_i\right) \]
		Rozpatrzmy ciąg $X_i=A_i-B_i$: zmienne $X_i$ mają identyczny rozkład prawdopodobieństwa, $EX_i=EA_i-EB_i=0$, $D^2X_i=D^2A_i+D^2B_i=\frac{35}{6}$ (z niezależności zmiennych $A_i$ oraz $B_i$).
		Zatem dla ciągów $(X_i)$ oraz $(\overline{X_n})$ spełnione są warunki twierdzenia Lindenberga-Levy'ego.
		Niech w takim razie \[U_n=\frac{\overline{X_n}-0}{\sqrt{\frac{35}{6}}}\sqrt{n}\] stanowi zmienną losową zbieżną wg dystrybuant do $N(0,1)$.
		Z treści zadania:
		\[ P(\left|\overline{X_n}\right| \leq 0{,}1) \geq 0{,}95 \]
		Przekształcamy nierówność w funkcji prawdopodobieństwa mnożąc stronami:
		\[
		P\left(\left|\frac{\overline{X_n}-0}{\sqrt{\frac{35}{6}}}\sqrt{n}\right| \leq \frac{0{,}1-0}{\sqrt{\frac{35}{6}}}\sqrt{n}\right) \geq 0{,}95 \\		
		\]
		Dla czytelności oznaczmy prawą stronę nierówności przez $\xi_n = \frac{0{,}1-0}{\sqrt{\frac{35}{6}}}\sqrt{n}$ i skorzystajmy z definicji zmiennej $U_n$:
		\[ P\left( \left|U_n\right| \leq \xi_n \right) \geq 0{,}95 \]
		Przepisujemy lewą stronę jako różnicę dystrybuant:
		\[ P\left( \left|U_n\right| \leq \xi_n \right) = P\left( -\xi_n \leq U_n \leq \xi_n \right) = F_{U_n}(\xi_n)-F_{U_n}(-\xi_n) = 2F_{U_n}(\xi_n)-1 \geq 0{,}95 \]
		Upraszczamy nierówność:
		\[ F_{U_n}(\xi_n) \geq \frac{1+0{,}95}{2}=0{,}975 \]
		Korzystamy z twierdzenia Lindenberga-Levy'ego i przybliżamy $F_{U_n}$ przez $\Phi$, tj. dystrybuantę standaryzowanego rozkładu normalnego.
		\[ \Phi(\xi_n) \geq 0{,}975 \]
		Za pomocą tablicy z dystrybuantą standaryzowanego rozkładu normalnego odszukujemy minimalną wartość argumentu, dla której powyższa nierówność zachodzi:
		\[ \xi_n \geq 1{,}96 \]
		Podstawiamy $\xi_n$ i rozwiązujemy:
		\begin{gather*}
			\frac{0{,}1-0}{\sqrt{\frac{35}{6}}}\sqrt{n} \geq 1{,}96 \\
			\sqrt{n} \geq 1{,}96\frac{\sqrt{\frac{35}{6}}}{0{,}1} = 19{,}6\sqrt{\frac{35}{6}} \\
			n \geq (19{,}6)^2 \cdot \frac{35}{6} \approx 2240{,}9
		\end{gather*}
		Wnioskujemy z tego, że minimalną wartością $n$ jest 2241.
		
	}	
	
	\item Dany jest ciąg niezależnych zmiennych losowych $X_1, X_2, \ldots$ taki, że  $X_n$ pochodzi z rozkładu jednostajnego na przedziale $\{1, \ldots, n\}$.
	Niech $Y_n=\frac{X_n}{n}$.
	Zbadaj czy dla dużych wartości $n$ zachodzi:
	\[ \frac{1}{n} \sum_{i=1}^n Y_n \approx \frac{1}{n} \sum_{i=1}^n EY_n \]
	
	\ans{
		Obliczmy wariancje zmiennych losowych $X_n$:
		\begin{gather*}
		EX_n = \sum_{i=1}^n \left(i\cdot \frac{1}{n}\right) = \frac{1}{n} \frac{n(n+1)}{2} = \frac{n+1}{2} \\
		EX_n^2 = \sum_{i=1}^n \left(i^2\cdot \frac{1}{n}\right) = \frac{1}{n} \frac{n(n+1)(2n+1)}{6} = \frac{(n+1)(2n+1)}{6} \\
		D^2X_n = EX_n^2-(EX_n)^2 = \frac{(n+1)(2n+1)}{6}-\frac{(n+1)^2}{4}=\frac{n^2-1}{12}
		\end{gather*}
		Rozważmy następnie wariancje zmiennych losowych $Y_n$:
		\begin{gather*}
		D^2Y_n = D^2\left(\frac{X_n}{n}\right) = \frac{D^2X_n}{n^2}=\frac{n^2-1}{12n^2}=\frac{1}{12}-\frac{1}{12n^2} \leq \frac{1}{12}
		\end{gather*}
		Zatem wartości $D^2Y_n$ są ograniczone z góry przez wspólną stałą ($\frac{1}{12}$).
		Spełnione są warunki mocnego prawa wielkich liczb Czebyszewa, a zatem
		\[ P\left(\lim_{n\to\infty} \frac{1}{n} \sum_{i=1}^n Y_n - \frac{1}{n} \sum_{i=1}^n EY_n =0 \right) =1 \]
		i możemy powiedzieć, że
		\[ \frac{1}{n} \sum_{i=1}^n Y_n \approx \frac{1}{n} \sum_{i=1}^n EY_n  \]
	}
\end{enumerate}

\vfill
\subsection*{Ściąga}
\begin{description}
	\item[Mocne prawo wielkich liczb Chinczyna]
	Niech $X_1, X_2, \ldots$ jest ciągiem niezależnych zmiennych losowych o tym samym rozkładzie o skończonej wariancji i wartości oczekiwanej $\mu$. Zachodzi:
	\[ P\left( \lim_{n\to\infty} X_n = \mu \right) = 1 \]
	\item[Mocne prawo wielkich liczb Czebyszewa]
	Niech $X_1, X_2, \ldots$ jest ciągiem niezależnych zmiennych losowych takich, że wariancje są ograniczone wspólną stałą.
	Niech $\overline{X_n}=\frac{1}{n}\sum_{i=1}^{n} X_i$ będzie średnią arytmetyczną pierwszych $n$ wyrazów tego ciągu, a $\overline{\mu_n}=\frac{1}{n}\sum_{i=1}^n EX_i$ średnią arytmetyczną ich wartości oczekiwanych. Zachodzi:
	\[ P\left( \lim_{n\to\infty} \overline{X_n}-\overline{\mu_n} =0 \right)) =1 \]
	\item[Wniosek z twierdzenia Moivre'a--Laplace'a]~\\
	Jeżeli $np\geq 5$ oraz $np(1-p)\geq 5$, to $B(n,p)\approx N(np, np(1-p)))$
	\item[Twierdzenia Lindeberga-Levy'ego]
	Niech $X_1, X_2, \ldots$ będzie ciągiem niezależnych zmiennych losowych o tym samym rozkładzie, takim, że $EX_i=\mu$ i $DX_i=\sigma$.
	Niech $\overline{X_n}=\frac{1}{n}\sum_{i=1}^{n} X_i$ będzie średnią arytmetyczną pierwszych $n$ wyrazów tego ciągu.
	Zmienna losowa $U_n=\frac{\overline{X_n}-\mu}{\sigma}\sqrt{n}$ zbiega wg dystrybuant do $N(0,1)$:
	\[ U_n=\frac{\overline{X_n}-\mu}{\sigma}\sqrt{n} \approx N(0,1) \]
\end{description}

\cleardoublepage
\section{Powtórka}
\begin{enumerate}
	\item W Bucklandzie na jesieni przygotowują sery na zimę. W każdym ze 100 magazynów
	zgromadzonych jest po 1000 serów. Prawdopodobieństwo zepsucia się pojedynczego
	krążka sera wynosi $4\cdot 10^{-3}$ i~nie zależy od pozostałych serów. Niech
	$i=1,2,\ldots,100$ oznacza numer magazynu i niech zmienna losowa $X_i$ oznacza
	liczbę zepsutych serów składowanych w $i$-tym magazynie. Ponadto, niech zmienna
	losowa $Y$ odpowiada sumarycznej liczbie zepsutych serów
	we wszystkich magazynach.
	
	\begin{enumerate}
		\item Podaj  (w formie funkcji prawdopodobieństwa) rozkład prawdopodobieństwa zmiennej losowej $X_1$.
		\item Oblicz (z dokładnością do dwóch miejsc po przecinku) prawdopodobieństwo, że w magazynie nr 10 zepsują się mniej niż trzy sery.
		\item Podaj średnią liczbę zepsutych serów w magazynie nr 15.
		\item Oblicz średnią liczbę zepsutych serów łącznie we wszystkich magazynach w Bucklandzie.
		\item Oblicz wartość $DY$.
		\item Oblicz prawdopodobieństwo, że w Bucklandzie zepsują się więcej niż 424 sery.
		\item Oblicz prawdopodobieństwo, że w Bucklandzie zepsuje się pomiędzy 385, a 421 serów.
	\end{enumerate}

\item Ted Cotton po pracy udaje się do kasyna w Bree, które niedawno wprowadziło następującą grę: gracz rozpoczyna z jednym punktem i rzuca uczciwą kostką sześciościenną.
Jeżeli wypadnie 1, gracz traci wszystkie punkty i kończy grę.
Jeżeli wypadnie 2, 3 lub 4, gracz podwaja liczbę punktów i kończy grę.
Jeżeli wypadnie 5 lub 6, gracz podwaja liczbę punktów i może ponownie rzucić kostką.
Nie ma górnego ograniczenia na liczbę rzutów ani na wysokość wygranej.
Niech zmienna losowa $X$ oznacza liczbę punktów gracza na końcu gry.

Podpowiedź: \[
\sum_{n=0}^\infty q^n = 
\begin{cases} 
\frac{1}{1-q} & \left|q\right|<1 \\
\infty & \text{wpp}
\end{cases}
\]

\begin{enumerate}
	\item Podaj zbiór punktów skokowych, tj. możliwych wartości, zmiennej losowej $X$.
	\ans{$\{0, 2, 4, \ldots, 2^{n}, \ldots\}=\{0\}\cup\{2^n\colon n\in\mathbb{N}_{+} \}$}
	\item Podaj funkcję prawdopodobieństwa $P$ zmiennej losowej $X$.
	\ans{
		\[
		P(X=k)=\begin{cases} 
		\left(\frac{1}{3}\right)^{n-1}\frac{1}{2} & k=2^n\, n\in\mathbb{N}_{+} \\
		\frac{1}{4} & k=0\quad \text{(wyprowadzenie niżej)}
		\end{cases} \]
	}
	\item Oblicz prawdopodobieństwo, że gracz skończy grę z niezerową liczbą punktów.
	\ans{\[P(X>0)=\sum_{n=1}^\infty \left(\frac{1}{3}\right)^{n-1}\frac{1}{2}=
		\frac{1}{2}\sum_{n=0}^\infty \left(\frac{1}{3}\right)^n = \frac{1}{2}\cdot \frac{1}{1-\frac{1}{3}}=\frac{3}{4} \]}
	\item Oblicz średnią liczbę punktów gracza na końcu gry.
	\ans{\[
		EX = \sum_{n=1}^\infty 2^{n}\left(\frac{1}{3}\right)^{n-1}\frac{1}{2} = \sum_{n=1}^\infty \left(\frac{2}{3}\right)^{n-1} = \sum_{n=0}^\infty \left(\frac{2}{3}\right)^n = \frac{1}{1-\frac{2}{3}} = 3
		\]}
	\item Obliczy odchylenie standardowe zmiennej losowej $X$.
	\ans{$DX=\sqrt{E(X^2)-(EX)^2}$ \\
		\[
		EX^2=\sum_{n=1}^\infty \left(2^{n}\right)^2\left(\frac{1}{3}\right)^{n-1}\frac{1}{2} =
		\frac{1}{2} \sum_{n=1}^\infty 4^{n}\left(\frac{1}{3}\right)^{n-1} =
		\frac{4}{2} \sum_{n=1}^\infty 4^{n-1}\left(\frac{1}{3}\right)^{n-1} =
		\frac{1}{2} \sum_{n=0}^\infty \left(\frac{4}{3}\right)^n \to\infty
		\]
		Wyciągamy z tego wniosek, że nie można obliczyć odchylenia standardowego tej zmiennej losowej.
	}
	\item Jaka jest najbardziej prawdopodobna wartość zmiennej losowej $X$? 
	\ans{2}
\end{enumerate}
	
	\item Czarodziej Gandalf bada starożytny krasnoludzki artefakt w kształcie urny.
	Przeznaczenie artefaktu jest niejasne, a jedynej wskazówki dostarcza umieszczona na nim inskrypcja: 
	\emph{W środku znajduje się 6 kul: 2 czarne i 4 białe albo 4 czarne i 2 białe.
		Raz na pełnię księżyca możesz wylosować jednocześnie i uczciwie 2 kule, które natychmiast po obejrzeniu musisz odłożyć z powrotem do wnętrza urny.}
	Próby wyciągnięcia wszystkich kul albo zajrzenia do urny nic nie dają.
	Gandalf jest przekonany, że napis na urnie jest prawdziwy i w urnie znajduje się jeden albo drugi zestaw kul, nie ma trzeciej możliwości.
	Podczas pełni Gandalf przeprowadził eksperyment i~wyciągnął z urny \emph{2 czarne kule}.
	W związku z tym Gandalf przypisał pierwszemu zestawowi prawdopodobieństwo $\frac{1}{7}$.
	
	\begin{enumerate}
		\item Przypisz symbole następującym zdarzeniom i wykorzystaj podczas dalszego rozwiązywania zadania:
		\begin{itemize}
			\item w urnie znajdują się 2 kule czarne i 4 kule białe; \ans{$H_1$}
			\item w urnie znajdują się 4 kule czarne i 2 kule białe; \ans{$H_2$}
			\item Gandalf wyciągnął 2 kule czarne.  \ans{$E$}
		\end{itemize}
		\item Jakie jest prawdopodobieństwo  zdarzenia \emph{w urnie znajdują się 2 kule czarne i 4 kule białe}, a jakie zdarzenia \emph{w urnie znajdują się 4 kule czarne i 2 kule białe}?
		\ans{$P(H_1)=\frac{1}{7} \qquad P(H_2)=\frac{6}{7}$}
		\item Jakie jest prawdopodobieństwo  zdarzenia \emph{Gandalf wyciągnął 2 kule czarne} pod warunkiem, że \emph{w urnie znajdują się 2 kule czarne i 4 kule białe}?
		\ans{\[P(E|H_1)=\frac{1}{{6\choose 2}}=\frac{1}{15}\]}
		\item Jakie jest prawdopodobieństwo  zdarzenia \emph{Gandalf wyciągnął 2 kule czarne} pod warunkiem, że \emph{w urnie znajdują się 4 kule czarne i 2 kule białe}?
		\ans{\[P(E|H_2)=\frac{{4\choose 2}}{{6\choose 2}}=\frac{6}{15}\]}
		\item Jakie jest prawdopodobieństwo  zdarzenia \emph{Gandalf wyciągnął 2 kule czarne}?
		\ans{\[P(E)=P(E|H_1)P(H_1)+P(E|H_2)P(H_2)=\frac{1}{15}\frac{1}{7}+\frac{6}{15}\frac{6}{7}=\frac{37}{105}\]}
	\end{enumerate}
	Podczas kolejnej pełni Gandalf przeprowadził kolejny eksperyment i ponownie wyciągnął z urny \emph{2 czarne kule}.
	\begin{enumerate}
		\setcounter{enumii}{5}
		\item Jakie jest prawdopodobieństwo, że \emph{w urnie znajdują się 2 kule czarne i 4 kule białe} w świetle wyniku eksperymentu (tj. pod warunkiem, że \emph{Gandalf wyciągnął 2 kule czarne})?
		\ans{\[P(H_1|E)=\frac{P(E|H_1)P(H_1)}{P(E)}=\frac{\frac{1}{15}\frac{1}{7}}{\frac{37}{105}}=\frac{1}{37} \]}
		\item Jakie jest prawdopodobieństwo, że \emph{w urnie znajdują się 4 kule czarne i 2 kule białe} w świetle wyniku eksperymentu (tj. pod warunkiem, że \emph{Gandalf wyciągnął 2 kule czarne})?
		\ans{\[P(H_2|E)=1-P(H_1|E)=\frac{36}{37} \]}
		\item W świetle wyniku eksperymentu, który zestaw kul w urnie jest bardziej prawdopodobny?
		\ans{Bardziej prawdopodobne jest, że w urnie znajdują się 4 kule czarne i 2 kule białe}
	\end{enumerate}

\item Ted Cotton, jeden z pracowników urzędu pocztowego w Hobbitonie, poczynił następującą obserwację: w 48 przypadkach na 50 po wyjściu klienta z urzędu następny klient przyjdzie do urzędu przed upływem 2 minut.
Co więcej, jeżeli przez 2 minuty nikt nie przyjdzie, to sytuacja się powtarza: w 48 przypadkach na 50 przez kolejne 2 minuty przyjdzie klient itd.
Z urzędu pocztowego właśnie wyszedł klient.
Niech $T$ będzie zmienną losową \emph{typu ciągłego} charakteryzującą się \emph{brakiem pamięci}, a~oznaczającą czas w minutach oczekiwania na przyjście następnego klienta.

\begin{enumerate}
	\item Podaj rozkład prawdopodobieństwa zmiennej losowej $T$: jego nazwę, parametry i~dystrybuatnę.
	\ans
	{
		Rozkład wykładniczy
		\begin{gather*}
		F(x) = \begin{cases} 1-e^{-\lambda x} & x>0 \\ 0 & \text{wpp} \end{cases} \\
		P(T<2)=F(2)=\frac{48}{50} \\
		1-e^{-2\lambda}=\frac{24}{25} \\
		e^{-2\lambda}=\frac{1}{25} \\ 
		-2\lambda = \ln \frac{1}{25} = -2 \ln 5 \\
		\lambda = \ln 5 \approx 1{,}609
		\end{gather*}
	}
	\item Podaj średni czas oczekiwania na kolejnego klienta.
	\ans{ $ET = \frac{1}{\lambda} = \frac{1}{\ln 5} \approx 0{,}621$ }
	\item Podaj odchylenie standardowe zmiennej losowej $T$.
	\ans{$DT=ET=\frac{1}{\lambda} = \frac{1}{\ln 5} \approx 0{,}621$ }
	\item Jakie jest prawdopodobieństwo, że czas oczekiwania na klienta będzie pomiędzy 1, a 3 minut?
	\ans{$P(1\leq T\leq 3) = F(3)-F(1) = 1-e^{-3\lambda}-(1-e^{-\lambda}) = e^{-\lambda}-e^{-3\lambda}=5^{-1}-5^{-3}=\frac{5^2-1}{5^3}=\frac{24}{125}=0{,}192$}
	\item Ted twierdzi, że kiedyś przez godzinę nikt nie przyszedł do urzędu pocztowego. Jakie jest prawdopodobieństwo takiego zdarzenia? Jakie są jego możliwe przyczyny?
	\ans{$P(T>60)=1-F(60)=1-(1-e^{-60\lambda}) = 5^{-60} \approx 0$}
\end{enumerate}

\end{enumerate}

\end{document}
