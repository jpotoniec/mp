\documentclass{beamer}
\usepackage{polski}
\usepackage[utf8]{inputenc}
\usetheme{Boadilla}
\title{Metody probabilistyczne I}
\author{Jędrzej Potoniec}
\date{}
\begin{document}
\begin{frame}
\titlepage
\end{frame}
\part{Organizacja przedmiotu}
\frame{\partpage}
\begin{frame}{Kontakt}
mgr inż. Jędrzej Potoniec \\
\url{Jedrzej.Potoniec@cs.put.poznan.pl}\\
\url{http://www.cs.put.poznan.pl/jpotoniec}
\end{frame}
\begin{frame}{Zasady oceniania}
Egzamin zaliczający obie części przedmiotu:
\begin{description}
\item[ćwiczenia] 75\% puntków
\item[wykład] 25\% punktów
\item[można] mieć kartkę ze wzorami i~kalkulator (bez WiFi, GSM itp)
\item[nie można] komunikować się z innymi
\item[nie trzeba] uczyć się na pamięc definicji, twierdzeń (ale trzeba je umieć stosować!)
\item[dodatkowo] punkty za aktywność
\end{description}
\end{frame}
\begin{frame}{Skala ocen}
\begin{center}
\begin{tabular}{l|r}
\% punktów & ocena \\
\hline
$\left(-\infty; 50\right]$ & 2,0 \\
$\left(50; 60\right]$ & 3,0 \\
$\left(60; 70\right]$ & 3,5 \\
$\left(70; 80\right]$ & 4,0 \\
$\left(80; 90\right]$ & 4,5 \\
$\left(90; \infty\right)$ & 5,0 \\
\end{tabular}
\end{center}
\end{frame}
\begin{frame}{Obecność}
\begin{block}{Regulamin studiów, rozdział II, par.8 pkt. 2}
Uczestnictwo w zajęciach objętych planem studiów jest obowiązkowe dla nauczycieli akademickich i~studentów.
Uczestnictwo w ćwiczeniach, zajęciach laboratoryjnych, projektowych, seminariach, pracowniach, lektoratach i~zajęciach WF jest kontrolowane przez prowadzącego.
\end{block}
{\tiny Regulamin studiów uchwalony przez Senat Akademicki Politechniki Poznańskiej Uchwałą Nr 175 z dnia 25 kwietnia 2012 r. Zmiany wprowadzone Uchwałą Nr 185 z dnia 27 czerwca 2012 r.}
\end{frame}
\begin{frame}{Inspiracje}
\begin{itemize}
\item \emph{Metody probabilistyczne} prof. J. Węglarz @ PUT
\item \emph{CS 223 -- Random Processes and Algorithms} @ Harvard \\
\url{http://www.eecs.harvard.edu/~michaelm/CS223/handouts.html}
\end{itemize}
\end{frame}
\begin{frame}{Literatura}
\begin{enumerate}
\item M. Mitzenmacher, E. Upfal \emph{Metody probabilistyczne i~obliczenia} (WNT 2009)
\item W. Krysicki, J. Bartos i~in. \emph{Rachunek prawdopodobieństwa i~statystyka matematyczna w zadaniach} (PWN 2010)
\item A. Plucińska, E. Pluciński \emph{Probabilistyka: rachunek prawdopodobieństwa, statystyka matematyczna, procesy stochastyczne} (WNT 2000)
\item Amir D. Aczel \emph{Statystyka w zarządzaniu} (PWN 2006)
\item S. Takahashi, Trend-Pro Co. Ltd. \emph{Manga Guide to Statistics} (No starch press 2008)
\item M. Heller \emph{Filozofia przypadku: kosmiczna fuga z preludium i~coda} (Copernicus Center Press 2012)
\end{enumerate}
\end{frame}
\begin{frame}{Plan wykładu}
\begin{enumerate}
\item Wstęp
\only<1>{\begin{enumerate}
\item Informacje organizacyjne
\item \emph{Zastosowania probabilistyki w informatyce}
\end{enumerate}}
\item Aksjomatyka rachunku prawdopodobieństwa
\only<2>{\begin{enumerate}
\item Przestrzeń zdarzeń elementarnych
\item Zdarzenia i działania na nich
\item Aksjomatyczna definicja prawdopodobieństwa
\item \emph{Probabilistyczne porównywanie wielomianów}
\item Prawdopodobieństwo warunkowe, zupełne i twierdzenie Bayesa
\item Niezależność zdarzeń
\item \emph{Klasyfikator Bayesa i filtry antyspamowe}
\end{enumerate}}
\item Zmienne losowe
\only<3>{
	\begin{enumerate}
	\item Pojęcie zmiennej losowej, funkcji prawdopodobieństwa i~dystrybuanty
	\item Typy zmiennych losowych
	\item Momenty zmiennych losowych, nierówności Markowa i~Czebyszewa
	\item \emph{Podstawowa metoda Monte-Carlo}
	\item Przykładowe rozkłady prawdopodobieństwa: jednostajne, dwupunktowy, dwumianowy, Poissona, geometryczny, Gaussa
	\item Problem znakowania pakietów
	\item Dwuwymiarowe zmienne losowe
	\item Kowariancja, korelacja, regresja
	\item Ciągi zmiennych losowych
	\item Twierdzenia graniczne
	\end{enumerate}
}
\item Procesy losowe
\only<4>{
	\begin{enumerate}
\item Pojęcie procesu losowego i jego opis
\item Momenty procesu losowego
\item Procesy stacjonarne
\item Procesy o przyrostach niezależnych, procesy Markowa
\item Rozkład wykładniczy
\item Proces Poissona
\item \emph{Kolejka \texttt{M|M|1}}
\end{enumerate}}
\end{enumerate}
\end{frame}
\begin{frame}{Harmonogram wykładów}
\begin{description}
\item[28.02] %TODO
\item[09.05]
\item[30.05]
\item[14.06]
\end{description}
\end{frame}
\begin{frame}{Harmonogram ćwiczeń}
\begin{description}
\item[10.05] %TODO
\item[16.05]
\item[30/31.05]
\item[20.06]
\end{description}
\end{frame}

\part{Zastosowania probabilistyki w informatyce}
\frame{\partpage}
\begin{frame}{Filtrowanie poczty}
\centering\includegraphics[width=.85\textwidth]{spam.jpg}

{\tiny ,,Spam \#Spam'' by Mike Mozart CC-BY-SA 2.0 \url{https://flic.kr/p/nm6DGC}}
\end{frame}
%TODO: load balancing?
\begin{frame}{Kryptografia}
\centering\includegraphics[width=.9\textwidth]{dsa.png}

{\tiny \url{http://nvlpubs.nist.gov/nistpubs/FIPS/NIST.FIPS.186-4.pdf} (fragmenty)}
\end{frame}
\begin{frame}{Rozpoznawanie mowy}
\centering\includegraphics[width=.8\textwidth,bb=0 50mm 220mm 220mm,clip]{siri.jpg}

{\tiny ,,Siri will you marry me LOL'' by J.S. Zolliker CC-BY-SA 2.0 \url{https://flic.kr/p/aCUCY5}}
\end{frame}
\begin{frame}{Testy interfejsu użytkownika}
\centering\includegraphics[width=.9\textwidth]{ab.png}

{\tiny \url{http://www.testsignificance.com/}}
\end{frame}
\begin{frame}{Kompresja i kodowanie}
\centering\includegraphics[width=.7\textwidth]{morse.png}

{\tiny By Rhey T. Snodgrass \& Victor F. Camp, 1922 [Public domain], via Wikimedia Commons}
\end{frame}

%TODO: big data?

\part{Aksjomatyka rachunku prawdopodobieństwa}
\frame{\partpage}
\begin{frame}{Przestrzeń zdarzeń elementarnych}
\begin{itemize}
\item 
\end{itemize}
\end{frame}
\end{document}

