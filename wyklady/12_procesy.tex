\documentclass{mp}
\usetikzlibrary{shadings}
\graphicspath{{12_procesy}}
\subtitle{Procesy losowe}
\begin{document}
\frame{\titlepage}
\begin{frame}{Sumaryczny ruch w punkcie wymiany ruchu \emph{Thinx}}
\uncover<3->{\[\{X_t(\omega)\colon t\in T, \omega\in\Omega\} \quad T\subseteq \R \quad \uncover<4-5>{\only<4>{\alert{t=const}}\only<5>{\alert{\omega=const}}}\]}
\center
\begin{tikzpicture}[x=.1cm,y=2cm]
\input{12_procesy/thinx.tex}
\end{tikzpicture}

{\tiny Dane pochodzą z \url{http://www.thinx.pl/siec/statystyki_ruchu}}
\note<1>
{
	Dla ustalonego $t$: wartość procesu losowego \\
	Dla ustalonego $\omega$: realizacja procesu losowego \\
	Jeżeli $\left|T\right|\leq\aleph_0$, to powstaje ciąg lub wektor. Dla skupienia uwagi $T$ jest przedziałem.
}
\end{frame}
\begin{frame}{Rozkład procesu losowego}
%troche z tego jest na poczatku rozdzialu 12-tego u Borowkowa, troche jest w notatkach od profa, a troche wypracowalem sam na podstawie definicji funkcji mierzalnej
\begin{gather*}
\uncover<+->{(\Omega,\mathcal{A}_\Omega,P) \qquad \{X(t,\omega)\colon t\in T, \omega\in\Omega\} \\}
\uncover<+->{\delta\colon \Omega\to \Omega' \qquad \delta(\omega)=X(\cdot,\omega) \\}
\uncover<+->{\mathcal{A}_{\Omega'}=\{A\subseteq \Omega' \colon \uncover<+->{\delta^{-1}(A)=\{\omega\in\Omega\colon X(\cdot,\omega)\in A\}\in\mathcal{A}_\Omega} \} \\}
\uncover<+->{P'\colon \mathcal{A}_{\Omega'}\to \left<0,1\right> \qquad \forall A\in \mathcal{A}_{\Omega'}\colon P'(A)=P(\delta^{-1}(A)) \\}
\uncover<+->{(\Omega',\mathcal{A}_{\Omega'},\alert{P'})}
\end{gather*}
\end{frame}
\begin{frame}{$n$-wymiarowy rozkład procesu losowego}
\begin{gather*}
\uncover<+->{t_1,t_2,\ldots,t_n \in T \\}
\uncover<+->{\left(X_{t_1},X_{t_2},\ldots,X_{t_n}\right) \\}
\uncover<+->{F(x_1,t_1; x_2,t_2; \ldots; x_n,t_n)=P(X_{t_1}<x_1,X_{t_2}<x_2,\ldots,X_{t_n}<x_n) }
\end{gather*}
\uncover<+->
{
	%Borowkow, str. 189 i dodatek 2 + notatki
\begin{block}{Twierdzenie (wniosek z twierdzenia Kołmogorowa o~rozkładach zgodnych)}
Znajomość wszystkich $n$-wymiarowych rozkładów dla dowolnego, skończonego $n$ potrzeba i wystarcza do pełnego scharakteryzowania procesu losowego.
\end{block}
}
\end{frame}
\begin{frame}{Momenty jednowymiarowe}
\[ \uncover<2->{\mu(t)=E\left(X_t\right)} \qquad \uncover<3->{\sigma(t)=D(X_t)=\sqrt{E\left((X_t-\mu(t))^2\right)}} \]
\center
\begin{overprint}
\onslide<1> \only<1>{\includegraphics[width=\textwidth]{12_procesy/mean-0.png}}
\onslide<2> \only<2>{\includegraphics[width=\textwidth]{12_procesy/mean-1.png}}
\onslide<3> \only<3>{\includegraphics[width=\textwidth]{12_procesy/mean-2.png}}
\end{overprint}
\end{frame}
\begin{frame}{Momenty dwuwymiarowe}
\note<1>{$R_{XX}$ funkcja autokorelacji; $r_{XX}$ unormowana funkcja autokorelacji albo współczynnik korelacji}
\[R_{XX}(t_1,t_2)=E\left[(X_{t_1}-\mu(t_1))(X_{t_2}-\mu(t_2))\right] \qquad r_{XX}(t_1,t_2)=\frac{R_{XX}(t_1,t_2)}{\sigma(t_1)\sigma(t_2)} \]
\center
\includegraphics[width=.5\textwidth]{12_procesy/corr.png}
\end{frame}
\begin{frame}{Wektory procesów losowych}
\[ \left[X_t,Y_t\right] \]
\begin{description}
\item<2->[funkcja korelacji wzajemnej] \[ R_{XY}(t_1,t_2)=E\left[(X_{t_1}-\mu_X(t_1))(Y_{t_2}-\mu_Y(t_2))\right] \]
\item<3->[macierz korelacyjna] \[ \begin{bmatrix} R_{XX}(t_1,t_2) & R_{XY}(t_1,t_2) \\ R_{YX}(t_1,t_2) & R_{YY}(t_1,t_2) \end{bmatrix} \]
\end{description}
\end{frame}
\begin{frame}{Proces stacjonarny w węższym sensie}
\begin{block}{Określenie}
\only<-5>
{
Jeżeli dla dowolnych:
\begin{itemize}
\item<1-> $n\in\N$
\item<2-> $t_1,t_2,\ldots,t_n\in T$
\item<3-> $\tau\in\R\colon (t_1+\tau),(t_2+\tau),\ldots,(t_n+\tau)\in T$
\item<4-> $x_1,x_2,\ldots,x_n\in \R$
\end{itemize}
}
\uncover<5->{\[ F(x_1,t_1; x_2,t_2; \ldots, x_n,t_n)=F(x_1,t_1+\tau; x_2,t_2+\tau; \ldots; x_n,t_n+\tau) \]}
\end{block}
\only<6->
{
\begin{block}{Wnioski}
\begin{gather*}
\uncover<6->{\forall x\in\R \forall t\in T: F(x,t)=F(x)\\}
\uncover<7->{\mu(t)=\mu=const \qquad \sigma(t)=\sigma=const\\}
\uncover<8->{\forall x_1,x_2\in\R \forall t,t+\tau\in T\colon \uncover<9->{F(x_1,t; x_2,t+\tau)=F(x_1,0; x_2,\tau)=F(x_1,x_2,\tau)}\\}
\uncover<10->{\forall t,t+\tau\in T\colon R_{XX}(t,t+\tau)=R_{XX}(0,\tau)=R_{XX}(\tau)}
\end{gather*}
\end{block}
}
\end{frame}
\begin{frame}{Proces stacjonarny w szerszym sensie}
\begin{block}{Określenie}
\begin{gather*}
 \mu(t)=const<\infty \qquad \sigma(t)=const<\infty \\
 \forall t_1,t_2\in T\colon R(t_1,t_2)=R(t_2-t_1)<\infty
\end{gather*}
\end{block}
\begin{block}{Twierdzenie}
Proces stacjonarny w węższym sensie, dla którego $E(X^2_t)<\infty$, jest stacjonarny w szerszym sensie.
\end{block}
\end{frame}
\begin{frame}{Proces o przyrostach stacjonarnych}
\[ Y^\tau_t=X_{t+\tau}-X_t \qquad t\in\{t\in T\colon t+\tau\in T\} \]
\uncover<2>{Dla dowolnego $\tau\in\R$, $\{Y^\tau_t\}$ jest stacjonarny w węższym sensie.}
%TODO: przykład
\end{frame}
\begin{frame}{Proces o przyrostach niezależnych}
\begin{gather*}
\forall n\in \N\, \forall t_1<t_2<\ldots<t_n\in T\colon\\
X_{t_1},X_{t_2}-X_{t_1},\ldots,X_{t_n}-X_{t_{n-1}} \text{ są niezależne}
\end{gather*}
%TODO: przykład
\end{frame}
\begin{frame}{Proces Markowa}
\begin{block}{Określenie}
\begin{gather*}
\forall n\in \N\, \forall t_1<t_2<\ldots<t_n\in T\, \forall x_1,x_2,\ldots,x_n\in\R \colon\\
P(X_{t_n}<x_n|X_{t_{n-1}}=x_{n-1},\ldots,X_{t_1}=x_1) = P(X_{t_n}<x_n|X_{t_{n-1}}=x_{n-1})
\end{gather*}
\end{block}
\begin{block}{Twierdzenie}
Jeżeli $\{X_t\}$ jest procesem o przyrostach niezależnych i $\exists c\in\R\colon P(X_{\min T}=c)=1$, to $\{X_t\}$ jest procesem Markowa.
\end{block}
\end{frame}
\end{document}
