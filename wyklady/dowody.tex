\documentclass{mwart}
\usepackage{polski}
\usepackage[utf8]{inputenc}
\usepackage{amsmath,amssymb,amsthm}
\usepackage{geometry}
\title{Dowody wybranych twierdzeń}

\newtheorem*{theorem}{Twierdzenie}


\begin{document}
\section*{Twierdzenie Bayesa}
\begin{theorem}
Jeżeli zdarzenia $A_1, A_2, \ldots$ tworzą podział przestrzeni $\Omega$, $P(A_i)>0$ dla wszystkich $i=1,2,\ldots$ oraz $B\subseteq\Omega$ jest zdarzeniem, dla którego $P(B)>0$, to słuszny jest następujący wzór:
\[ P(A_i|B) = \frac{P(A_i)\cdot P(B|A_i)}{\sum_{j=1}^\infty \left[ P(A_j)\cdot P(B|A_j)\right]} \qquad \text{dla wszystkich } i=1,2,\ldots \]
\end{theorem}

\begin{proof}
Ropczynamy od prawej strony i systematycznie ją przekształcając dążymy do uzyskania lewej strony.
\[ \frac{P(A_i)\cdot P(B|A_i)}{\sum_{j=1}^\infty \left[ P(A_j)\cdot P(B|A_j)\right]} \]
Przekształcamy prawdopobieństwo warunkowe z definicji: 
\[ \frac{P(A_i)\cdot \frac{P(B\cap A_i)}{P(A_i)}}{\sum_{j=1}^\infty \left[ P(A_j)\cdot \frac{P(B\cap A_j)}{P(A_j)}\right]}
\]
Skracamy ułamki:
\[
 \frac{P(B\cap A_i)}{\sum_{j=1}^\infty P(B\cap A_j)}
\]
Skoro zdarzenia $A_j$ stanowią podział przestrzeni, to znaczy, że są rozłączne, a~w~takim razie iloczyny $B\cap A_j$ również są rozłączne, a zatem korzystając z~III~aksjomatu Kołmogorowa możemy przepisać w mianowniku sumę prawdopodobieństw jako prawdopodobieństwo sumy:
\[
 \frac{P(B\cap A_i)}{P\left(\bigcup_{j=1}^\infty [B\cap A_j]\right)}
\]
Korzystając z rozdzielności iloczynu względem sumy możemy wyciągnąć $B$ przed znak sumy:
\[
 \frac{P(B\cap A_i)}{P\left(B \cap \left[\bigcup_{j=1}^\infty A_j\right]\right)}
\]
Zdarzenia $A_j$ stanowią podział przestrzeni, a zatem sumują się do zdarzenia pewnego $\Omega$, a w takim razie:
\[
 \frac{P(B\cap A_i)}{P\left(B \cap \Omega\right)}
\]
Ponieważ $B\subseteq\Omega$, w takim razie $B\cap\Omega=B$ i otrzymujemy:
\[
 \frac{P(B\cap A_i)}{P\left(B\right)}
\]
Korzystając z przemienności iloczynu zdarzeń zamieniamy kolejność w iloczynie w liczniku:
\[
 \frac{P(A_i\cap B)}{P\left(B\right)}
\]
Otrzymaliśmy definicję prawdopodobieństwa warunkowego
\[ P(A_i|B) \]
czyli lewą stronę równania podanego w twierdzeniu.
\end{proof}

\clearpage
\section*{Wartość najbardziej prawdopodobna (modalna) w rozkładzie Bernouliego}
\begin{theorem}
Niech zmienna losowa $X$ ma rozkład Bernouliego o parametrach $n$ oraz $p$.
Jeżeli $(n+1)p$ jest liczbą całkowitą, to zmienna losowa $X$ ma dwa punkty skokowe o największym prawdopodobieństwie: $(n+1)p$ oraz $(n+1)p-1$.
W przeciwnym wypadku, tzn. gdy $(n+1)p$ nie jest liczbą całkowitą, jest dokładnie jeden taki punkt skokowy: $\lfloor (n+1)p\rfloor$
\end{theorem}

\begin{proof}

Załóżmy, że $k\in\{1,2,\ldots,n\}$ i rozważmy nierówność 
\begin{equation}
P(X=k-1)\leq P(X=k) \label{moda1} 
\end{equation}
\begin{gather*}
P(X=k-1)\leq P(X=k) \\
{n \choose k-1}p^{k-1}(1-p)^{n-(k-1)} \leq {n \choose k}p^k(1-p)^{n-k} \\
\frac{n!}{(k-1)!(n-(k-1))!}p^{k-1}(1-p)^{n-(k-1)} \leq \frac{n!}{k!(n-k)!}p^k(1-p)^{n-k} \qquad \bigg/ :\frac{n!}{(k-1)!(n-k)!}p^{k-1}(1-p)^{n-k} \\
\frac{1-p}{n-k+1} \leq \frac{p}{k} \qquad \bigg/ \cdot k(n-k+1) \\
(1-p)k \leq p(n-k+1) \\
k-pk \leq pn-pk+p \\
k \leq p(n+1)
\end{gather*}
Zatem żeby nierówność \ref{moda1} była spełniona potrzeba i wystarcza by $k\leq p(+1)$.

Teraz załóżmy, że $k\in\{0,1, \ldots, n-1\}$ i rozważmy nierówność
\begin{equation}
P(X=k)\geq P(X=k+1) \label{moda2}
\end{equation}
\begin{gather*}
P(X=k)\geq P(X=k+1) \\
{n \choose k}p^k(1-p)^{n-k} \geq {n \choose k+1}p^{k+1}(1-p)^{n-(k+1)}  \\
\frac{n!}{k!(n-k)!}p^k(1-p)^{n-k} \geq \frac{n!}{(k+1)!(n-(k+1))!}p^{k+1}(1-p)^{n-(k+1)} \qquad \bigg/ :\frac{n!}{k!(n-k-1)!}p^k(1-p)^{n-k-1}  \\
\frac{1-p}{n-k} \geq \frac{p}{k+1} \qquad \bigg/ \cdot (n-k)(k+1) \\
(1-p)(k+1) \geq p(n-k) \\
k+1-pk-p \geq pn-pk \\
k \geq p(n+1)-1
\end{gather*}
W takim razie, żeby nierówność \ref{moda2} była spełniona potrzeba i wystarcza by $k\geq p(n+1)-1$

Rozpatrzymy trzy przypadki:
\begin{enumerate}
\item Jeżeli zachodzi $P(X=0)\geq P(X=1)\geq \ldots \geq P(X=n)$ to znaczy, że największe prawdopodobieństwo jest w punkcie $0$.
Na mocy nierówności \ref{moda2} to jest równoważne $0\geq p(n+1)-1$, a zatem $1\geq p(n+1)$, czyli $\lfloor p(n+1) \rfloor=0$ i otrzymujemy wzór z twierdzenia.
%TODO tu cos dopisac, bo ten przeskok jest zbyt gwaltowny
\item Jeżeli zachodzi $P(X=0)\leq P(X=1)\leq \ldots \leq P(X=n)$ to znaczy, że największe prawdopodobieństwo jest w punkcie $n$.
Na mocy nierówności \ref{moda1} to jest równoważne $n\leq p(n+1)$, czyli $\lfloor p(n+1) \rfloor=n$ i otrzymujemy wzór z twierdzenia.
\item Rozważmy pozostałe przypadki, tzn. gdy co najmniej jeden z najbardziej prawdopodobnych punktów skokowych jest w zbiorze $\{1,2,\ldots,n-1\}$. Możemy więc rozpatrzyć łącznie nierówności \ref{moda1} i \ref{moda2}, otrzymując
\begin{equation}
p(n+1)-1 \leq k \leq p(n+1) \label{moda3}
\end{equation} 
Jeżeli $p(n+1)$ nie jest liczbą całkowitą, to nierówność \ref{moda3} ma dokładnie jedno rozwiązanie $k=\lfloor p(n+1)\rfloor$.
W przeciwnym razie rozwiązania są dwa: $k=p(n+1)$ oraz $k=p(n+1)-1$, czyli tak jak podano w twierdzeniu.
\end{enumerate}
\end{proof}
\end{document}